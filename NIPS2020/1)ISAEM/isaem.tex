\documentclass[11pt]{article}
%\usepackage{fullpage,graphicx,algorithm,algorithmic,bm,amsmath,amsthm,amssymb,color,hyperref,cite,natbib}

% if you need to pass options to natbib, use, e.g.:
%\PassOptionsToPackage{numbers}{natbib}
\usepackage{natbib,fullpage}
\usepackage{bm,amsmath,amsthm,amssymb,multicol,algorithmic,algorithm,enumitem}
\usepackage{color,wrapfig}
\usepackage[textwidth=1cm,textsize=footnotesize]{todonotes}
% ready for submission
\usepackage{neurips_2020}

\usepackage{lipsum}
\usepackage[colorlinks=true,
linkcolor=red,
urlcolor=blue,
citecolor=blue]{hyperref}

\usepackage{xargs}
\usepackage{stmaryrd}

\setlength{\parskip}{.2cm}

\newtheorem{Fact}{Fact}
\newtheorem{Lemma}{Lemma}
\newtheorem{Prop}{Proposition}
\newtheorem{Theorem}{Theorem}
\newtheorem{Def}{Definition}
\newtheorem{Corollary}{Corollary}
\newtheorem{Conjecture}{Conjecture}
\newtheorem{Property}{Property}
\newtheorem{Observation}{Observation}
%\theorembodyfont{\rmfamily}
\newtheorem{Exa}{Example}
\newtheorem{assumption}{H\!\!}
\newtheorem{assumptionA}{G\!\!}
\newtheorem{Remark}{Remark}
\newtheorem*{Lemma*}{Lemma}
\newtheorem*{Theorem*}{Theorem}
 \makeatletter
\renewenvironment{proof}[1][\proofname]{%
   \par\pushQED{\qed}\normalfont%
   \topsep6\p@\@plus6\p@\relax
   \trivlist\item[\hskip\labelsep\bfseries#1]%
   \ignorespaces
}{%
   \popQED\endtrivlist\@endpefalse
}
\makeatother
\usepackage{shortcuts_OPT}

%\renewcommand{\textwidth}{5.5in}

% Here's the definition of Sb, stolen from amstex
    \makeatletter
    \def\multilimits@{\bgroup
  \Let@
  \restore@math@cr
  \default@tag
 \baselineskip\fontdimen10 \scriptfont\tw@
 \advance\baselineskip\fontdimen12 \scriptfont\tw@
 \lineskip\thr@@\fontdimen8 \scriptfont\thr@@
 \lineskiplimit\lineskip
 \vbox\bgroup\ialign\bgroup\hfil$\m@th\scriptstyle{##}$\hfil\crcr}
    \def\Sb{_\multilimits@}
    \def\endSb{\crcr\egroup\egroup\egroup}
\makeatother

\newtheoremstyle{t}         %name
    {\baselineskip}{2\topsep}      %space above and below
    {\rm}                   %Body font
    {0pt}{\bfseries}  %Heading indent and font
    {}                      %after heading
    { }                      %head after space
    {\thmname{#1}\thmnumber{#2}.}

\theoremstyle{t}
\newtheorem{q}{Q}
\parindent=0pt

\makeatletter
\DeclareRobustCommand*\cal{\@fontswitch\relax\mathcal}
\makeatother

\newcommand{\tcr}[1]{\textcolor{red}{#1}}
\newcommand{\mathbbm}[1]{\text{\usefont{U}{bbm}{m}{n}#1}}

\begin{document}
\title{Fast Two-Time-Scale Noisy EM Algorithms}
\author{
  Belhal Karimi \\
  Cognitive And Computing Lab\\
  Baidu Research\\
  Beijing, China \\
  \texttt{belhal.karimi@baidu.com} 
   \And
  Ping Li \\
  Cognitive And Computing Lab\\
  Baidu Research\\
  Beijing, China \\
  \texttt{liping@baidu.com} }
\date{\today}

\maketitle

\begin{abstract}
T.B.C
\end{abstract}


\section{Introduction}
We formulate the following empirical risk minimization problem as:
\beq \label{eq:em_motivate}
\min_{ \param \in \Param }~ \overline{\calL} ( \param ) \eqdef \Pen (\param) + \calL ( \param )~~\text{with}~~\calL ( \param ) = \frac{1}{n} \sum_{i=1}^n \calL_i( \param) \eqdef  \frac{1}{n} \sum_{i=1}^n \big\{ - \log g( y_i ; \param ) \big\}\eqs,
\eeq
where $\{y_i\}_{i=1}^n$ are the observations, $\Param$ is a convex subset of $\rset^d$ for the parameters,  $\Pen : \Param \rightarrow \rset$ is a smooth convex regularization function   and for each $\param \in \Param$, $g(y;\param)$ is the (incomplete) likelihood of each individual  observation.
The objective function $ \overline{\calL} ( \param )$ is possibly \emph{non-convex} and is assumed to be lower bounded $ \overline{\calL} ( \param ) > - \infty$ for all $\param \in \Param$.

%We assume that $ \overline{\calL} ( \param ) > - \infty$ for all $\param \in \Param$.
In the latent variable model,  $g(y_i ; \param)$, is the marginal of the
complete data likelihood defined as $f(z_i,y_i; \param)$, i.e. $g(y_i; \param) = \int_{\Zset} f (z_i,y_i;\param) \mu(\rmd z_i)$, where $\{ z_i \}_{i=1}^n$ are the (unobserved)
latent variables.   
We make the assumption of a complete model belonging to the curved exponential family, \ie
\beq \label{eq:exp}
f(z_i,y_i; \param) = h  (z_i,y_i) \exp \big( \pscal{S(z_i,y_i)}{\phi(\param)} - \psi(\param) \big)\eqs,
\eeq
where $\psi(\param)$, $h(z_i,y_i)$ are scalar functions, $\phi(\param) \in \rset^k$ is a vector function, and $S(z_i,y_i) \in \rset^k$ is the complete data sufficient statistics.

\paragraph{Prior Work} 
Cite Kuhn \citep{kuhn2019properties} (for ISAEM) and incremental EM like papers.
As well as Optim papers (Variance reduction, SAGA etc.)


Describe the main contributions of the paper as bullet points

Then small paragraph on the structure.

\paragraph{Some Notations} 


\section{Two-Time-Scale Stochastic EM Algorithms}
Full batch EM is a two steps procedure. The {\sf E-step} amounts to computing the conditional expectation of the complete data sufficient statistics, 
\begin{equation}
\label{eq:definition-overline-bss}
\overline{\bss}(\param)= \frac{1}{n} \sum_{i=1}^n \overline{\bss}_i(\param) \quad  \text{where}  \quad \overline{\bss}_i(\param)= \int_{\Zset} S(z_i,y_i) p(z_i|y_i;\param) \mu(\rmd z_i) \,.
\end{equation}
The {\sf M-step} is given by
\beq \label{eq:mstep}
\textsf{M-step:}~~\hat{\param} = \overline{\param}( \overline{\bss}(\param) ) \eqdef \argmin_{ \vartheta \in \Param } ~\big\{ \Pen( \vartheta ) + \psi( \vartheta) - \pscal{ \overline{\bss}(\param)}{ \phi ( \vartheta) } \big\},
\eeq

\subsection{Monte Carlo Integration and Stochastic Approximation} \label{sec:sEM}
For complex and possibly nonlinear models, the expectation under the posterior distribution defined in \eqref{eq:definition-overline-bss} is not tractable. In that case, the first solution involves computing a Monte Carlo integration of that latter term. 
For all $ i \in \inter$, draw for $m \in \llbracket 1, M \rrbracket$, samples $z_{i,m} \sim p(z_i|y_i;\theta)$ and compute the MC integration $\tilde{\bss}$ of the deterministic quantity $\overline{\bss}(\param)$:
\beq\label{eq:mcstep}
\textsf{MC-step}:~ \tilde{\bss} = \frac{1}{n} \sum_{i=1}^n\frac{1}{M} \sum_{m=1}^M S(z_{i,m}, y_i)
\eeq
and compute $\hat{\param} = \overline{\param}( \hat{\bss} ) $.

This algorithm bypasses the intractable expectation issue but is rather computationally expensive in order to reach point wise convergence ($M$ needs to be large).

As a result, an alternative to that stochastic algorithm is to use a Robbins-Monro (RM) type of update.
We denote
\beq\label{eq:stats}
\tilde{S}^{(k+1)} = \frac{1}{n} \sum_{i=1}^n \tilde{S}^{(k+1)}_i = \frac{1}{n} \sum_{i=1}^n\frac{1}{M} \sum_{m=1}^M S(z_{i,m}^{(k)}, y_i)
\eeq
where $z_{i,m}^{(k)} \sim p(z_i|y_i;\theta^{(k)})$.
At iteration $k$, the sufficient statistics $\hat{\bss}^{(k+1)}$ is approximated as follows:

\beq\label{eq:rmstep}
\textsf{SA-step}:~ \hat{\bss}^{(k+1)} =  \hat{\bss}^{(k)}  + \gamma_{k+1}(\tilde{S}^{(k+1)} - \hat{\bss}^{(k)} )
\eeq
where $\{ \gamma_{k} \}_{k=1}^\infty \in [0,1]$ is a sequence of decreasing step sizes to ensure asymptotic convergence.
This is called the Stochastic Approximation of the EM (SAEM), see \citep{delyon1999} and allows a smooth convergence to the target parameter.
It represents the \textit{first level} of our algorithm (needed to temper the variance and noise implied by MC integration).

In the next section, we derive variants of this algorithm to adapt of the sheer size of data of today's applications.

\subsection{Incremental and Bi-Level Inexact EM Methods} \label{sec:sEM}
Strategies to scale to large datasets include classical incremental and variance reduced variants.
We will explicit a general update that will cover those variants and that represents the \textit{second level} of our algorithm, namely the incremental update of the noisy statistics $\hat{S}^{(k)}$ inside the RM type of update.

\beq \label{eq:sestep}
\textsf{Inexact-step}:~\tilde{S}^{(k+1)} = \tilde{S}^{(k)} + \rho_{k+1} \big( \StocEstep^{(k+1)}- \tilde{S}^{(k)}  \big),
\eeq
Note $\{ \rho_{k} \}_{k=1}^\infty \in [0,1]$ is a sequence of step sizes, $\StocEstep^{(k)}$ is a proxy for $\tilde{S}^{(k)}$,
If the stepsize is equal to one and the proxy $\StocEstep^{(k)} = \hat{S}^{(k)}$, i.e., computed in a full batch manner as in \eqref{eq:stats}, then we recover the SAEM algorithm.
Also if $\rho_{k}=1$, $\gamma_{k}=1$ and $\StocEstep^{(k)} = \tilde{S}^{(k)}$, then we recover the Monte Carlo EM algorithm.

We now introduce three variants of the SAEM update depending on different definitions of the proxy $\StocEstep^{(k)}$ and the choice of the stepsize $\rho_k$.
Let $i_k \in \inter$ be a random index drawn at iteration $k$ and $\tau_i^k = \max \{ k' : i_{k'} = i,~k' < k \}$ be the iteration index where $i \in \inter$ is last drawn prior to iteration $k$.
For iteration $k \geq 0$, the \FISAEM\ method draws \emph{two} indices \emph{independently} and uniformly as $i_k, j_k \in \inter$. In addition to $\tau_i^k$ which was defined \wrt $i_k$, we define $t_j^k = \{ k' : j_{k'} = j , k' < k \}$ to be the iteration index where the sample $j \in \inter$ is last drawn as $j_k$ prior to iteration $k$. With the initialization $\overline{\StocEstep}^{(0)} = \overline{\bss}^{(0)}$, we use a slightly different update rule from SAGA inspired by \citep{reddi2016fast}. Then, we obtain:
\begin{align}
&\emph{(\ISAEM\ \citep{karimi2019non, kuhn2019properties})} & \StocEstep^{(k+1)} &= \StocEstep^{(k)} + {\textstyle \frac{1}{n}}\big( \tilde{S}_{i_k}^{(k)}  - \tilde{S}_{i_k}^{(\tau_{i_k}^k)} \big) \label{eq:isaem} \\
&\emph{(\SAEMVR\ This paper )} &\StocEstep^{(k+1)} &= \tilde{S}^{(\ell(k))} +  \big( \tilde{S}_{i_k}^{(k)}  -\tilde{S}_{i_k}^{(\ell(k))}   \big) \label{eq:vrsaem}\\
&\emph{(\FISAEM\ This paper )} &\StocEstep^{(k+1)} &= \overline{\StocEstep}^{(k)} + \big( \tilde{S}_{i_k}^{(k)}  - \tilde{S}_{i_k}^{(t_{i_k}^k)} \big) \label{eq:fisaem}\\
&    &\overline{\StocEstep}^{(k+1)} &= \overline{\StocEstep}^{(k)} + n^{-1} \big( \tilde{S}_{j_k}^{(k)}  - \tilde{S}_{j_k}^{(t_{j_k}^k)} \big).
%\emph{(SAGA-EM)} & ~~~~\StocEstep^{(k+1)} = s & \gamma_{k+1} =
\end{align}
The stepsize is set to $\rho_{k+1} = 1$ for the \ISAEM\ method; $\rho_{k+1} = \gamma$ is  constant for the \SAEMVR\ and \FISAEM\ methods.
Moreover, for \ISAEM\ we initialize with $\StocEstep^{(0)} = \tilde{S}^{(0)}$; for \SAEMVR\, we set an epoch size of $m$ and define $\ell(k) \eqdef m \lfloor k/m \rfloor$ as the first iteration number in the epoch that iteration $k$ is in. \vspace{-.2cm}


\subsection{Two-Time-Scale Noisy EM methods}
We now introduce the general method derived using the two variance reduction techniques described above.
Algorithm~\ref{alg:ttsem} leverages both levels \eqref{eq:rmstep} and \eqref{eq:sestep} in order to output a vector of fitted parameters $\hat{\param}^{(K)}$ where $K$ is some randomly chosen termination point.

The updates in \eqref{eq:twolevels} is said to have two timescales as the step sizes satisfy $\lim \limits_{k \to \infty} \gamma_k/\rho_k < 1$ such that $ \tilde{S}^{(k+1)} $  is updated at a faster timescale than $\hat{\bss}^{(k+1)}$.

\begin{algorithm}[H]
\caption{Two-Time-Scale Noisy EM methods.}\label{alg:ttsem}
  \begin{algorithmic}[1]
  \STATE \textbf{Input:} initializations $\hat{\param}^{(0)} \leftarrow 0$, $\hat{\bss}^{(0)} \leftarrow \hat{S}^{(0)}$, $K_{\sf max}$ $\leftarrow$ max.~iteration number. \STATE Set the terminating iteration number, $K \in \{0,\dots,K_{\sf max}-1\}$, as a discrete r.v.~with:\vspace{-.1cm}
  \beq \label{eq:random}
   P( K = k ) = \frac{ \gamma_{k} }{\sum_{\ell=0}^{K_{\sf max}-1} \gamma_\ell}.\vspace{-.2cm}
  \eeq
  \FOR {$k=0,1,2,\dots, K$}
  \STATE Draw index $i_k \in \inter$ uniformly (and $j_k \in \inter$ for \FISAEM).
     \STATE Compute $\hat{S}_i^{(k)}$ using the {\sf MC-step} \eqref{eq:mcstep},  for the drawn indices.
   \STATE Compute the surrogate sufficient statistics $\StocEstep^{(k+1)}$ using \eqref{eq:isaem} or \eqref{eq:vrsaem} or \eqref{eq:fisaem}.
   \STATE Compute $\hat{S}^{(k+1)}$ and $\hat{\bss}^{(k+1)}$ using respectively \eqref{eq:sestep} and \eqref{eq:rmstep}:
\beq \label{eq:twolevels}
\begin{split}
& \tilde{S}^{(k+1)} = \tilde{S}^{(k)} + \rho_{k+1} \big( \StocEstep^{(k+1)}- \tilde{S}^{(k)}  \big)\\
&  \hat{\bss}^{(k+1)} =  \hat{\bss}^{(k)}  + \gamma_{k+1}(\tilde{S}^{(k+1)} - \hat{\bss}^{(k)} )
\end{split}
\eeq
   \STATE Compute $\hat{\param}^{(k+1)}$ via the {\sf M-step} \eqref{eq:mstep}.
\ENDFOR
\STATE \textbf{Return}: $\hat{\param}^{(K)}$.
%\STATE \textbf{Return:} $\prm_t$.
  \end{algorithmic}
\end{algorithm}



\section{Global and Finite Time Analysis of the Scheme} \label{sec:main}
First, we consider the following minimization problem on the statistics space:
\beq\label{eq:em_sspace}
\min_{ {\bss} \in \Sset }~  V ( {\bss} ) \eqdef \overline\calL( \op(\bss) ) = \Pen (  \op(\bss) ) + \frac{1}{n} \sum_{i=1}^n {\cal L}_i (  \op(\bss) )
\eeq
It has been shown that this minimization problem is equivalent to the optimization problem \eqref{eq:em_motivate}, see \citep[Lemma2]{karimi2019global}

\begin{assumption}\label{ass:convset}
$\Theta$ is an open set of $\rset^d$ and the sets $\Zset, \Sset$ are measurable open sets such that:
\beq
\Sset \supset \left\{  n^{-1} \sum_{i=1}^n u_i, u_i \in {\rm conv}(\overline{\bss}_i(\param))  \right\}
\eeq
where $\overline{\bss}_i(\param)$ is defined in \eqref{eq:definition-overline-bss}.
\end{assumption}

\begin{assumption}\label{ass:expected}
The conditional distribution is smooth on ${\rm int}(\Param)$. For any $i \in \inter$, $z \in \Zset$, $\param, \param' \in {\rm int} (\Param)^2$, we have
$\big| p( z | y_i; \param ) - p( z | y_i; \param' ) \big| \leq  \Lip{p} \| \param - \param' \|$.
%For any $i \in \inter$, the map $\param \to \overline{\bss}_i(\param)$ is continuously differentiable in $\param$.
\end{assumption}

We also recall from the introduction that we consider curved exponential family models. besides:
\begin{assumption} \label{ass:reg}
For any $\bm{s} \in \Sset$, the function $\param \mapsto L(s,\param) \eqdef \Pen( \param ) + \psi( \param) - \pscal{ \bss}{ \phi ( \param) }$ admits a unique global minimum $\mstep{\bss} \in {\rm int}(\Param)$.
In addition, $\jacob{\phi}{\param}{\overline{\param}(\bss )}$ is full rank and $\overline{\param}( \bss )$ is $\Lip{\theta}$-Lipschitz.
\end{assumption}
Similar to \citep{karimi2019global}, we denote by $\hess{L}{\param}(\bss,\param)$ the Hessian (w.r.t to $\param$ for a given value of $\bss$) of the function $\param \mapsto L(\bss,\param)= \Pen(\param) + \psi(\param) -\pscal{\bss}{\phi(\param)}$, and define
\beq\label{eq:Bss}
\operatorname{B}( \bss ) \eqdef\jacob{ \phi }{ \param }{ \mstep{\bss} } \Big( \hess{L}{\param}( {\bss},  \mstep{\bss} )  \Big)^{-1} \jacob{ \phi }{ \param }{ \mstep{\bss} }^\top.
\eeq
\begin{assumption}\label{ass:eigen}
It holds that $ \upsilon_{\max} \eqdef \sup_{\bss \in \Sset} \| \operatorname{B}( \bss ) \| < \infty$ and $0 < \upsilon_{\min}  \eqdef \inf_{\bss \in \Sset} \lambda_{\rm min} ( \operatorname{B}( \bss ) )$.
There exists a constant $\Lip{B}$ such that for all $\bss, \bss' \in \Sset^2$, we have $ \| \operatorname{B}( \bss ) - \operatorname{B}( \bss' )  \| \leq \Lip{B} \| {\bss} - {\bss}' \|$.
\end{assumption}

We now formulate the main difference with the work done in \citep{karimi2019global}. 
The class of algorithms we develop in this paper are two time-scale where the first stage corresponds to the variance reduction trick used in \citep{karimi2019global} in order to accelerate incremental methods and kill the variance induced by the index sampling. 
The second stage is the Robbins-Monro type of update that aims to kill the variance induced by the MC approximations

Indeed the expectations \eqref{eq:definition-overline-bss} are never available and requires Monte Carlo approximation.
Thus, at iteration $k+1$, we introduce the errors when approximating the quantity $ \overline{\bss}_i(\hat{\param}(\hat{\bss}^{(k-1)}))$.
For all $i \in \inter$, $r > 0$ and $\vartheta \in \Theta$, define:
\beq\label{eq:mcerror}
\eta_{i, \vartheta}^{(r)} \eqdef \tilde{S}_{i}^{(r)} -  \overline{\bss}_i(\vartheta)
\eeq

For instance, we consider that the MC approximation is unbiased if for all $ i \in \inter$ and $m \in \llbracket 1, M \rrbracket$, the samples $z_{i,m} \sim p(z_i|y_i;\theta)$ are i.i.d. under the posterior distribution, \ie $\EE[\eta_{i, \vartheta}^{(r)}|{\cal F}_r] = 0$ where  ${\cal F}_r$ is the filtration up to iteration $r$.

The following results are derived under the assumption of control of the fluctuations implied by the approximation stated as follows:
\begin{assumption}\label{ass:mcerror}
There exist a positive sequence of MC batch size $\{M_k\}_{k > 0}$ and constants $(C, C_{\eta})$ such that for all $k >0$, $i \in \inter$ and $\vartheta \in \Theta$:
\beq\label{eq:boundederror}
\EE\left[\norm{\eta_{i, \vartheta}^{(r)}}^2 \right] \leq \frac{C_{\eta}}{M_r} \quad \textrm{and} \quad \EE\left[\norm{\EE[\eta_{i, \vartheta}^{(r)}|{\cal F}_r]}^2\right] \leq \frac{C}{M_r}
\eeq
\end{assumption}



\begin{Lemma} \label{lem:smooth}
\citep{karimi2019global} Assume H\ref{ass:expected}, H\ref{ass:reg}, H\ref{ass:eigen}.  
For all $\bss,\bss' \in \Sset$ and $i \in \inter$, we have
\beq \label{eq:smooth}
\| \overline{\bss}_i ( \overline{\param} ({\bss})) - \overline{\bss}_i ( \overline{\param} ({\bss}' )) \| \leq \Lip{{\bss}} \| {\bss} - {\bss}' \|,~~\| \grd  V ( {\bss} ) - \grd  V ( {\bss}' ) \| \leq \Lip{V} \| {\bss} - {\bss}' \|,
\eeq
where $\Lip{\bss} \eqdef C_{\Zset} \Lip{p} \Lip{\theta}$ and $\Lip{V}  \eqdef \upsilon_{\max} \big( 1 + \Lip{{\bss}} \big) + \Lip{B} C_{\Sset}$.
\end{Lemma}

\subsection{Global Convergence of Incremental Noisy EM Algorithms}
Following the asymptotic analysis of update \eqref{eq:isaem}, we present a finite-time analysis of the incremental variant of the Stochastic Approximation of the EM algorithm.

The first intermediate result is the computation of the quantity $\hat{S}^{(k+1)} - \hat{\bss}^{(k)}$, which corresponds to the dirft term of \eqref{eq:rmstep} and reads as follows:
\begin{Lemma} \label{lem:meanfield_isaem}
 Assume H\ref{ass:convset}. The update \eqref{eq:isaem} is equivalent to the following update on the resulting statistics 
\beq
\hat{\bss}^{(k+1)} =  \hat{\bss}^{(k)}  + \gamma_{k+1}(n^{-1}\sum_{i=1}^n \tilde{S}_i^{(\tau_i^k)} - \hat{\bss}^{(k)} )
\eeq
where $\tau_i^k = \max \{ k' : i_{k'} = i,~k' < k \}$. Also:
\beq
\EE\left[\norm{\tilde{S}^{(k+1)} - \hat{\bss}^{(k)}}^2 \right] \leq \EE\left[\norm{\overline{\bss}^{(k)} - \hat{\bss}^{(k)}}^2\right] +  2\Lip{\bss}^2\left(1 - \frac{1}{n} \right)^2  n^{-1}\sum_{i=1}^n\EE\left[\norm{  \hat{\bss}^{(k)} - \hat{\bss}^{(\tau_i^{k})} }^2\right] + \frac{2C}{M_k}
\eeq
where $\overline{\bss}^{(k)}$ is defined by \eqref{eq:definition-overline-bss}.
\end{Lemma}

The following main result for the \ISAEM\ algorithm is derived under a control of the Monte Carlo fluctuations as described by assumption H~\ref{ass:mcerror}.
Typically, the controls exhibited below are of interest when the number of MC samples $M_k$ increase with the iteration index $f$.

\begin{Theorem}\label{thm:isaem}
Let $K_{\max }$ be a positive integer. 
Let $\left\{\gamma_{k}, k \in \mathbb{N}\right\}$ be a sequence of positive step sizes and consider the \ISAEM\ sequence $\left\{\hat{\bss}^{(k)}, k \in \mathbb{N}\right\}$ obtained with $\rho_{k+1}=1$ for any $k$.

Assume that $ \hat{\bss}^{(k)} \in \mathcal{S}$ for any $k \leq K_{\max }$.
\end{Theorem} 

%Assume that $\widehat{S}^{k} \in \mathcal{S}$ for any $k \leq K_{\max }$ Let $\nu, \bar{\nu} \in\{0,1\}$ with the convention $\nu=0$ iff the approximations are unbiased, and $\bar{\nu}=0$ iff for any $k \geq 0,\left\|\eta_{k+1}^{(1)}\right\|=\left\|\eta_{k+1}^{(2)}\right\|=\varepsilon^{(0)}=0$
%For any positive numbers $\beta_{1}, \cdots, \beta_{K_{\max }-1}$ and $\beta_{0} \in\left(0, v_{\min } / v_{\max }^{2}\right),$ it holds
%$\sum_{k=0}^{K_{\max }-1} \alpha_{k} \mathbb{E}\left[\left\|\bar{s} \circ \operatorname{T}\left(\widehat{S}^{k}\right)-\widehat{S}^{k}\right\|^{2}\right]+\sum_{k=0}^{K_{\max }-1} \delta_{k} \mathbb{E}\left[\left\|\frac{1}{n} \sum_{i=1}^{n} \widetilde{S}_{k+1, i}-\bar{s} \circ \top\left(\widehat{S}^{k}\right)\right\|^{2}\right]$
%$\leq \mathbb{E}\left[V\left(\widehat{S}^{0}\right)\right]-\mathbb{E}\left[V\left(\widehat{S}^{K_{\max }}\right)\right]$
%$+\xi_{0}\left(K_{\max }, n\right) \mathbb{E}\left[\varepsilon^{(0)}\right]+\bar{\nu} \Xi_{1}\left(\eta^{(1)}, K_{\max }, n\right)+\bar{\nu} \Xi_{2}\left(\eta^{(2)}, K_{\max }, n\right)$
%for any $k=0, \ldots, K_{\max }-1$
%$\alpha_{k} \stackrel{\text { def }}{=} \gamma_{k+1}\left(v_{\min }-\nu v_{\max }^{2} \beta_{0}-(1+\nu) \frac{L_{\dot{V}}}{2} \gamma_{k+1}\left\{1+(1+\bar{\nu})(1+\nu) L^{2} \Lambda_{k}\right\}\right)$
%$\delta_{k} \stackrel{\text { def }}{=}(1+\nu) \frac{L_{\dot{V}}}{2} \gamma_{k+1}^{2}\left(1+(1+\bar{\nu})(1+\nu) \frac{\Lambda_{k}}{\left(1+\beta_{k+1}^{-1}\right)}\right)$
%with $\Lambda_{K_{\max }-1}=0$ and for $k=0, \ldots, K_{\max }-2$
%$\Lambda_{k} \stackrel{\text { def }}{=}\left(1+\frac{1}{\beta_{k+1}}\right) \sum_{j=k+1}^{K_{\max }-1} \gamma_{j+1}^{2} \prod_{\ell=k+2}^{j}\left(1-\frac{1}{n}+\beta_{\ell}+(1+\bar{\nu})(1+\nu) L^{2} \gamma_{\ell}^{2}\right)$
%$\xi_{0}, \Xi_{1}$ and $\Xi_{2}$ are non negative real numbers; their explicit expressions can be found in Section $4.6,$ Eqs $(54),(55)$ and $(56) .$ By convention, $\prod_{\ell \in \emptyset} a_{\ell}=1$

\subsection{Global Convergence of Two-Time-Scale Noisy EM Algorithms}
We now proceed by giving our main result regarding the global convergence of the \FISAEM\ algorithm.

\textcolor{red}{TO COMPLETE}


\section{Numerical Examples}
\subsection{Gaussian Mixture Models}
Given $n$ observations $\{y_i\}_{i=1}^n$, we want to fit a Gaussian Mixture Model (GMM) whose distribution is modeled as a Gaussian mixture of $M$ components, each with a unit variance. 
Let $z_i \in \inter[M]$ be the latent labels of each component, the complete log-likelihood is defined as:
\beq \label{eq:comp_like} \textstyle
\log f( z_i, y_i; \param) =
% \sum_{m=1}^M \indiacc{m}(z_i) \left[\log(\omega_m) - (y_i-\mu_m)^2/2 \right] =
\sum_{m=1}^{M} \indiacc{m}(z_i) \left[ \log(\omega_m) - \mu_m^2/2 \right] + \sum_{m=1}^M \indiacc{m}(z_i) \mu_m y_i + {\rm constant} \eqsp.
\eeq
where $\param \eqdef (\bomega, \bmu)$ with $\bomega= \{\omega_{m}\}_{m=1}^{M-1}$ are the mixing weights with the convention $\omega_M= 1 - \sum_{m=1}^{M-1} \omega_m$  and $\bmu= \{\mu_m \}_{m =1}^M$ are the means.  We use the penalization 
%$\Pen( \param ) \eqdef  \epsilon \sum_{m=1}^M \big\{ \mu_m^2 / 2 - \log ( \omega_m ) \big\} - \epsilon \log \big( 1 - \sum_{m=1}^{M-1} \omega_m \big)$ where $\epsilon > 0$ is a small regularization parameter. 
$\Pen(\param)= \frac{\delta}{2}\sum_{m=1}^M \mu_m^2 - \log \Dir(\bomega; M, \epsilon)$ where $\delta > 0$ and $\Dir(\cdot; M,\epsilon)$ is the $M$ dimensional symmetric Dirichlet distribution with concentration parameter $\epsilon > 0$.
The constraint set on $\param$ is given by
\beq \textstyle
\Param = \{ \omega_m,~m=1,...,M-1 : \omega_m \geq 0,~\sum_{m=1}^{M-1} \omega_m \leq 1\} \times \{ \mu_m \in \rset ,~m=1,...,M \}.
\eeq
In the following experiments of synthetic data, we generate samples from a GMM model with $M=2$ components with two mixtures with means $\mu_1 = - \mu_2 = 0.5$.


We use $n = 10^4$ synthetic samples and run the bEM method until convergence (to double precision) to obtain the ML estimate $\mu^\star$. We compare the bEM, SAEM, \ISAEM, \SAEMVR\ and \FISAEM\ methods in terms of their precision measured by $| \mu - \mu^\star |^2$. 
%We set the stepsize of the \SEM\ as $\gamma_k = 3/(k+10)$, and the stepsizes of the \SEMVR\ and the \FIEM\ to a constant stepsize proportional to $1/n^{2/3}$ and equal to $\gamma = 0.003$. 
The left plot of Figure \ref{fig:gmmplots} shows the convergence of the precision $|\mu - \mu^*|^2$ for the different methods against the epoch(s) elapsed (one epoch equals $n$ iterations). We observe that the \SAEMVR\ and \FISAEM\ methods outperform the other methods, supporting our analytical results.




\subsection{Deformable Template Model for Image Analysis}
We now run our different methods using an example taken from \citep{allassonniere2010construction}.
Let $(y_i, i \in \inter)$ be observed images. 
Let $u \in \mathcal{U} \subset \rset^2$ denote the pixel index on the image and $x_u \in \mathcal{D} \subset \rset^2$ its location.

The model used in this experiment suggests that each image $y_i$ is a deformation of a template, noted $I: \mathcal{D} \to \rset$, common to all images of the dataset:
\beq\label{eq:deformablemodel}
y_{i}(u)=I\left(x_{u}-\Phi_{i}\left(x_{u}\right)\right)+\varepsilon_{i}(u)
\eeq
where $\phi_i: \rset^2 \to \rset^2$ is a deformation function, and $\varepsilon_{i} \sim \mathcal{N}(0,\sigma^2)$ is an observation error.

The template model, given $(p_k, k \in \llbracket 1, k_p \rrbracket)$ landmarks on the template, a fixed known kernel $\mathbf{K}_{\mathbf{p}}$ and a vector of parameters $\beta \in \rset^{k_p}$ is defined as follows:
\beq
I_{\xi}=\mathbf{K}_{\mathbf{p}} \beta, \quad \textrm{where} \quad \left(\mathbf{K}_{\mathbf{p}} \beta \right)(x)=\sum_{k=1}^{k_{p}} \mathbf{K}_{\mathbf{p}}\left(x, p_{k}\right) \beta_k
\eeq

Besides, we parameterize the deformation model given some landmarks $(g_k, k \in \llbracket 1, k_g \rrbracket)$ and a fixed kernel $\mathbf{K}_{\mathbf{g}}$ as:
\beq
\Phi_{i}(x)=\left(\mathbf{K}_{\mathbf{g}} z_{i}\right)(x)=\sum_{k=1}^{k_{s}} \mathbf{K}_{\mathbf{g}}\left(x, g_{k}\right)\left(z_{i}^{(1)}(k), z_{i}^{(2)}(k)\right)
\eeq
where $z_i \sim \mathcal(0,\Gamma)$ and $z_i \in \left( \rset^{k_g}\right)^2$.



\section{Conclusion}


\newpage
\linespread{1.1}
\normalsize

\bibliographystyle{abbrvnat}
\bibliography{references}

\linespread{1}
\newpage

\appendix
\section{Proof of Theorem~\ref{thm:isaem}}

\begin{Theorem*}
Let $K_{\max }$ be a positive integer. 
Let $\left\{\gamma_{k}, k \in \mathbb{N}\right\}$ be a sequence of positive step sizes and consider the \ISAEM\ sequence $\left\{\hat{\bss}^{(k)}, k \in \mathbb{N}\right\}$ obtained with $\rho_{k+1}=1$ for any $k$.

Assume that $ \hat{\bss}^{(k)} \in \mathcal{S}$ for any $k \leq K_{\max }$.
\end{Theorem*} 

\begin{proof}
Under some regularity conditions of the Lyapunov function $V$, cf. Lemma \ref{eq:smooth}, and the following growth condition for all $\bss \in \Sset$,
\beq \label{eq:semigrad}
\upsilon_{\min}^{-1} \pscal{\grd V ( {\bss} ) }{ {\bss} - \os( \op ({\bss})) }
\geq \big\| {\bss} - \os( \op ({\bss})) \big\|^2 \geq \upsilon_{\max}^{-2} \| \grd V ( {\bss} ) \|^2,
\eeq
proven in \citep[Lemma 3]{karimi2019global}, we can write:
\beq
\begin{split}
V( \hs{k+1} ) & \leq V( \hs{k} ) - \gamma_{k+1} \pscal{ \hs{k} - \tilde{S}^{(k+1)} }{ \grd V( \hs{k} ) } + \frac{\gamma_{k+1}^2 \Lip{V}}{2} \| \hs{k} - \tilde{S}^{(k+1)} \|^2 \\
\end{split}
\eeq

Taking the expectation on both sides and using the growth condition \eqref{eq:semigrad}, we obtain:
\beq
\begin{split}
\EE[V( \hs{k+1} )] & \leq \EE[V( \hs{k} )]  - \gamma_{k+1}\upsilon_{\min} \EE\left[\norm{\overline{\bss}^{(k)} - \hat{\bss}^{(k)}}^2\right] + \EE\left[\frac{\gamma_{k+1}^2 \Lip{V}}{2} \| \hs{k} - \tilde{S}^{(k+1)} \|^2\right] \\
& - \gamma_{k+1} \textcolor{red}{\EE\left[ \pscal{\overline{\bss}^{(k)}- \tilde{S}^{(k+1)} }{ \grd V( \hs{k} ) }   \right]}
\end{split}
\eeq

We then establish an auxiliary Lemma yielding an upper-bound on the quantity $ \EE\left[ \pscal{\overline{\bss}^{(k)}- \tilde{S}^{(k+1)} }{ \grd V( \hs{k} ) }   \right]$ where:
\beq
\overline{\bss}^{(k)}- \tilde{S}^{(k+1)}  = \overline{\bss}^{(k)}- \left (\tilde{S}^{(k)}  + \frac{1}{n}\big( \tilde{S}_{i_k}^{(k)}  - \tilde{S}_{i_k}^{(\tau_{i_k}^k)} \big) \right)
\eeq

\textcolor{red}{
\begin{Lemma} \label{lem:s}
\beq
\EE\left[ \pscal{\overline{\bss}^{(k)}- \tilde{S}^{(k+1)} }{ \grd V( \hs{k} ) }   \right] \leq 
\eeq
\end{Lemma}
}

Using Lemma~\ref{lem:meanfield_isaem}:
\beq
\begin{split}
\EE[V( \hs{k+1} )] & \leq \EE[V( \hs{k} )]  - \gamma_{k+1} \left( \upsilon_{\min} - \frac{\gamma_{k+1} \Lip{V}}{2} \right)  \EE\left[\norm{\overline{\bss}^{(k)} - \hat{\bss}^{(k)}}^2\right] \\
&+ \gamma^2_{k+1} \Lip{V}\Lip{\bss}^2\left(1 - \frac{1}{n} \right)^2  n^{-1}\sum_{i=1}^n\EE\left[\norm{  \hat{\bss}^{(k)} - \hat{\bss}^{(\tau_i^{k})} }^2\right] + \frac{\gamma^2_{k+1} \Lip{V}C}{M_k}\\
& - \gamma_{k+1} \textcolor{red}{\EE\left[ \pscal{\overline{\bss}^{(k)}- \tilde{S}^{(k+1)} }{ \grd V( \hs{k} ) }   \right]}
\end{split}
\eeq

Besides, 
\beq
\begin{split}
  n^{-1}\sum_{i=1}^n\EE\left[\norm{ \hat{\bss}^{(k+1)} -\hat{\bss}^{(\tau_i^{k+1})} }^2\right] &=   n^{-1} \sum_{i=1}^n
\Big( \frac{1}{n} \EE[ \| \hat{\bss}^{(k+1)}  - \hat{\bss}^{(k)} \|^2 ] + \frac{n-1}{n} \EE[ \| \hat{\bss}^{(k+1)}  -\hat{\bss}^{(\tau_i^{k})} \|^2 ]  \Big)\\
\end{split}
\eeq
yielding for any numbers $\beta_k > 0$,
\beq
\begin{split}
& \EE[ \| \hat{\bss}^{(k+1)} - \hs{t_i^k} \|^2 ] \\
& = \EE \Big[ \| \hat{\bss}^{(k+1)} - \hat{\bss}^{(k)} \|^2 + \| \hat{\bss}^{(k)} - \hs{t_i^k} \|^2 + 2 \pscal{\hat{\bss}^{(k+1)} - \hat{\bss}^{(k)}}{\hat{\bss}^{(k)}- \hs{t_i^k}} \Big] \\
& = \EE \Big[ \| \hat{\bss}^{(k+1)} - \hat{\bss}^{(k)} \|^2 + \| \hat{\bss}^{(k)} - \hs{t_i^k} \|^2 - 2 \gamma_{k+1} \pscal{
\hat{\bss}^{(k)} - \os^{(k)} }{\hat{\bss}^{(k)}- \hs{t_i^k}} \Big] \\
& \leq  \EE \Big[ \| \hat{\bss}^{(k+1)} - \hat{\bss}^{(k)} \|^2 + \| \hat{\bss}^{(k)} - \hs{t_i^k} \|^2 +  \frac{\gamma_{k+1}}{\beta_{k+1}} \| \hat{\bss}^{(k)} - \os^{(k)} \|^2 + \gamma_{k+1} \beta_{k+1} \| \hat{\bss}^{(k)}- \hs{t_i^k} \|^2 \Big]
\end{split}
\eeq



\end{proof}


\end{document}
