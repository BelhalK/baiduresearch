\documentclass[11pt]{article}

% if you need to pass options to natbib, use, e.g.:
%\PassOptionsToPackage{numbers}{natbib}
\usepackage[colorlinks,linkcolor=blue,filecolor=blue,citecolor=magenta,urlcolor=blue]{hyperref}
\usepackage{bm,amsmath,amsthm,amssymb,multicol,algorithmic,algorithm,enumitem,graphicx,subfigure}
\usepackage{xargs}
\usepackage{stmaryrd}
\usepackage{natbib}

% ready for submission
\usepackage{neurips_2020}
\def\M{\mathcal{M}}
\def\A{\mathcal{A}}
\def\Z{\mathcal{Z}}
\def\S{\mathcal{S}}
\def\D{\mathcal{D}}
\def\R{\mathcal{R}}
\def\P{\mathcal{P}}
\def\K{\mathcal{K}}
\def\E{\mathbb{E}}
\def\F{\mathfrak{F}}
\def\l{\boldsymbol{\ell}}

\newtheorem{Fact}{Fact}
\newtheorem{Lemma}{Lemma}
\newtheorem{Prop}{Proposition}
\newtheorem{Theorem}{Theorem} 
\newtheorem{Def}{Definition}
\newtheorem{Corollary}{Corollary}
\newtheorem{Conjecture}{Conjecture}
\newtheorem{Property}{Property}
\newtheorem{Observation}{Observation}
\newtheorem{Exa}{Example}
\newtheorem{assumption}{H\!\!}
\newtheorem{Remark}{Remark}
\newtheorem*{Lemma*}{Lemma}
\newtheorem*{Theorem*}{Theorem}
\newtheorem*{Corollary*}{Corollary}
 
\newcommand{\eqsp}{\;}
\newcommand{\beq}{\begin{equation}}
\newcommand{\eeq}{\end{equation}}
\newcommand{\eqdef}{\mathrel{\mathop:}=}
\def\EE{\mathbb{E}}
\newcommand{\norm}[1]{\left\Vert #1 \right\Vert}
\newcommand{\pscal}[2]{\left\langle#1\,|\,#2 \right\rangle}
\def\major{\mathsf{M}}
\def\rset{\ensuremath{\mathbb{R}}}

\begin{document}
\title{TBD}

\author{
TBD
}

\date{\today}

\maketitle

\begin{abstract}
To be completed...
\end{abstract}

\section{Introduction}\label{sec:introduction}



\section{Problem Statement and Notations}\label{sec:notations}

\textcolor{red}{to formalize well inspired by how this motion planning with obstacles problem is presented in other papers}

\noindent\textit{Motion planning Problem description}: Let there be an environment of $\mathbb{R}^d$ where $d = 2$ or $3$; obstacles $\mathcal{O}_i\in\mathbb{R}^d$, $i = \{1, 2, \ldots, n\}$, a robot of geometry $\mathcal{B}$, denote the configuration of the robot as $q\in\mathbb{R}^n$, where $n$ is the number of Degree of Freedom (DoF) of the robot. Let the robot has controls $u\in\mathcal{U}$, and let there be start $s\in\mathbb{R}^n$ and goal $g\in\mathbb{R}^n$; find a sequence of robot configurations or a sequence of controls so that the robot can go from $s$ to $g$. We can assume there is an oracle function $\mathcal{F}:(q\times \cup\mathcal{O})\rightarrow \{0, 1\}$ that can return collision detection result in a given environment for any given configuration of the robot. 

One common approach is called sampling based motion planning, which is achieved through placing samples in $\mathcal{R}^n$, the configuration space, and retain valid non-collision samples in the configuration space by inquiring $\mathcal{F}$. The invalid samples are discarded (or retained in some cases), and the valid samples are connected if the path connecting the configurations is valid (pass the validity check after inquiring $\mathcal{F}$). A good set of samples will lead to solutions much faster compared to random samples, though may not always be optimal. 

If we want to use a learning method to create the samples, with bias, we may be able to find paths faster. 


\section{Using a Energy Learning Approach}\label{sec:learning}

\textcolor{red}{Formalize and state the learning problem.}

\section{Conclusion}\label{sec:conclusion}



\newpage
\bibliographystyle{plain}
\bibliography{ref}

\newpage
\appendix 

\section{Appendix}\label{sec:appendix}



\end{document}

