%%%%%%%%%%%%%%%%%%%%بسمه تعالی%%%%%%%%%%%%%%%%%%%%%%
\documentclass[10pt]{article}
\usepackage[utf8]{inputenc}
\usepackage{fullpage}

\usepackage[utf8]{inputenc} % allow utf-8 input
\usepackage[T1]{fontenc}    % use 8-bit T1 fonts
%\usepackage{url,apacite}    % package url to prevent horrible linebreaks

\usepackage{booktabs}       % professional-quality tables
\usepackage{amsfonts}       % blackboard math symbols
\usepackage{nicefrac}       % compact symbols for 1/2, etc.
\usepackage{microtype}      % microtypography
\usepackage{amsmath}
\usepackage{tcolorbox}
\definecolor{my-green}{cmyk}{0.2, 0.04, 0.1, 0.04, 0.8}
\usepackage{graphicx}
\usepackage{amsthm}
\usepackage{amssymb}
% \usepackage{cite}
\usepackage{multirow}
\usepackage{makecell}
\usepackage[title]{appendix}
\usepackage{algpseudocode}
\usepackage{algorithm}
\usepackage{framed}
\usepackage{mdframed}
\usepackage[pdftex,bookmarks=true,pdfstartview=FitH,colorlinks,linkcolor=magenta,filecolor=blue,citecolor=magenta,urlcolor=blue,pagebackref=false]{hyperref}%
\usepackage{wrapfig}
\usepackage[font=footnotesize,labelsep=space,labelfont=bf]{caption}

\usepackage[top=1in, left=0.9in, right=0.9in, bottom=0.9in]{geometry}
\usepackage{blindtext}
\usepackage{subfigure}
\usepackage{afterpage}
\usepackage{amssymb}% http://ctan.org/pkg/amssymb
\usepackage{pifont}
\usepackage{mathtools}
\DeclarePairedDelimiter{\ceil}{\lceil}{\rceil}
\DeclarePairedDelimiter{\floor}{\lfloor}{\rfloor}
\newcommand{\pl}{Polyak-\L{}ojasiewicz}
\newcommand{\todoM}[1]{\textcolor{blue}{ToDo (Farzin): #1}}
\newcommand{\todo}[1]{\textcolor{red}{ToDo:~#1}}
\newcommand{\alert}[1]{\textcolor{red}{#1}}
\newcommand{\Var}{\mathrm{Var}}
\newcommand{\E}{\mathrm{E}}
\newcommand{\bm}[1]{\boldsymbol{#1}}
\newtheorem{theorem}{Theorem}
\newtheorem{lemma}{Lemma}
\newtheorem{remark}{Remark}
\newtheorem{assumption}{Assumption}
\newtheorem{proposition}{Proposition}
\newtheorem{property}{Property}
\newtheorem{corollary}{Corollary}
\newtheorem{definition}{Definition}
\newtheorem{claim}[theorem]{\bf Claim}
\newtheorem{fact}[theorem]{Fact}
\newtheorem{example}[theorem]{Example}

\newcommand{\belhal}[1]{\todo{{\bf BK:} #1}}

\allowdisplaybreaks

% \usepackage{epsf}
%\usepackage{fancyheadings}
\usepackage{graphics}
%\usepackage{graphicx}

%\usepackage{subfigure}
\usepackage{subcaption}
\usepackage{psfrag}

\usepackage{color}

\usepackage{mathtools}
\usepackage{amsfonts}
\usepackage{amsthm}
\usepackage{amsmath}
\usepackage{amssymb}
%\usepackage{fdsymbol}
%\usepackage[linesnumbered, ruled, vlined]{algorithm2e}
%\usepackage{ntheorem}

%\usepackage[backend=bibtex,firstinits=true]{biblatex}

%\setlength{\textwidth}{\paperwidth}
%\addtolength{\textwidth}{-6cm}
%\setlength{\textheight}{\paperheight}
%\addtolength{\textheight}{-4cm}
%\addtolength{\textheight}{-1.1\headheight}
%\addtolength{\textheight}{-\headsep}
%\addtolength{\textheight}{-\footskip}
%\setlength{\oddsidemargin}{0.5cm}
%\setlength{\evensidemargin}{0.5cm}




%%%%%%%%%%%%%%%%%%%%%%%%%%%%%%%%%%%%%%%%%%%%%%%%%%%%%%%%%%%%%%%%%%%%%%
% MACROS HERE

%%%%%%%
%\theoremstyle{plain}

% {Theorem, Proposition, Lemma, Corollary} numbered sequentially
% throughout the paper

%\theoremprework{\begin{minipage}{\textwidth}}
%\theorempostwork{\end{minipage}}

\newtheorem{theorem}{Theorem}
\newtheorem{proposition}{Proposition}
\newtheorem{lemma}{Lemma}
\newtheorem{corollary}{Corollary}
\newtheorem{definition}{Definition}
\newtheorem{remark}{Remark}
\newtheorem{conjecture}{Conjecture}

%%%%%%%%%%%%%%%%%%%%%%%%%%%%%%%%%%%%%%%%%%%%%%%%%%%%%%%%%%%%%%%%%%%%%%%
% WIDEBAR COMMAND
%\newlength{\widebarargwidth}
%\newlength{\widebarargheight}
%\newlength{\widebarargdepth}
%\DeclareRobustCommand{\widebar}[1]{%
%  \settowidth{\widebarargwidth}{\ensuremath{#1}}%
%  \settoheight{\widebarargheight}{\ensuremath{#1}}%
%  \settodepth{\widebarargdepth}{\ensuremath{#1}}%
%  \addtolength{\widebarargwidth}{-0.3\widebarargheight}%
%  \addtolength{\widebarargwidth}{-0.3\widebarargdepth}%
%  \makebox[0pt][l]{\hspace{0.3\widebarargheight}%
%    \hspace{0.3\widebarargdepth}%
%    \addtolength{\widebarargheight}{0.3ex}%
%    \rule[\widebarargheight]{0.95\widebarargwidth}{0.1ex}}%
%  {#1}}



%%% New version of \caption puts things in smaller type, single-spaced 
%%% and indents them to set them off more from the text.
%\makeatletter
%\long\def\@makecaption#1#2{
%        \vskip 0.8ex
%        \setbox\@tempboxa\hbox{\small {\bf #1:} #2}
%        \parindent 1.5em  %% How can we use the global value of this???
%        \dimen0=\hsize
%        \advance\dimen0 by -3em
%        \ifdim \wd\@tempboxa >\dimen0
%                \hbox to \hsize{
%                        \parindent 0em
%                        \hfil 
%                        \parbox{\dimen0}{\def\baselinestretch{0.96}\small
%                                {\bf #1.} #2
%                                %%\unhbox\@tempboxa
%                                } 
%                        \hfil}
%        \else \hbox to \hsize{\hfil \box\@tempboxa \hfil}
%        \fi
%        }
%\makeatother



%% COMMENTING commands

\long\def\comment#1{}

\newcommand{\bm}[1]{\boldsymbol{#1}}
\newcommand{\bit}[1]{\boldsymbol{#1}}

\newcommand{\red}[1]{\textcolor{red}{#1}}
\newcommand{\mjwcomment}[1]{{\bf{{\red{{MJW --- #1}}}}}}
\newcommand{\blue}[1]{\textcolor{blue}{#1}}
\newcommand{\mycomment}[1]{{\bf{{\blue{{AP --- #1}}}}}}
\newcommand{\green}[1]{\textcolor{green}{#1}}
\newcommand{\vidyacomment}[1]{{\bf{{\green{{VM --- #1}}}}}}
\newcommand{\cyan}[1]{\textcolor{cyan}{#1}}
\newcommand{\cmcomment}[1]{{\bf{{\cyan{{CM --- #1}}}}}}

% Some vector/matrix norms
\newcommand{\matnorm}[3]{|\!|\!| #1 | \! | \!|_{{#2}, {#3}}}
\newcommand{\matsnorm}[2]{|\!|\!| #1 | \! | \!|_{{#2}}}
\newcommand{\vecnorm}[2]{\| #1\|_{#2}}

\newcommand{\defn}{:\,=}

%Hamming metric
\newcommand{\dH}{\mathsf{d}_{\mathsf{H}} }

\newcommand{\id}{\mathsf{id}}

\newcommand{\KT}{\mathsf{KT}}

\newcommand{\dimone}{n_1}
\newcommand{\dimtwo}{n_2}
\newcommand{\maxdimonetwo}{\left(\dimone \lor \dimtwo \right)}

\newcommand{\Aone}{{\bf C1} }
\newcommand{\Atwo}{{\bf C2} }

\newcommand{\csst}{\mathbb{C}_{\mathsf{SST}}}
\newcommand{\cdiff}{\mathbb{C}_{\mathsf{DIFF}}}
\newcommand{\Cpara}{\mathbb{C}_{\mathsf{PARA}}}
\newcommand{\Cbiso}{\mathbb{C}_{\mathsf{BISO}}}
\newcommand{\Cns}{\mathbb{C}_{\mathsf{NS}}}
\newcommand{\Cnspi}{M_{\mathsf{NS}, \pi}}
\newcommand{\Mns}{M_{\mathsf{NS}}}
\newcommand{\Cnspihat}{M_{\mathsf{NS}, \pihat}}

\newcommand{\sk}{S_k}
\newcommand{\bl}{\mathsf{bl}}

\newcommand{\bltilde}{\widetilde{\mathsf{bl}}}
\newcommand{\Bl}{\mathsf{B}}
\newcommand{\Rw}{\mathsf{R}}

\newcommand{\tauhat}{\widehat{\tau}}
\newcommand{\Chatt}{\widehat{C}_t}

\newcommand{\rhat}{\widehat{\mathsf{r}}}
\newcommand{\rspace}{\mathsf{r}}

\newcommand{\elts}{\mathsf{s}}
\newcommand{\Ntilde}{\widetilde{N}}

\newcommand{\lambdahat}{\widehat{\lambda}}

\newcommand{\Cmodel}{\mathbb{C}}

\newcommand{\pobs}{p_{\mathsf{obs}}}

\newcommand{\sgn}{\mathsf{sgn}}

\newcommand{\nucnorm}[1]{\ensuremath{\matsnorm{#1}{\footnotesize{\mbox{nuc}}}}}
\newcommand{\fronorm}[1]{\ensuremath{\matsnorm{#1}{\footnotesize{\mbox{fro}}}}}
\newcommand{\opnorm}[1]{\ensuremath{\matsnorm{#1}{\tiny{\mbox{op}}}}}


% Inner product
\newcommand{\inprod}[2]{\ensuremath{\langle #1 , \, #2 \rangle}}

% Kullback-Leibler
\newcommand{\kull}[2]{\ensuremath{D(#1\; \| \; #2)}}

% Probability
\newcommand{\E}{\ensuremath{{\mathbb{E}}}}
\newcommand{\Prob}{\ensuremath{{\mathbb{P}}}}

% Observations, dimension etc.
\newcommand{\numobs}{\ensuremath{n}}
\newcommand{\usedim}{\ensuremath{d}}

\newcommand{\Comp}{\ensuremath{C}}

\newcommand{\Mrisk}{\ensuremath{\mathcal{M}}}
\newcommand{\Erisk}{\ensuremath{\mathcal{E}}}
\newcommand{\Arisk}{\ensuremath{\mathcal{A}}}

\newcommand{\diam}{\ensuremath{\mathsf{diam}}}
%Numbering
\newcommand{\1}{\ensuremath{{\sf (i)}}}
\newcommand{\2}{\ensuremath{{\sf (ii)}}}
\newcommand{\3}{\ensuremath{{\sf (iii)}}}
\newcommand{\4}{\ensuremath{{\sf (iv)}}}
\newcommand{\5}{\ensuremath{{\sf (v)}}}
%Eigenvector / eigenvalue related notation
\newcommand{\eig}[1]{\ensuremath{\lambda_{#1}}}
\newcommand{\eigmax}{\ensuremath{\eig{\max}}}
\newcommand{\eigmin}{\ensuremath{\eig{\min}}}

\DeclareMathOperator{\modd}{mod}
\DeclareMathOperator{\diag}{diag}
\DeclareMathOperator{\Var}{var}
\DeclareMathOperator{\cov}{cov}
\DeclareMathOperator{\abs}{abs}
\DeclareMathOperator{\floor}{floor}
\DeclareMathOperator{\vol}{vol}
\DeclareMathOperator{\child}{child}
\DeclareMathOperator{\parent}{parent}
\DeclareMathOperator{\sign}{sign}
\DeclareMathOperator{\rank}{{\sf rank}}
\DeclareMathOperator{\card}{{\sf card}}
\DeclareMathOperator{\range}{{\sf range}}
\DeclareMathOperator{\toeplitz}{toeplitz}


% 
\newcommand{\NORMAL}{\ensuremath{\mathcal{N}}}
\newcommand{\BER}{\ensuremath{\mbox{\sf Ber}}}
\newcommand{\BIN}{\ensuremath{\mbox{\sf Bin}}}
\newcommand{\Hyp}{\ensuremath{\mbox{\sf Hyp}}}

\newcommand{\Xspace}{\ensuremath{\mathcal{X}}}
\newcommand{\Yspace}{\ensuremath{\mathcal{Y}}}
\newcommand{\Zspace}{\ensuremath{\mathcal{Z}}}
\newcommand{\Fspace}{\ensuremath{\mathcal{F}}}
\newcommand{\Ospace}{\ensuremath{\mathcal{O}}}

\newcommand{\Bupper}{\ensuremath{\overline{B}}}
\newcommand{\Blower}{\ensuremath{\underline{B}}}
% Basic statistics notation
% True parameter
\newcommand{\thetastar}{\ensuremath{\theta^*}}
% Estimate one
\newcommand{\thetahat}{\ensuremath{\widehat{\theta}}}

\newcommand{\xhat}{\ensuremath{\widehat{x}}}
\newcommand{\Deltahat}{\ensuremath{\widehat{\Delta}}}

\newcommand{\Deltatilde}{\ensuremath{\widetilde{\Delta}}}
\newcommand{\udiff}{\ensuremath{\mathbb{U}^{{\sf diff}}_m(A)}}
\newcommand{\Xhat}{\ensuremath{\widehat{X}}}
\newcommand{\Yhat}{\ensuremath{\widehat{Y}}}
\newcommand{\Ytilde}{\ensuremath{\widetilde{Y}}}
\newcommand{\Mhat}{\ensuremath{\widehat{M}}}
\newcommand{\Mstar}{\ensuremath{M^*}}
\newcommand{\Mtilde}{\ensuremath{\widetilde{M}}}
\newcommand{\bhat}{\ensuremath{\widehat{b}}}
\newcommand{\Mhatasp}{\ensuremath{\widehat{M}}_{\mathsf{ASP}}}
\newcommand{\ASP}{\ensuremath{\mathsf{ASP}}}

\newcommand{\CRL}{\ensuremath{\mathsf{CRL}}}

\newcommand{\Mhattds}{\ensuremath{\widehat{M}}_{{\sf TDS}}}

\newcommand{\Mhatod}{\ensuremath{\widehat{M}}_{{\sf 1D}}}
\newcommand{\Mhatls}{\ensuremath{\widehat{M}}_{{\sf LS}}}
\newcommand{\Mhatpermr}{\ensuremath{\widehat{M}}_{{\sf permr}}}

\newcommand{\BAP}{\ensuremath{\mathsf{BAP}}}
\newcommand{\agg}{\ensuremath{\mathsf{Agg}}}
\newcommand{\Mhatbap}{\ensuremath{\widehat{M}}_{\mathsf{BAP}}}

\newcommand{\Btilde}{\ensuremath{\widetilde{B}}}

\newcommand{\xhatml}{\ensuremath{\widehat{x}_{{\sf ML}}}}
\newcommand{\Xhatml}{\ensuremath{\widehat{X}_{{\sf ML}}}}
% Estimate two
\newcommand{\thetatil}{\ensuremath{\widetilde{\theta}}}

\newcommand{\gt}{\ensuremath{\widetilde{g}}}

\newcommand{\yvec}{\ensuremath{y}}
\newcommand{\Xmat}{\ensuremath{X}}

\newcommand{\widgraph}[2]{\includegraphics[keepaspectratio,width=#1]{#2}}


%Problem-specific notation
\newcommand{\Ind}{\ensuremath{\mathbb{I}}}
\newcommand{\real}{\ensuremath{\mathbb{R}}}

\newcommand{\Pihat}{\ensuremath{\widehat{\Pi}}}
\newcommand{\pihat}{\ensuremath{\widehat{\pi}}}
\newcommand{\pihatborda}{\ensuremath{\widehat{\pi}_{\sf bor}}}
\newcommand{\pihatpre}{\ensuremath{\widehat{\pi}_{\sf pre}}}
%\newcommand{\pihatref}{\ensuremath{\widehat{\pi}_{\sf ref}}}
\newcommand{\pihattds}{\ensuremath{\widehat{\pi}_{\sf tds}}}
\newcommand{\pihatoned}{\ensuremath{\widehat{\pi}_{{\sf 1D}}}}
\newcommand{\pihatftds}{\ensuremath{\widehat{\pi}_{{\sf tds}}}}
\newcommand{\sigmahat}{\ensuremath{\widehat{\sigma}}}
\newcommand{\Sigmahat}{\ensuremath{\widehat{\Sigma}}}
\newcommand{\sigmahatborda}{\ensuremath{\widehat{\sigma}_{\sf bor}}}
%\newcommand{\sigmahatref}{\ensuremath{\widehat{\sigma}_{\sf ref}}}
\newcommand{\sigmahattds}{\ensuremath{\widehat{\sigma}_{\sf tds}}}
\newcommand{\sigmahatftds}{\ensuremath{\widehat{\sigma}_{\sf tds}}}
\newcommand{\sigmahatpre}{\ensuremath{\widehat{\sigma}_{\sf pre}}}

\newcommand{\pihatasp}{\ensuremath{\widehat{\pi}_{\mathsf{ASP}}}}
\newcommand{\Pihatml}{\ensuremath{\widehat{\Pi}_{{\sf ML}}}}
\newcommand{\Pitil}{\ensuremath{\widetilde{\Pi}}}

\newcommand{\Projpi}{\ensuremath{P_{\Pi}}}
\newcommand{\Projorthpi}{\ensuremath{P_{\Pi}^{\perp}}}
\newcommand{\Projorthpistar}{\ensuremath{P_{\Pi^*}^{\perp}}}

\newcommand{\hballpi}{\ensuremath{\mathbb{B}^{\sf H}_\Pi}}
\newcommand{\hballpistar}{\ensuremath{\mathbb{B}_{\sf H}}}
%\newcommand{\hballpistar}{\ensuremath{\mathcal{P}_n}}
\newcommand{\SNR}{\ensuremath{\frac{\|x^*\|_2^2}{\sigma^2}}}
\newcommand{\col}{\ensuremath{{\sf col}}}
\newcommand{\row}{\ensuremath{{\sf row}}}
\newcommand{\Pn}{\ensuremath{\mathcal{P}_n}}

\newcommand{\EE}{\ensuremath{\mathbb{E}}}
%changed to probability of error
\newcommand{\Rpihat}{\ensuremath{\Pr\{ \Pihat \neq \Pi^* \} }}

\newcommand{\Rpihatml}{\ensuremath{\Pr\{ \Pihatml \neq \Pi^* \}}}
\newcommand{\BigO}{\ensuremath{\mathcal{O}}}


\newcommand{\numitems}{\ensuremath{n}}

% commands that CM changed/introduced
\newcommand{\symgp}{\mathfrak{S}}
\newcommand{\vars}{\zeta}
\newcommand{\maxvarone}{(\vars^2 \lor 1)}
\newcommand{\varplusone}{(\vars + 1)}
\newcommand{\allones}{\mathbf{e}}
\newcommand{\bfone}{\mathbf{1}}
\DeclareMathOperator{\trace}{\mathsf{tr}}
\DeclareMathOperator{\Poi}{\mathsf{Poi}}
\DeclareMathOperator{\var}{\mathsf{var}}
\DeclarePairedDelimiter\tv{\mathsf{tv}(}{)}

%\newcommand{\Cperm}{\mathbb{C}_{\mathsf{Perm}}^{\mathsf{r},\mathsf{c}}}
\newcommand{\Cperm}{\mathbb{C}_{\mathsf{Perm}}}
%\newcommand{\Cpermr}{\mathbb{C}_{\mathsf{Perm}}^{\mathsf{r}}}
%\newcommand{\Cpermc}{\mathbb{C}_{\mathsf{Perm}}^{\mathsf{c}}}
\newcommand{\Cskew}{\mathbb{C}_{\mathsf{skew}}}

\newcommand{\tds}{{\sf TDS}}

\newcommand{\Ypoi}{Y_{\sf Poi}}
%\newcommand{\Ysortcol}{Y_{\sf sortcol}}
%\newcommand{\Ysortrow}{Y_{\sf sortrow}}

\newcommand{\BL}{\mathsf{BL}}

\newcommand{\Mps}{M_{\pi, \sigma}}
\newcommand{\Dps}{\Delta_{\pi, \sigma}}
\newcommand{\Cdiffps}{\cdiff(\pi, \sigma)}
\newcommand{\Zps}{Z_{\pi, \sigma}}
\newcommand{\Ztil}{\Zps}
\newcommand{\At}{\mathcal{A}_t}
\newcommand{\Wtil}{\widetilde{W}}
\newcommand{\cE}{\mathcal{E}}

\newcommand{\bllarge}{\mathsf{BL}^{\mathbb{L}}}
\newcommand{\blsmall}{\mathsf{BL}^{\mathbb{S}}}
%\newcommand{\blunion}{\cup \mathsf{bl}}
%\newcommand{\bllargeunion}{\cup \bllarge}
%\newcommand{\blsmallunion}{\cup \blsmall}
\newcommand{\blone}{\BL^{(1)}}
\newcommand{\bltwo}{\BL^{(2)}}
%\newcommand{\blthree}{\BL^{(3)}}
\newcommand{\blt}{\BL^{(t)}}
%\newcommand{\bltp}{\BL^{(t')}}

\newcommand{\blocksize}{\dimtwo \sqrt{\frac{\dimone}{N} \log (\dimone \dimtwo) }}

\newcommand\blfootnote[1]{%
  \begingroup
  \renewcommand\thefootnote{}\footnote{#1}%
  \addtocounter{footnote}{-1}%
  \endgroup
}

%%%%%%%%%%%%%%%%%%%%%%%%%%%%%%%%%%%%%%%%%%%%%%%%%%%%%%%%%%%%%%%%%%%%%%


\begin{document}

\begin{center}

{\bf{\LARGE{FedSKETCH: Communication-Efficient and Private \\\vspace*{.2in} Federated Learning via Sketching}}}
\vspace*{.2in}

{{
%\begin{tabular}{ccc}
%Farzin Haddadpour &
%\end{tabular}
}}
%\begin{tabular}{c}
%School of Electrical Engineering and Computer Science\\
%The Pennsylvania State University\\
%University Park, PA, USA \\
%\texttt{\{fxh18\}@psu.edu}
%\end{tabular}


\vspace*{.2in}



{\large{
\begin{tabular}{c}
%$^\star$
% &  $^\dagger$ 
% & $^{\dagger, \ddagger}$
\end{tabular}
}}
%\vspace*{.2in}

\begin{tabular}{c}
%Microsoft Bing 
%\texttt{Please do not distribute}
%$^\star$
%Department of Electrical Engineering and Computer Sciences, UC Berkeley$^\dagger$ \\
%Department of Statistics, UC Berkeley$^\ddagger$
\end{tabular}

\vspace*{.2in}


\date{\today}

\end{center}

%%%%%%%%%%%%%%%%%%%%%%%%%%%%%%%%%%%%%%%%%%%%%%%%%%%%
\begin{abstract}
Communication complexity and privacy are the two key challenges in Federated Learning where the goal is to perform a distributed learning through a large volume of devices. 
In this work, we introduce \texttt{FedSKETCH} and \texttt{FedSKETCHGATE} algorithms to address both challenges in Federated learning jointly, where these algorithms are intended to be used for homogeneous and heterogeneous data distribution settings respectively. The key idea is to compress the accumulation of local gradients using count sketch, therefore, the server does not have access to the gradients themselves which provides privacy. Furthermore, due to the lower dimension of sketching used, our method exhibits communication-efficiency property as well. 
We provide, for the aforementioned schemes, sharp convergence guarantees. 
 Finally, we back up our theory with various set of experiments.
\end{abstract}
%%%%%%%%%%%%%%%%%%%%%%%%%%%%%%%%%%%%%%%%%%%%%%%%%%%%
\section{Introduction}
Increasing applications in machine learning include the learning of a complex model across a large amount of devices in a distributed manner.
In the particular case of federated learning, the training data is stored across these multiple devices and can not be centralized.
Two natural problems arise from this setting. 
First, communications bottlenecks appear when a central server and the multiple devices must exchange gradient-informed quantities.
Then, privacy-related issues due to the protection of the sensitive individual data must be taken into account.

The former has extensively been tackled via quantization \cite{alistarh2017qsgd}, sparsification \cite{wangni2018gradient} and compression \cite{bernstein2018signsgd} methods yielding to a drastic reduction of the number of bits required to communicate those gradient-related information.
Solving the privacy issue has been widely executed injecting an additional layer of random noise in order to respect differential-privacy property of the method.


With the focus of communication-efficiency, \cite{ivkin2019communication} proposes a distributed SGD algorithm using sketching and they provide the convergence analysis in homogeneous data distribution setting. 

Also with focus on privacy,  in~\cite{li2019privacy}, the authors derive a single framework in order to tackle these issues jointly and introduce \texttt{DiffSketch} based on the Count Sketch operator. Compression and privacy is performed using random hash functions such that no third parties are able to access the original data. Yet, \cite{li2019privacy} does not provide the convergence analysis for the \texttt{DiffSketch} in Federated setting. In this work, we provide a thorough convergence analysis for the Federated Learning using sketching.

The main contributions of this paper are summarized as follows:
\begin{itemize}
    \item Based on the current compression methods, we provide a new algorithm -- \texttt{HEAPRIX} -- that displays an unbiased estimator of the full gradient we ought to communicate to the central parameter server. We theoretically show that \texttt{HEAPRIX} jointly reduces the cost of communication between devices and server, preserves privacy and is unbiased.
    
    \item We develop a general algorithm for communication-efficient and privacy preserving federated learning based on this novel compression algorithm. 
Those methods, namely \texttt{FedSKETCH} and \texttt{FedSKETCHGATE}, are derived under \textit{homogeneous} and \textit{heterogeneous} data distribution settings.
   
    \item Non asymptotic analysis of our method is established for convex, \pl\: (generalization of strongly-convex) and nonconvex functions in Theorem \ref{thm:homog_case} and Theorem \ref{thm:hetreg_case} for respectively the i.i.d. and non i.i.d. case,  and highlight an improvement in the number of iteration required to achieve a stationarity point.
\end{itemize}
\section{Related Work}
In this section, we provide a summary of the prior related research efforts as follows:

\paragraph{ Local SGD with Periodic Averaging:}
Compared to baseline SGD where model averaging happens in every iteration, the main idea behind \emph{Local SGD with periodic averaging} comes from the intuition of variance reduction by periodic model averaging \cite{zhang2016parallel} with purpose of saving communication rounds. While Local SGD has been proposed in \cite{mcmahan2016communication,konevcny2016federated} under the title of Federated Learning Setting, the convergence analysis of Local SGD is studied in~\cite{zhou2017convergence,yu2018parallel,stich2018local,wang2018cooperative}. The convergence analysis of Local SGD is improved in the follow up works~\cite{haddadpour2019trading,basu2019qsparse,haddadpour2019convergence,bayoumi2020tighter,stich2019error} in majority for homogeneous data distribution setting. The convergence analysis is further extended to heterogeneous setting, wherein studied under the title of \emph{Federated Learning}, with improved rates in~\cite{yu2019linear,li2019convergence,sahu2018convergence,liang2019variance,haddadpour2019convergence,karimireddy2019scaffold}. 

Additionally, a few recent Federated Learning/Local SGD with adaptive gradient methods can be found in \cite{reddi2020adaptive,}.

\todo{Revise this section!}


\paragraph{Gradient Compression Based Algorithms for Distributed Setting:} \cite{ivkin2019communication} develop a solution for leveraging sketches of full gradients in a distributed setting while training a global model using SGD \cite{robbins1951stochastic, bottou2008tradeoffs}. They introduce \texttt{Sketched-SGD} and establish a communication complexity of order $\mathcal{O}(\log(d))$ (per round) where $d$ is the dimension of the parameters, i.e. the dimension of the gradient.
Other recent solutions to reduce the communication cost include quantized gradient as developed in \cite{alistarh2017qsgd,lin2017deep,stich2018sparsified,horvath2019stochastic}. 
Yet, their dependence on the number of devices $p$ makes them harder to be used in some settings. Additionally, there are other research efforts such as \cite{haddadpour2020federated,reisizadeh2019fedpaq,basu2019qsparse,horvath2019stochastic} that exploit compression in Federated Learning or distributed communication-efficient optimization. 
Finally, the recent work in \cite{horvath2020better} exploits variance reduction technique with compression jointly in distributed optimization.



\paragraph{Privacy-preserving Setting:} Differentially private methods for federated learning have been extensively developed and studied in \cite{li2019privacy,liu2019enhancing} recently. 

The remaining of the paper is organized as follows.
Section \ref{sec:problem} gives a formal presentation of the general problem. 
Section \ref{sec:compression} describes the various compression algorithms used for communication efficiency and privacy preservation, and introduces our new compression method.
The training algorithms are provided in Section \ref{sec:algos} and their respective analysis in the strongly-convex or nonconvex cases are provided Section \ref{sec:cnvg-an}.

\textbf{Notation:} For the rest of the paper we indicate the number of communication rounds and number of bits per round per device with $R$ and $B$ respectively. For the rest of the paper we indicate the count sketch of any vector $\boldsymbol{x}$ with $\mathbf{S}(\boldsymbol{x})$.








 

%\section{Related Work}



%\section{Gradient Descent based Algorithm and Main Results}

%\subsection{Distributed SGD (Baseline)}

%%%%%%%%%%%%%%%%%%%%%%%%%%%%%%%%%%%%%%%%%%%%%%%%%%%%%%%%
%%%%%%%%%%%%%%%%%%%%%%%%%%%%%%%%%%%%%%%%%%%%%%%%%%%%%%%%

\section{Problem Setting}
In this paper our goal is to solve the following optimization problem using $p$ distributed devices:
\begin{align}
    f(\boldsymbol{x})\triangleq \left[\min_{\boldsymbol{x}\in \mathbb{R}^{d}}\frac{1}{p}\sum_{j=1}^{p}F_j(\boldsymbol{x})\right]
\end{align}
where $F_j(\boldsymbol{x})=\mathbb{E}_{\xi\in\mathcal{D}_j}\left[f_j\left(\boldsymbol{x},\xi\right)\right]$ is the local cost function at device $j$.

\section{Federated Learning via Sketching}
In the following we provide two sections, starting with differential private and communication algorithm using sketches. In the following subsection, we present communication-efficient variant of algorithm provided in first section.

\subsection{Deferentially private and communication efficient algorithms}


%%%%%%%%%%%%%%%%%%%%%%%%%%%%%%%%%%%%%%
%%%%%%%%%%%%%%%%%%%%%%%%%%%%%%%%%%%%%%%
\begin{algorithm}[H]
\caption{\texttt{CS}: Count Sketch to compress ${\mathbf{g}}\in\mathbb{R}^{d}$. }\label{Alg:csketch}
\begin{algorithmic}[1]
\State \textbf{Inputs:} ${\mathbf{g}}\in\mathbb{R}^{d}, t, k, \mathbf{S}_{t\times k}, h_i (1\leq i\leq t), sign_i (1\leq i\leq t)$
\State \textbf{Compress vector $\tilde{\mathbf{g}}\in\mathbb{R}^{d}$ into $\mathbf{S}\left(\tilde{\mathbf{g}}\right)$:}
\State \textbf{for} $\mathbf{g}_i\in\mathbf{g}$ \textbf{do}
\State \quad\textbf{for $j=1,\cdots,t$ do}
\State \quad\quad $\mathbf{S}[j][h_j(i)]=\mathbf{S}[j-1][h_{j-1}(i)]+\text{sign}_j(i).\mathbf{g}_i$ 
\State \quad\textbf{end for}
\State \textbf{end for}
\State \textbf{return} $\mathbf{S}_{t\times k}$
\State \textbf{Query} $\mathbf{g}_S\in\mathbb{R}^d$ \textbf{from $\mathbf{S(g)}$:}
\State \textbf{for} $i=1,\ldots,d$ \textbf{do}
\State \quad\quad $\mathbf{S}_\mathbf{g}=\text{Median}\{\text{sign}_j(i).\mathbf{S}[j][h_j(i)]:1\leq j\leq t\}$ 
\State \textbf{end for}
\State \textbf{Output:} $\mathbf{S}\left(\mathbf{g}\right)$
\belhal{What is this function $\mathbf{S}(\cdot)$?. Do you mean the matrix $\mathbf{S}_g$ ? }
\vspace{- 0.1cm}
\end{algorithmic}
\end{algorithm}
%%%%%%%%%%%%%%%%%%%%%%%%%%%%%%%%%%%%%%%%%%




%%%%%%%%%%%%%%%%%%%%%%%%%%%%%%%%%%%%%%%%%%%%%%%%%%%%%%%%%%%
%%%%%%%%%%%%%%%%%%%%%%%%%%%%%%%%%%%%%%%%%%%%%%%%%%%%%%%%%%%%
\begin{algorithm}[H]
\caption{\texttt{FEDSKETCH}($R$, $\tau, \eta, \gamma$): Private Federated Learning with Sketching. }\label{Alg:PFLHom}
\begin{algorithmic}[1]
\State \textbf{Inputs:} $\boldsymbol{x}^{(0)}$ as an initial  model shared by all local devices, the number of communication rounds $R$, the the number of local updates $\tau$, and global and local learning rates $\gamma$ and $\eta$, respectively
%\State Server chooses a subset $\mathcal{P}_0$ of $K$ devices at random (device $j$ is chosen with probability $q_j$);
\State \textbf{for $r=0, \ldots, R-1$ do}
\State $\qquad$\textbf{parallel for device $j=1,\ldots,n$ do}:
\State $\qquad\quad$ Set $\boldsymbol{x}^{(r)}=\boldsymbol{x}^{(r-1)}-\gamma\underline{\mathbf{S}}^{(r)}$
\State $\qquad\quad$ Set $\boldsymbol{x}_j^{(0,r)}=\boldsymbol{x}^{(r)}$ 
\State $\qquad\quad $\textbf{for} $c=0,\ldots,\tau-1$ \textbf{do}
\State $\qquad\quad\quad$ Sample a mini-batch $\xi_j^{(\ell,r)}$ and compute $\tilde{\mathbf{g}}_{j}^{(\ell,r)}\triangleq\nabla{f}_j(\boldsymbol{x}^{(\ell,r)}_j,\xi_j^{(c,r)})$
\State $\qquad\quad\quad$ $\boldsymbol{x}^{(\ell+1,r)}_{j}=\boldsymbol{x}^{(\ell,r)}_j-\eta~ \tilde{\mathbf{g}}_{j}^{(c,r)}$ \label{eq:update-rule-alg}
\State $\qquad\quad$\textbf{end for}
\State $\qquad\quad\quad$Device $j$ sends $\mathbf{S}\left[\boldsymbol{x}_j^{(0,r)}-~{\boldsymbol{x}}_{j}^{(\tau,r)}\right]$ back to the server.
\State $\qquad$Server \textbf{computes} 
\State $\qquad\qquad {\mathbf{S}}^{(r)}=\frac{1}{p}\sum_{j=1}\mathbf{S}\left[\boldsymbol{x}_j^{(0,r)}-~{\boldsymbol{x}}_{j}^{(\tau,r)}\right]$ and \textbf{broadcasts} ${\mathbf{S}}^{(r)}$ to all devices.
%\State $\qquad$Sever chooses a set of devices  $\mathcal{P}_t$ with distribution $q_j$.
%\State $\qquad\quad$ \textbf{end if}
\State $\qquad$\textbf{end parallel for}
\State \textbf{end}
\State \textbf{Output:} ${\boldsymbol{x}}^{(R-1)}$
\vspace{- 0.1cm}
%\State \todo{How about we call the algorithm local SGD with decoupled rates (LSDR) and specialize the analysis for IID and non-IID cases}
\end{algorithmic}
\end{algorithm}
%%%%%%%%%%%%%%%%%%%%%%%%%%%%%%%%%%%%%%%%%%

%%
%%%%%%%%%%%%%%%%%%%%%%%%%%%%%%%%%%%%%%%%%%%%%%%%%%%%%%%%%%%%
\begin{algorithm}[H]
\caption{\texttt{FEDSKETCHgt}($R$, $\tau, \eta, \gamma$): Private Federated Learning with Sketching and gradient tracking. }\label{Alg:PFLHet}
\begin{algorithmic}[1]
\State \textbf{Inputs:} $\boldsymbol{x}^{(0)}$ as an initial  model shared by all local devices, the number of communication rounds $R$, the the number of local updates $\tau$, and global and local learning rates $\gamma$ and $\eta$, respectively
%\State Server chooses a subset $\mathcal{P}_0$ of $K$ devices at random (device $j$ is chosen with probability $q_j$);
\State \textbf{for $r=0, \ldots, R-1$ do}
\State $\qquad$\textbf{parallel for device $j=1,\ldots,n$ do}:
\State $\qquad\quad$ Set $\mathbf{c}_j^{(r)}=\mathbf{c}_j^{(r-1)}-\frac{1}{\tau}\left(\mathbf{S}^{(r-1)}-\mathbf{S}^{(r-1)}_j\right)$
\State $\qquad\quad$ Set $\boldsymbol{x}_j^{(0,r)}=\boldsymbol{x}^{(r-1)}-\gamma{\mathbf{S}}^{(r-1)}$ 
\State $\qquad\quad $\textbf{for} $\ell=0,\ldots,\tau-1$ \textbf{do}
\State $\qquad\quad\quad$ Sample a minibatch $\xi_j^{(\ell,r)}$ and compute $\tilde{\mathbf{g}}_{j}^{(\ell,r)}\triangleq\nabla{f}_j(\boldsymbol{x}^{(\ell,r)}_j,\xi_j^{(\ell,r)})$
\State $\qquad\quad\quad$ $\boldsymbol{x}^{(\ell+1,r)}_{j}=\boldsymbol{x}^{(\ell,r)}_j-\eta~\left( \tilde{\mathbf{g}}_{j}^{(\ell,r)}-\mathbf{c}_j^{(r)}\right)$ \label{eq:update-rule-alg}
\State $\qquad\quad$\textbf{end for}
\State $\qquad\quad\quad$Device $j$ sends $\mathbf{S}^{(r)}_j\triangleq\mathbf{S}\left[\boldsymbol{x}_j^{(0,r)}-~{\boldsymbol{x}}_{j}^{(\tau,r)}\right]$ back to the server.
\State $\qquad$Server \textbf{computes} 
\State $\qquad\qquad {\mathbf{S}}^{(r)}=\frac{1}{p}\sum_{j=1}\mathbf{S}^{(r)}_j$ and  \textbf{broadcasts} $\mathbf{S}^{(r)}$ to all devices.
%\State $\qquad$Sever chooses a set of devices  $\mathcal{P}_t$ with distribution $q_j$.
%\State $\qquad\quad$ \textbf{end if}
\State $\qquad$\textbf{end parallel for}
\State \textbf{end}
\State \textbf{Output:} ${\boldsymbol{x}}^{(R-1)}$
\vspace{- 0.1cm}
%\State \todo{How about we call the algorithm local SGD with decoupled rates (LSDR) and specialize the analysis for IID and non-IID cases}
\end{algorithmic}
\end{algorithm}
%%%%%%%%%%%%%%%%%%%%%%%%%%%%%%%%%%%%%%%%%%



%%%%%%%%%%%%%%%%%%%%%%%%%%%%%%%%%%%%%%%%%%%%%%%%%%%%%%%%%
%%%%%%%%%%%%%%%%%%%%%%%%%%%%%%%%%%%%%%%%%%%%%%%%%%%%%%%%%

\subsection{Communication-efficient algorithm}
Here we propose the communication-efficient algorithm:

%%%%%%%%%%%%%%%%%%%%%%%%%%%%%%%%%%
\begin{algorithm}[H]
\caption{\texttt{HEAVYMIX}~\cite{ivkin2019communication} }\label{Alg:sketch}
\begin{algorithmic}[1]
\State \textbf{Inputs:} $\mathbf{S}_{\mathbf{g}}$; parameter-$k$
\State \textbf{Compress vector $\tilde{\mathbf{g}}\in\mathbb{R}^{d}$ into $\mathbf{S}\left(\tilde{\mathbf{g}}\right)$:}
\State Query $\hat{\ell}_2^2=\left(1\pm 0.5\right)\left\|\mathbf{g}\right\|^2$ from sketch $\mathbf{S}_{\mathbf{g}}$
\State $\forall j$ query $\hat{\mathbf{g}}_j^2=\hat{\mathbf{g}}_j^2\pm \frac{1}{2k}\left\|\mathbf{g}\right\|^2$ from sketch $\mathbf{S}_{\mathbf{g}}$
\State $H=\{j|\hat{\mathbf{g}}_j\geq \frac{\hat{\ell}_2^2}{k}\}$ and $NH=\{j|\hat{\mathbf{g}}_j<\frac{\hat{\ell}_2^2}{k}\}$
\State Top$_k=H\cup rand_\ell(NH)$, where $\ell=k-\left|H\right|$
\State Second round of communication to get exact values of Top$_k$ 
\State \textbf{Output:} $\mathbf{g}_S:\forall j\in\text{Top}_k:\mathbf{g}_{Si}=\mathbf{g}_{i}$ and $\forall\notin\text{Top}_k: \mathbf{g}_{Si}=0$
%\vspace{- 0.1cm}
\end{algorithmic}
\end{algorithm}
%%%%%%%%%%%%%%%%%%%%%%%%%%%%%%%%%%%%%%%%%%



\begin{algorithm}[H]
\caption{\texttt{FEDSKETCH-II}($R$, $\tau, \eta, \gamma$): Communication-efficient Federated Learning via Sketching. }\label{Alg:ce-h}
\begin{algorithmic}[1]
\State \textbf{Inputs:} $\boldsymbol{w}^{(0)}$ as an initial  model shared by all local devices, the number of communication rounds $R$, the the number of local updates $\tau$, and global and local learning rates $\gamma$ and $\eta$, respectively
%\State Server chooses a subset $\mathcal{P}_0$ of $K$ devices at random (device $j$ is chosen with probability $q_j$);
\State \textbf{for $r=0, \ldots, R-1$ do}
\State $\qquad$\textbf{parallel for device $j=1,\ldots,n$ do}:
\State $\qquad\quad$ Set $\boldsymbol{w}^{(r)}=\boldsymbol{w}^{(r-1)}-\gamma\underline{\mathbf{S}}^{(r)}$
\State $\qquad\quad$ Set $\boldsymbol{w}_j^{(0,r)}=\boldsymbol{w}^{(r)}$ 
\State $\qquad\quad $\textbf{for} $\ell=0,\ldots,\tau-1$ \textbf{do}
\State $\qquad\quad\quad$ Sample a mini-batch $\xi_j^{(\ell,r)}$ and compute $\tilde{\mathbf{g}}_{j}^{(\ell,r)}\triangleq\nabla{f}_j(\boldsymbol{w}^{(\ell,r)}_j,\xi_j^{(\ell,r)})$
\State $\qquad\quad\quad$ $\boldsymbol{w}^{(\ell+1,r)}_{j}=\boldsymbol{w}^{(\ell,r)}_j-\eta~ \tilde{\mathbf{g}}_{j}^{(\ell,r)}$ \label{eq:update-rule-alg}
\State $\qquad\quad$\textbf{end for}
\State $\qquad\quad\quad$Device $j$ sends $\mathbf{S}_j^{(r)}=\mathbf{S}\left(\boldsymbol{w}_j^{(0,r)}-~{\boldsymbol{w}}_{j}^{(\tau,r)}\right)$ back to the server.
\State $\qquad$Server \textbf{computes} 
\State $\qquad\qquad {\mathbf{S}}^{(r)}=\frac{1}{p}\sum_{j=1}^n\mathbf{S}\left(\boldsymbol{w}_j^{(0,r)}-~{\boldsymbol{w}}_{j}^{(\tau,r)}\right)$ %and \textbf{broadcasts} ${\mathbf{S}}^{(r)}$ to all devices.
\State $\qquad\qquad$ Sever runs $\underline{\mathbf{S}}^{(r)}= \texttt{HEAVYMIX}(\mathbf{S}^{(r)})$.
%\State $\qquad\quad$ \textbf{end if}
\State $\qquad$\textbf{end parallel for}
\State \textbf{end}
\State \textbf{Output:} ${\boldsymbol{w}}^{(R-1)}$
\vspace{- 0.1cm}
%\State \todo{How about we call the algorithm local SGD with decoupled rates (LSDR) and specialize the analysis for IID and non-IID cases}
\end{algorithmic}
\end{algorithm}




\subsection{Differentially Private Property}
\begin{definition}
A randomized mechanism $\mathcal{O}$ satisfies $\epsilon-$differential privacy, if for input data ${S}_1$ and ${S}_2$ differing by up to one element, and for any output $D$ of $\mathcal{O}$,
\begin{align}
    \Pr\left[\mathcal{O}(S_1)\in D\right]\leq \exp{\left(\epsilon\right)}\Pr\left[\mathcal{O}(S_2)\in D\right] 
\end{align}
\end{definition}
\todo{Add explanations that this scheme induces local privacy!}

\begin{assumption}[Input vector distribution]\label{assu:invecdist}
For the purpose of privacy analysis, similar to \cite{,}, we suppose that for any input vector $S$ with length $|S|=l$, each element $s_i\in S$ is drawn i.i.d. from a Gaussian distribution: $s_i\sim \mathcal{N}(0,\sigma^2)$, and bounded by a large probability:  $|s_i|\leq C, 1\leq i\leq p$ for some positive constant $C>0$.    
\end{assumption}

\begin{theorem}[$\epsilon-$ differential privacy of count sketch, \cite{li2019privacy}]
For a sketching algorithm $\mathcal{O}$ using Count Sketch $\mathbf{S}_{t\times k}$ with $t$ arrays of $k$ bins, for any input vector $S$ with length $l$ satisfying Assumption~\ref{assu:invecdist}, $\mathcal{O}$ achieves $t.\ln \left(1+\frac{\alpha C^2 k(k-1)}{\sigma^2(l-2)}(1+\ln(l-k) )\right)-$differential privacy with high probability, where $\alpha$ is a positive constant satisfying $\frac{\alpha C^2 k(k-1)}{\sigma^2(l-2)}(1+\ln(l-k) )\leq \frac{1}{2}-\frac{1}{\alpha}$.
\end{theorem}
The proof of this theorem can be found in \cite{li2019privacy}.

\section{Convergence analysis for differential privacy algorithms}

\subsection{Assumptions}


\begin{assumption}[Smoothness and Lower Boundedness]\label{Assu:1}
The local objective function $f_j(\cdot)$ of $j$th device is differentiable for $j\in [m]$ and $L$-smooth, i.e., $\|\nabla f_j(\boldsymbol{u})-\nabla f_j(\mathbf{v})\|\leq L\|\boldsymbol{u}-\mathbf{v}\|,\: \forall \;\boldsymbol{u},\mathbf{v}\in\mathbb{R}^d$. Moreover, the optimal objective function $f(\cdot)$ is bounded below by ${f^*} = \min_{\boldsymbol{x}} f(\boldsymbol{x})>-\infty$. 
\end{assumption}

\begin{assumption}[\pl]\label{assum:pl}
A function $f(\boldsymbol{x})$ satisfies the \pl~ condition with constant $\mu$ if $\frac{1}{2}\|\nabla f(\boldsymbol{x})\|_2^2\geq \mu\big(f(\boldsymbol{x})-f(\boldsymbol{x}^*)\big),\: \forall \boldsymbol{x}\in\mathbb{R}^d $ with $\boldsymbol{x}^*$ is an optimal solution.
\end{assumption}

\begin{property}[\cite{li2019privacy}]
For our proof purpose we will need the following crucial properties of the count sketch described in Algorithm~\ref{Alg:csketch}, for any real valued vector $\mathbf{x}\in \mathbb{R}^{d}$:
\begin{itemize}
    \item[1)] \emph{Unbiased estimation}: As it is also mentioned in \cite{li2019privacy}, we have:
    \begin{align}
        \mathbb{E}_{\mathbf{S}}\left[\mathbf{S}\left[\mathbf{x}\right]\right]=\mathbf{x}
    \end{align}
    \belhal{The biased case is interesting, no hopes dealing with it for now?}
    \item[2)] \emph{Bounded variance}: With $k=O\left(\frac{e}{\mu^2}\right)$, we have the following bound:
    \begin{align}
        \mathbb{E}_{\mathbf{S}}\left[\left\|\mathbf{S}\left[\mathbf{x}\right]-\mathbf{x}\right\|_2^2\right]\leq \mu^2 d\left\|\mathbf{x}\right\|_2^2
    \end{align}
\end{itemize}
\end{property}


\subsection{Convergence of  \texttt{FEDSKETCH-I} in homogeneous setting.} 
Now we focus on the homogeneous case in which the stochastic local gradient of each worker is an unbiased estimator of the global gradient.


\begin{assumption}[Bounded Variance]\label{Assu:1.5}
For all $j\in [m]$, we can sample an independent mini-batch $\ell_j$   of size $|\xi_j^{(\ell,r)}| = b$ and compute an unbiased stochastic gradient  $\tilde{\mathbf{g}}_j = \nabla f_j(\boldsymbol{w}; \xi_j), \mathbb{E}_{\xi_j}[\tilde{\mathbf{g}}_j] = \nabla f(\boldsymbol{w})=\mathbf{g}$ with  the variance bounded is bounded by a constant $\sigma^2$, i.e., $
\mathbb{E}_{\xi_j}\left[\|\tilde{\mathbf{g}}_j-\mathbf{g}\|^2\right]\leq \sigma^2$.
\end{assumption}


\begin{theorem}[General non-convex]
Given $0<k=O\left(\frac{e}{\mu^2}\right)\leq d$
and running Algorithm~\ref{Alg:PFLHom} with sketch of size $c=O\left(k\log \frac{d R}{\delta}\right)$,  under Assumptions~\ref{Assu:1} and \ref{Assu:1.5}, if 
\begin{align}
   1\geq {\tau L^2\eta^2\tau}+(\frac{\mu^2 d}{p}+1)\eta\gamma L{\tau}\label{eq:cnd-lrs-h} 
\end{align}
with probability at least $1-\delta$, we have:
\begin{align}\label{eq:thm1-result}
    \frac{1}{R}\sum_{r=0}^{R-1}\left\|\nabla f({\boldsymbol{x}}^{(r)})\right\|_2^2\leq \frac{2\left(f(\boldsymbol{x}^{(0)})-f(\boldsymbol{x}^{*})\right)}{\eta\gamma\tau R}+\frac{L\eta\gamma(\frac{\mu^2 d}{p}+1)}{p}\sigma^2+{L^2\eta^2\tau }\sigma^2
\end{align}
\end{theorem}


\begin{corollary}[Linear speed up] 
In Eq.~(\ref{eq:thm1-result}) by letting $\eta\gamma=O\left(\frac{1}{L}\sqrt{\frac{p}{R\tau\left(\frac{\mu^2 d}{p}+1\right)}}\right)$, and for $\gamma\geq p$  convergence rate reduces to:
\begin{align}
    \frac{1}{R}\sum_{r=0}^{R-1}\left\|\nabla f({\boldsymbol{w}}^{(r)})\right\|_2^2&\leq O\left(\frac{L\sqrt{\left(\frac{\mu^2 d}{p}+1\right)}\left(f(\boldsymbol{w}^{(0)})-f(\boldsymbol{w}^{*})\right)}{\sqrt{pR\tau}}+\frac{\left(\sqrt{\left(\frac{\mu^2 d}{p}+1\right)}\right)\sigma^2}{\sqrt{pR\tau}}+\frac{p\sigma^2}{R\left(\frac{\mu^2 d}{p}+1\right)\gamma^2}\right)\label{eq:convg-error}
\end{align}
Note that according to Eq.~(\ref{eq:convg-error}), if we pick  a fixed constant value for  $\gamma$, in order to achieve an $\epsilon$-accurate solution, $R=O\left(\frac{1}{\epsilon}\right)$ communication cost and $\tau=O\left(\frac{\left(\frac{\mu^2 d}{p}+1\right)}{p\epsilon}\right)$ are necessary.

\end{corollary}




\begin{remark}\label{rmk:cnd-lr}

Condition in Eq.~(\ref{eq:cnd-lrs-h}) can be rewritten as 
\begin{align}
    \eta&\leq \frac{-\gamma L\tau\left(\frac{\mu^2 d}{p}+1\right)+\sqrt{\gamma^2 \left(L\tau\left(\frac{\mu^2 d}{p}+1\right)\right)^2+4L^2\tau^2}}{2L^2\tau^2}\nonumber\\
    &= \frac{-\gamma L\tau\left(\frac{\mu^2 d}{p}+1\right)+L\tau\sqrt{\left(\frac{\mu^2 d}{p}+1\right)^2\gamma^2 +4}}{2L^2\tau^2}\nonumber\\
    &=\frac{\sqrt{\left(\frac{\mu^2 d}{p}+1\right)^2\gamma^2 +4}-\left(\frac{\mu^2 d}{p}+1\right)\gamma}{2L\tau}\label{eq:lrcnd}
\end{align}
So based on Eq.~(\ref{eq:lrcnd}), if we set $\eta=O\left(\frac{1}{L\gamma}\sqrt{\frac{p}{R\tau\left(\frac{\mu^2 d}{p}+1\right)}}\right)$, this implies that:
\begin{align}
    R\geq \frac{\tau p}{\left(\frac{\mu^2 d}{p}+1\right)\gamma^2\left(\sqrt{\left(\frac{\mu^2 d}{p}+1\right)^2\gamma^2+4}-\left(\frac{\mu^2 d}{p}+1\right)\gamma\right)^2}\label{eq:iidexact}
\end{align}
We note that $\gamma^2\left(\sqrt{\left(\frac{\mu^2 d}{p}+1\right)^2\gamma^2+4}-\left(\frac{\mu^2 d}{p}+1\right)\gamma\right)^2=\Theta(1)\leq 5 $ therefore even for $\gamma\geq p$ we need to have 
\begin{align}
    R\geq \frac{\tau p}{5\left(\frac{\mu^2 d}{p}+1\right)}=O\left(\frac{\tau p}{\left(\frac{\mu^2 d}{p}+1\right)}\right)
\end{align}
\textbf{Therefore for the choice of $\tau=O\left(\frac{\frac{\mu^2 d}{p}+1}{p\epsilon}\right)$ we need to have $R=O\left(\frac{1}{\epsilon}\right)$.}
\end{remark}


\begin{corollary}[Total communication cost]
As a consequence of Remark~\ref{rmk:cnd-lr}, the total communication cost per-worker becomes \begin{align}
O\left(Rc\right)&=O\left(Rk\log \left(\frac{d R}{\delta}\right)\right)=O\left(\frac{k }{\epsilon}\log \left(\frac{d }{\epsilon\delta}\right)\right)
\end{align}
We note that this result in addition to improving over the communication complexity of federated learning of the state-of-the-art from $O\left(\frac{d}{\epsilon}\right)$ in \cite{karimireddy2019scaffold,wang2018cooperative,liang2019variance} to $O\left(\frac{k p}{\epsilon}\log \left(\frac{d p}{\epsilon\delta}\right)\right)$, it also implies differential privacy. As a result, total communication cost is 
$$cpR=O\left(\frac{k p}{\epsilon}\log \left(\frac{d }{\epsilon\delta}\right)\right).$$ 
\end{corollary}

\begin{remark}
We note that the state-of-the-art in \cite{karimireddy2019scaffold} the total communication cost is 
\begin{align}
    cpR&=O\left(pd\left(\frac{1}{\epsilon}\right) \right)=O\left(\frac{pd}{\epsilon}\right) 
\end{align}
We improve this result, in terms of dependency to $d$, to 
\begin{align}
    cpR=O\left(\frac{k p}{\epsilon}\log \left(\frac{d }{\epsilon\delta}\right)\right)
\end{align}
In comparison to \cite{ivkin2019communication}, we improve the total communication per worker from $Rc=O\left(\frac{k }{\epsilon^2}\log \left(\frac{d }{\epsilon^2\delta}\right)\right)$ to $Rc=O\left(\frac{k }{\epsilon}\log \left(\frac{d }{\epsilon\delta}\right)\right)$.

\end{remark}

\begin{remark}
It is worthy to note that most of the available communication-efficient algorithm with quantization or compression only consider communication-efficiency from devices to server. However, Algorithm~\ref{Alg:PFLHom} also improves the communication efficiency from server to devices as well. 
\end{remark}
%%%%%%%%%%%%%%%%%%%%%%%%%%%%%%%%%%%%%%%%%
%%%%%%%%%%%%%%%%%%%%%%%%%%%%%%%%%%%%%%%%%
\begin{theorem}[Strongly convex or \pl]
Given $0<k=O\left(\frac{e}{\mu^2}\right)\leq d$
and running Algorithm~\ref{Alg:PFLHom} with sketch of size $c=O\left(k\log \frac{d R}{\delta}\right)$,  under Assumptions~\ref{Assu:1} and \ref{Assu:1.5},and the choice of learning rate $\eta=\frac{1}{L\gamma (\frac{\mu^2d}{p}+1) \tau}$ with probability at least $1-\delta$, we have:
\begin{align}
                \mathbb{E}\Big[f({\boldsymbol{w}}^{(R)})-f({\boldsymbol{w}}^{(*)})\Big]&\leq \exp{-\left(\frac{ R}{\kappa (\frac{\mu^2d}{p}+1)}\right)}\Big[f({\boldsymbol{w}}^{(0)})-f({\boldsymbol{w}}^{(*)})\Big]+\left(\frac{1}{2\gamma^2 {(\frac{\mu^2d}{p}+1)}^2 }+\frac{1}{2p}\right)\frac{\sigma^2}{\mu\tau}
\end{align}
\end{theorem}

\begin{remark}[linear speed up]
To achieve the convergence error of $\epsilon$, we need to have $R=O\left(\kappa(\frac{\mu^2d}{p}+1)\log\frac{1}{\epsilon}\right)$ and $\tau=\left(\frac{1}{\epsilon}\right)$. This leads to the total communication cost per worker of 
\begin{align}
cR&=O\left(k\kappa(\frac{\mu^2d}{p}+1)\log\left(\frac{\kappa(\frac{\mu^2d^2}{p}+d)\log\frac{1}{\epsilon}}{\delta}\right)\log\frac{1}{\epsilon} \right)
\end{align}
As a consequence, the total communication cost becomes:
\begin{align}
cpR&=O\left(k\kappa(\mu^2d+p)\log\left(\frac{\kappa(\frac{\mu^2d^2}{p}+d)\log\frac{1}{\epsilon}}{\delta}\right)\log\frac{1}{\epsilon} \right)
\end{align}
\end{remark}

\begin{remark}
We note that the state-of-the-art in \cite{karimireddy2019scaffold} the total communication cost is 
\begin{align}
    cpR=O\left(\kappa pd\log\left(\frac{1}{\epsilon}\right) \right)=O\left(\kappa pd\log\left(\frac{1}{\epsilon}\right)\right) 
\end{align}
We improve this result, in terms of dependency to $d$, to 
\begin{align}
    cpR=O\left(k\kappa(\mu^2d+p)\log\left(\frac{\kappa(\frac{\mu^2d^2}{p}+d)\log\frac{1}{\epsilon}}{\delta}\right)\log\frac{1}{\epsilon} \right)
\end{align}
Improving from $pd$ to $p+d$.
\end{remark}

\todo{Extending these results to general convex setting Later!}


\subsection{Convergence of  \texttt{} in the data heterogeneous setting.} 
\todo{TBA...}
\section{Convergence analysis for different sketching scheme}
We note that the main issue with Assumption~\ref{} is that since $d\neq 0$, you can not improve the convergence analysis. For this purpose, we propose Algorithm~\ref{}, where the proposed algorithm is not differentially private.

\subsection{Convergence of \texttt{FEDSKETCH-II} in the data homogeneous setting.} 
In this case, we use a different assumption as follows:
\begin{assumption}\label{Assu:2.5}
A (randomized) function, for $0<k\leq d$, $\text{Comp}_k:\mathbb{R}^{d}\rightarrow\mathbb{R}^{d}$ is called a compression operator, if we have 
\begin{align}
    \mathbb{E}\left[\left\|\boldsymbol{x}-\text{Comp}_k(\boldsymbol{x})\right\|^2_2\right]\leq \left(1-\frac{k}{d}\right)\left\|\boldsymbol{x}\right\|^2_2
\end{align}
\end{assumption}
\begin{remark}
Main distinction of Assumption~\ref{} from~\ref{} is that first we do not need unbiased estimation of compression. Additionally, unlike Assumption~\ref{}, if you let $k=d$, we have $\boldsymbol{x}=\text{Comp}_{k=d}(\boldsymbol{x})$.    
\end{remark}

We note that Algorithm~\ref{} satisfies this Assumption~\ref{} as shown in ~\cite{ivkin2019communication}.

\begin{theorem}[General non-convex]
Given $0<k=O\left(\frac{e}{\mu^2}\right)\leq d$
and running Algorithm~\ref{Alg:PFLHom} with sketch of size $c=O\left(k\log \frac{d R}{\delta}\right)$,  under Assumptions~\ref{Assu:1} and \ref{Assu:2.5}, if 
\begin{align}
       L^2\eta^2\tau^2+mL\tau\eta\left(1-\frac{k}{d}\right)+2\gamma L\eta\tau\left(2-\frac{k}{d}\right)-1\leq 0,\:\eta> \frac{1}{mL\tau},\label{eq:cnd-lrs-h-ii} 
\end{align}
with probability at least $1-\delta$, we have:
\begin{align}
    \frac{1}{R}\sum_{r=0}^{R-1}\left\|\nabla f({\boldsymbol{x}}^{(r)})\right\|_2^2\leq \frac{2 \mathbb{E}\left[f({\boldsymbol{x}}^{(0)})-f({\boldsymbol{x}}^{(*)})\right]}{R\tau \gamma \left({\eta}-\frac{1}{\tau mL}\right)}+\frac{2\eta^2\gamma L\left(2-\frac{k}{d}\right)\frac{\sigma^2}{p}}{ \left({\eta}-\frac{1}{\tau mL}\right)}+\frac{\eta^3L^2\tau}{\left({\eta}-\frac{1}{\tau mL}\right)}\sigma^2 
\end{align}
\end{theorem}
\begin{remark}[$k=d$]
\todo{TBA...}
\end{remark}

\begin{corollary}[Learning rate range]
Condition in Eq.~(\ref{}) can further simplified as 
\begin{align}
    \frac{1}{mL\tau}<\eta\leq \frac{-\left(m-\frac{mk}{d}+4\gamma-\frac{2\gamma k}{d}\right)+\sqrt{\left(m-\frac{mk}{d}+4\gamma-\frac{2\gamma k}{d}\right)^2+4}}{2L\tau}
\end{align}
We note that $m$ is a hyperparameter that we choose to pick the feasible range for learning rate. Now, if you set $\eta=\frac{1}{\gamma L}\sqrt{\frac{p}{R\tau\left(2-\frac{k}{d}\right)}}$ which implies the following:
\begin{itemize}
    \item  $\frac{1}{mL\tau}<\frac{1}{\gamma L}\sqrt{\frac{p}{R\tau\left(2-\frac{k}{d}\right)}} \implies R <\frac{m^2 p \tau}{\gamma^2\left(2-\frac{k}{d}\right)}$
    \item$\frac{1}{\gamma L}\sqrt{\frac{p}{R\tau\left(2-\frac{k}{d}\right)}}\leq \frac{-\left(m-\frac{mk}{d}+4\gamma-\frac{2\gamma k}{d}\right)+\sqrt{\left(m-\frac{mk}{d}+4\gamma-\frac{2\gamma k}{d}\right)^2+4}}{2L\tau} \implies R\geq \frac{p\tau}{\gamma^2\left(2-\frac{k}{d}\right)\left(-\left(m-\frac{mk}{d}+4\gamma-\frac{2\gamma k}{d}\right)+\sqrt{\left(m-\frac{mk}{d}+4\gamma-\frac{2\gamma k}{d}\right)^2+4}\right)^2}$
\end{itemize}
Therefore, we have the following range for the choice of $R$:
\begin{align}
    \frac{p\tau}{\gamma^2\left(2-\frac{k}{d}\right)\left(-\left(m-\frac{mk}{d}+4\gamma-\frac{2\gamma k}{d}\right)+\sqrt{\left(m-\frac{mk}{d}+4\gamma-\frac{2\gamma k}{d}\right)^2+4}\right)^2}\leq R<\frac{m^2 p \tau}{\gamma^2\left(2-\frac{k}{d}\right)}
\end{align}
\end{corollary}
\begin{corollary}
Based on Corollary~\ref{}, we choose $\eta=\frac{1}{\gamma}\sqrt{\frac{p}{R\tau\left(R\tau \left(2-\frac{k}{d}\right)\right)}}=\frac{n}{mL\tau}$ which also  implies $R=\frac{m^2p\tau}{\gamma^2\left(2-\frac{k}{d}\right)}$ with $1<n<m$, we have:
\begin{align}
        \frac{1}{R}\sum_{r=0}^{R-1}\left\|\nabla f({\boldsymbol{x}}^{(r)})\right\|_2^2&\leq \frac{2 \mathbb{E}\left[f({\boldsymbol{x}}^{(0)})-f({\boldsymbol{x}}^{(*)})\right]}{R\tau \gamma \left(\frac{n-1}{m\tau L}\right)}+\frac{2n^2\gamma L\left(2-\frac{k}{d}\right)\frac{\sigma^2}{p}}{m^2\tau^2L^2 \left(\frac{n-1}{m\tau L}\right)}+\frac{n^3L^2\tau}{m^3\tau^3L^3\left(\frac{n-1}{m\tau L}\right)}\sigma^2\nonumber\\
        &=\frac{2mL \mathbb{E}\left[f({\boldsymbol{x}}^{(0)})-f({\boldsymbol{x}}^{(*)})\right]}{\left(n-1\right)R \gamma }+\frac{2n^2\gamma \left(2-\frac{k}{d}\right)\sigma^2}{m\left(n-1\right) p\tau  }+\frac{n^3\sigma^2}{m^2\left(n-1\right)\tau}
\end{align}
Based on relation $R=\frac{m^2p\tau}{\gamma^2\left(2-\frac{k}{d}\right)}$ if we choose $\tau=\frac{\left(2-\frac{k}{d}\right)}{p\epsilon}$ and $m=np$ and $\gamma=m$ we have:
$$R=\frac{1}{\epsilon}$$ and 
\begin{align}
     \frac{1}{R}\sum_{r=0}^{R-1}\left\|\nabla f({\boldsymbol{x}}^{(r)})\right\|_2^2&\leq \frac{2\epsilon L \mathbb{E}\left[f({\boldsymbol{x}}^{(0)})-f({\boldsymbol{x}}^{(*)})\right]}{\left(n-1\right)}+\frac{2n\epsilon \sigma^2}{p\left(n-1\right)}+\frac{n\epsilon\sigma^2}{p\left(n-1\right)\left(2-\frac{k}{d}\right)}
\end{align}
\end{corollary}

\begin{theorem}
\todo{TBA...}
\end{theorem}

\todo{Convergence analysis for strongly convex objectives!}
\section{Experiments}
\section{Conclusion}
%%%%%%%%%%%%%%%%%%%%%%%%%%%%%%%%%%%%%%%%%%%%%%%%%%%%
\newpage
%\bibliographystyle{IEEEtran}
\bibliographystyle{plain}
\bibliography{references}
%%%%%%%%%%%%%%%%%%%%%%%%%%%%%%%%%%%%%%%%%%%%%%%%%%%%
\newpage
\appendix
\section{Appendix}
\section{Proof of main Theorems}
The proof of Theorem~\ref{thm:homog_case} follows directly from the results in~\cite{haddadpour2020federated}. For the sake of the completeness we review an assumptions from this reference for the quantiziation with their notation.

\begin{assumption}[\cite{haddadpour2020federated}]\label{Assu:quant}
The output of the compression operator $Q(\mathbf{x})$ is an unbiased estimator of its input $\mathbf{x}$, and its variance grows with the squared of the squared of $\ell_2$-norm of its argument, i.e., $\mathbb{E}\left[Q(\mathbf{x})\right]=\mathbf{x}$ and $\mathbb{E}\left[\left\|Q(\mathbf{x})-\mathbf{x}\right\|^2\right]\leq q\left\|\mathbf{x}\right\|^2$ .
\end{assumption}


\subsection{Proof of Theorem~\ref{thm:homog_case}}
Based on Assumption~\ref{Assu:quant} we have:
\begin{theorem}[\cite{haddadpour2020federated}]\label{thm:fromhaddad}
 Consider \texttt{FedCOM} in \cite{haddadpour2020federated}. Suppose that the conditions in Assumptions~\ref{Assu:1}, \ref{Assu:1.5} and \ref{Assu:quant} hold. If the local data distributions of all users are identical (homogeneous setting), then we have  
 \begin{itemize}
     \item \textbf{Nonconvex:}  By choosing stepsizes as $\eta=\frac{1}{L\gamma}\sqrt{\frac{p}{R\tau\left(\frac{q}{p}+1\right)}}$ and $\gamma\geq p$, the sequence of iterates satisfies  $\frac{1}{R}\sum_{r=0}^{R-1}\left\|\nabla f({\boldsymbol{w}}^{(r)})\right\|_2^2\leq {\epsilon}$ if we set
     $R=O\left(\frac{1}{\epsilon}\right)$ and $ \tau=O\left(\frac{\frac{q}{p}+1}{{p}\epsilon}\right)$.
     \item \textbf{Strongly convex or PL:}
      By choosing stepsizes as $\eta=\frac{1}{2L\left(\frac{q}{p}+1\right)\tau\gamma}$ and $\gamma\geq m$, we obtain that the iterates satisfy $\mathbb{E}\Big[f({\boldsymbol{w}}^{(R)})-f({\boldsymbol{w}}^{(*)})\Big]\leq \epsilon$ if  we set
     $R=O\left(\left(\frac{q}{p}+1\right)\kappa\log\left(\frac{1}{\epsilon}\right)\right)$ and $ \tau=O\left(\frac{1}{p\epsilon}\right)$.
     \item \textbf{Convex:} By choosing stepsizes as $\eta=\frac{1}{2L\left(\frac{q}{p}+1\right)\tau\gamma}$ and $\gamma\geq p$, we obtain that the iterates satisfy $ \mathbb{E}\Big[f({\boldsymbol{w}}^{(R)})-f({\boldsymbol{w}}^{(*)})\Big]\leq \epsilon$ if we set
     $R=O\left(\frac{L\left(1+\frac{q}{p}\right)}{\epsilon}\log\left(\frac{1}{\epsilon}\right)\right)$ and $ \tau=O\left(\frac{1}{p\epsilon^2}\right)$.
 \end{itemize}
\end{theorem}

\begin{proof}
Since the sketching \texttt{PRIVIX} and \texttt{HEAPRIX}, satisfy the Assumption~\ref{Assu:quant} with $q=\mu^2d$ and $q=\mu^2d-1$ respectively with probablity $1-\delta$.  Therefore, all the results in Theorem~\ref{thm:homog_case}, conclude from Theorem~\ref{thm:fromhaddad} with probability $1-\delta$ and plugging $q=\mu^2d$ and $q=\mu^2d-1$ respectively into the corresponding convergence bounds.
\end{proof}


\subsection{Proof of Theorem~\ref{thm:hetreg_case}}
For the heterogeneous setting, the results in~\cite{haddadpour2020federated} requires the following extra assumption that naturally holds for the sketching: 

\begin{assumption}[\cite{haddadpour2020federated}]\label{assum:009}
The compression scheme $Q$ for the heterogeneous data distribution setting satisfies the following condition $
    \mathbb{E}_Q[\|\frac{1}{m}\sum_{j=1}^m Q(\boldsymbol{x}_j)\|^2-\|Q(\frac{1}{m}\sum_{j=1}^m \boldsymbol{x}_j)\|^2]\leq G_q$.
\end{assumption}
We note that since sketching is a linear compressor, in the case of our algorithms for heterogeneous setting we have $G_q=0$. 

Next, we restate the Theorem in \cite{haddadpour2020federated} here as follows:

\begin{theorem}\label{thm:fromhaddad-het}
 Consider \texttt{FedCOMGATE} in \cite{haddadpour2020federated}. If Assumptions~\ref{Assu:1}, \ref{Assu:2}, \ref{Assu:quant}  and \ref{assum:009} hold, then even for the case the local data distribution of users are different  (heterogeneous setting) we have
 \begin{itemize}
     \item \textbf{Non-convex:} By choosing stepsizes as $\eta=\frac{1}{L\gamma}\sqrt{\frac{p}{R\tau\left(q+1\right)}}$ and $\gamma\geq p$, we obtain that the iterates satsify  $\frac{1}{R}\sum_{r=0}^{R-1}\left\|\nabla f({\boldsymbol{w}}^{(r)})\right\|_2^2\leq \epsilon$ if we set
     $R=O\left(\frac{q+1}{\epsilon}\right)$ and $ \tau=O\left(\frac{1}{p\epsilon}\right)$.
     \item \textbf{Strongly convex or PL:}
      By choosing stepsizes as $\eta=\frac{1}{2L\left(\frac{q}{p}+1\right)\tau\gamma}$ and ${\gamma\geq \sqrt{p\tau}}$, we obtain that the iterates satisfy $\mathbb{E}\Big[f({\boldsymbol{w}}^{(R)})-f({\boldsymbol{w}}^{(*)})\Big]\leq \epsilon$ if we set
      $R=O\left(\left(q+1\right)\kappa\log\left(\frac{1}{\epsilon}\right)\right)$ and $ \tau=O\left(\frac{1}{p\epsilon}\right)$.
     \item \textbf{Convex:}  By choosing stepsizes as $\eta=\frac{1}{2L\left(q+1\right)\tau\gamma}$ and ${\gamma\geq \sqrt{p\tau}}$, we obtain that the iterates satisfy $\mathbb{E}\Big[f({\boldsymbol{w}}^{(R)})-f({\boldsymbol{w}}^{(*)})\Big]\leq \epsilon$ if we set
     $R=O\left(\frac{L\left(1+q\right)}{\epsilon}\log\left(\frac{1}{\epsilon}\right)\right)$ and $ \tau=O\left(\frac{1}{p\epsilon^2}\right)$.
 \end{itemize}
 
\end{theorem}
\begin{proof}
Since the sketching \texttt{PRIVIX} and \texttt{HEAPRIX}, satisfy the Assumption~\ref{Assu:quant} with $q=\mu^2d$ and $q=\mu^2d-1$ respectively with probablity $1-\delta$.  Therefore, all the results in Theorem~\ref{thm:hetreg_case}, conclude from Theorem~\ref{thm:fromhaddad-het} with probability $1-\delta$ and plugging $q=\mu^2d$ and $q=\mu^2d-1$ respectively into the convergence bounds.
\end{proof}
%%%%%%%%%%%%%%%%%%%%%%%%%%%%%%%%%%%%%%%%%%%%%%%%
%%%%%%%%%%%%%%%%%%%%%%%%%%%%%%%%%%%%%%%%%%%%%%%%
\section{Convergence result for \texttt{FEDSKETCH} without memory}
From the $L$-smoothness gradient assumption on global objective, by using  $\underline{\mathbf{S}}^{(r)}=\tilde{\mathbf{g}}^{(r)}$ in inequality (\ref{eq:decent-smoothe}) we have:
\begin{align}
    f({\boldsymbol{x}}^{(r+1)})-f({\boldsymbol{x}}^{(r)})\leq -\gamma \big\langle\nabla f({\boldsymbol{x}}^{(r)}),\tilde{\mathbf{g}}^{(r)}\big\rangle+\frac{\gamma^2 L}{2}\|\tilde{\mathbf{g}}^{(r)}\|^2\label{eq:Lipschitz-c1}
\end{align}
We define the following:
\begin{align}
    \tilde{\mathbf{g}}_{\mathbf{S}}^{(r)}=\frac{\eta}{p}\sum_{j=1}^{p}\mathbf{S}\left[\sum_{c=0}^{\tau-1}\tilde{\mathbf{g}}_j^{(c,r)}\right]
\end{align}
Additionally, we define an auxiliary variable as 
\begin{align}
    \tilde{\mathbf{g}}^{(r)}=\frac{\eta}{p}\sum_{j=1}^{p}\left[\sum_{c=0}^{\tau-1}\tilde{\mathbf{g}}_j^{(c,r)}\right]
\end{align}
%%%%%%%%%%%%%%%%%%%%%%%%%%%%%%%%%%%%%%%%%%%%
By taking expectation on both sides of above inequality over sampling, we get:
\begin{align}
    \mathbb{E}\left[\mathbb{E}_\mathbf{S}\Big[f({\boldsymbol{x}}^{(r+1)})-f({\boldsymbol{x}}^{(r)})\Big]\right]&\leq -\gamma\mathbb{E}\left[\mathbb{E}_\mathbf{S}\left[ \big\langle\nabla f({\boldsymbol{x}}^{(r)}),\tilde{\mathbf{g}}_\mathbf{S}^{(r)}\big\rangle\right]\right]+\frac{\gamma^2 L}{2}\mathbb{E}\left[\mathbb{E}_\mathbf{S}\|\tilde{\mathbf{g}}_\mathbf{S}^{(r)}\|^2\right]\nonumber\\
    &=-\gamma\mathbb{E}\left[\mathbb{E}_\mathbf{S}\left[ \big\langle\nabla f({\boldsymbol{x}}^{(r)}),\tilde{\mathbf{g}}^{(r)}\big\rangle\right]\right]+\gamma\mathbb{E}\left[\mathbb{E}_\mathbf{S}\left[ \big\langle\nabla f({\boldsymbol{x}}^{(r)}),\tilde{\mathbf{g}}^{(r)}-\tilde{\mathbf{g}}_{\mathbf{S}}^{(r)}\big\rangle\right]\right]\nonumber\\
    &\qquad+\frac{\gamma^2 L}{2}\mathbb{E}\left[\mathbb{E}_\mathbf{S}\|\tilde{\mathbf{g}}_\mathbf{S}^{(r)}-\tilde{\mathbf{g}}^{(r)}+\tilde{\mathbf{g}}^{(r)}\|^2\right] \nonumber\\
    &\stackrel{(a)}{=}-\gamma\mathbb{E}\left[\mathbb{E}_\mathbf{S}\left[ \big\langle\nabla f({\boldsymbol{x}}^{(r)}),\tilde{\mathbf{g}}^{(r)}\big\rangle\right]\right]+\gamma\left[\mathbb{E}_\mathbf{S}\left[ \big\langle\nabla f({\boldsymbol{x}}^{(r)}),{\mathbf{g}}^{(r)}-{\mathbf{g}}_{\mathbf{S}}^{(r)}\big\rangle\right]\right]\nonumber\\
    &\qquad+\frac{\gamma^2 L}{2}\mathbb{E}\left[\mathbb{E}_\mathbf{S}\|\tilde{\mathbf{g}}_\mathbf{S}^{(r)}-\tilde{\mathbf{g}}^{(r)}+\tilde{\mathbf{g}}^{(r)}\|^2\right]\nonumber\\
    &\stackrel{(b)}{\leq}-\gamma\mathbb{E}\left[\mathbb{E}_\mathbf{S}\left[ \big\langle\nabla f({\boldsymbol{x}}^{(r)}),\tilde{\mathbf{g}}^{(r)}\big\rangle\right]\right]+\frac{\gamma}{2}\left[ \frac{1}{mL}\left\|\nabla f({\boldsymbol{x}}^{(r)})\right\|^2_2+mL\mathbb{E}_\mathbf{S}\left[\left\|{\mathbf{g}}^{(r)}-{\mathbf{g}}_{\mathbf{S}}^{(r)}\right\|^2_2\right]\right]\nonumber\\
    &\qquad+{\gamma^2 L}\mathbb{E}\left[\mathbb{E}_\mathbf{S}\left\|\tilde{\mathbf{g}}_\mathbf{S}^{(r)}-\tilde{\mathbf{g}}^{(r)}\right\|+\left\|\tilde{\mathbf{g}}^{(r)}\right\|^2\right] \nonumber\\
    &\stackrel{(c)}{\leq}-\gamma\mathbb{E}\left[ \big\langle\nabla f({\boldsymbol{x}}^{(r)}),\tilde{\mathbf{g}}^{(r)}\big\rangle\right]+\frac{\gamma}{2}\left[ \frac{1}{mL}\left\|\nabla f({\boldsymbol{x}}^{(r)})\right\|^2_2+mL\left(1-\frac{k}{d}\right)\left\|{\mathbf{g}}^{(r)}\right\|^2_2\right]\nonumber\\
    &\qquad+{\gamma^2 L}\mathbb{E}\left[\left(1-\frac{k}{d}\right)\left\|\tilde{\mathbf{g}}^{(r)}\right\|_2^2+\left\|\tilde{\mathbf{g}}^{(r)}\right\|_2^2\right]\nonumber\\
    &\stackrel{(d)}{=}-\gamma\underbrace{\mathbb{E}\left[ \big\langle\nabla f({\boldsymbol{x}}^{(r)}),\tilde{\mathbf{g}}^{(r)}\big\rangle\right]}_{(\mathrm{I})}+ \frac{\gamma}{2mL}\left\|\nabla f({\boldsymbol{x}}^{(r)})\right\|^2_2+\frac{mL\gamma}{2}\left(1-\frac{k}{d}\right)\underbrace{\left\|{\mathbf{g}}^{(r)}\right\|^2_2}_{(\mathrm{II})}\nonumber\\
    &\qquad+{\gamma^2 L}\left(2-\frac{k}{d}\right)\underbrace{\mathbb{E}\left[\left\|\tilde{\mathbf{g}}^{(r)}\right\|_2^2\right]}_{(\mathrm{III})}\label{eq:Lipschitz-c-gd-alt}
\end{align}
To bound term ($\mathrm{I}$) in Eq.~(\ref{eq:Lipschitz-c-gd-alt}) we use the combination of Lemmas~\ref{} and \ref{} we obtain:
\begin{align}
    -\gamma\mathbb{E}\left[ \big\langle\nabla f({\boldsymbol{x}}^{(r)}),\tilde{\mathbf{g}}^{(r)}\big\rangle\right]\leq \frac{\gamma}{2}\eta\frac{1}{p}\sum_{j=1}^p\sum_{c=0}^{\tau-1}\left[-\left\|\nabla f({\boldsymbol{x}}^{(r)})\right\|_2^2-\left\|\mathbf{g}_j^{(\ell,r)}\right\|_2^2+L^2\eta^2\sum_{\ell=0}^{\tau-1}\left[\tau\left\|{\mathbf{g}}_j^{(\ell,r)}\right\|_2^2+\sigma^2\right]\right]
\end{align}
Term $(\mathrm{II})$ can be bounded simply as follows:
\begin{align}
    \left\|{\mathbf{g}}^{(r)}\right\|^2_2&=\left\|\frac{\eta}{p}\sum_{j=1}^{p}\left[\sum_{c=0}^{\tau-1}{\mathbf{g}}_j^{(c,r)}\right]\right\|^2_2\nonumber\\
    &\leq\frac{\tau\eta^2}{p}\sum_{j=1}^{p}\sum_{c=0}^{\tau-1}\left\|\mathbf{g}_j^{(c,r)}\right\|^2_2
\end{align}

Next we bound term $(\mathrm{III})$ using the following lemma:
\begin{lemma}
\begin{align}
    \mathbb{E}\left[\left\|\tilde{\mathbf{g}}^{(r)}\right\|_2^2\right]\leq \frac{\eta^2\tau}{p}\sum_{j=1}^{p}\sum_{c=0}^{\tau-1}\left\|\mathbf{g}_j^{(c,r)}\right\|^2_2+\frac{\eta^2\tau}{p}\sigma^2
\end{align}
\end{lemma}
\begin{proof}
\begin{align}
    \mathbb{E}\left[\left\|\tilde{\mathbf{g}}^{(r)}\right\|_2^2\right]&=\mathbb{E}\left[\left\|\tilde{\mathbf{g}}^{(r)}-\mathbb{E}\left[\tilde{\mathbf{g}}^{(r)}\right]\right\|_2^2\right]+\left\|\mathbb{E}\left[\tilde{\mathbf{g}}^{(r)}\right]\right\|^2_2\nonumber\\
    &= \mathbb{E}\left[\left\|\tilde{\mathbf{g}}^{(r)}-{\mathbf{g}}^{(r)}\right\|_2^2\right]+\left\|{\mathbf{g}}^{(r)}\right\|^2_2\nonumber\\
    &= \mathbb{E}\left[\left\|\frac{\eta}{p}\sum_{j=1}^{p}\left[\sum_{c=0}^{\tau-1}\tilde{\mathbf{g}}_j^{(c,r)}\right]-\frac{\eta}{p}\sum_{j=1}^{p}\left[\sum_{c=0}^{\tau-1}\mathbf{g}_j^{(c,r)}\right]\right\|_2^2\right]+\left\|\frac{\eta}{p}\sum_{j=1}^{p}\left[\sum_{c=0}^{\tau-1}\mathbf{g}_j^{(c,r)}\right]\right\|^2_2\nonumber\\
&=\frac{\eta^2}{p^2}\sum_{j=1}^{p}\sum_{c=0}^{\tau-1}\mathbb{E}\left[\left\|\tilde{\mathbf{g}}_j^{(c,r)}-\mathbf{g}_j^{(c,r)}\right\|_2^2\right]+\left\|\frac{\eta}{p}\sum_{j=1}^{p}\left[\sum_{c=0}^{\tau-1}\mathbf{g}_j^{(c,r)}\right]\right\|^2_2 \nonumber\\
&\leq \frac{\eta^2}{p^2}\sum_{j=1}^{p}\sum_{c=0}^{\tau-1}\mathbb{E}\left[\left\|\tilde{\mathbf{g}}_j^{(c,r)}-\mathbf{g}_j^{(c,r)}\right\|_2^2\right]+\frac{\eta^2\tau}{p}\sum_{j=1}^{p}\sum_{c=0}^{\tau-1}\left\|\mathbf{g}_j^{(c,r)}\right\|^2_2\nonumber\\
&\leq \frac{\eta^2}{p^2}\sum_{j=1}^{p}\sum_{c=0}^{\tau-1}\sigma^2+\frac{\eta^2\tau}{p}\sum_{j=1}^{p}\sum_{c=0}^{\tau-1}\left\|\mathbf{g}_j^{(c,r)}\right\|^2_2\nonumber\\
&=\frac{\eta^2\tau}{p}\sum_{j=1}^{p}\sum_{c=0}^{\tau-1}\left\|\mathbf{g}_j^{(c,r)}\right\|^2_2+\frac{\eta^2\tau}{p}\sigma^2
\end{align}
\end{proof}
Next, we put all the pieces together as follows:
\begin{align}
    \mathbb{E}\left[\mathbb{E}_\mathbf{S}\Big[f({\boldsymbol{x}}^{(r+1)})-f({\boldsymbol{x}}^{(r)})\Big]\right]&\leq \frac{\gamma}{2}\eta\frac{1}{p}\sum_{j=1}^p\sum_{\ell=0}^{\tau-1}\left[-\left\|\nabla f({\boldsymbol{x}}^{(r)})\right\|_2^2-\left\|\mathbf{g}_j^{(\ell,r)}\right\|_2^2+L^2\eta^2\sum_{\ell=0}^{\tau-1}\left[\tau\left\|{\mathbf{g}}_j^{(\ell,r)}\right\|_2^2+\sigma^2\right]\right]\nonumber\\
    &\quad+ \frac{\gamma}{2mL}\left\|\nabla f({\boldsymbol{x}}^{(r)})\right\|^2_2+\frac{mL\gamma}{2}\left(1-\frac{k}{d}\right)\frac{\tau\eta^2}{p}\sum_{j=1}^{p}\sum_{\ell=0}^{\tau-1}\left\|\mathbf{g}_j^{(\ell,r)}\right\|^2_2\nonumber\\
    &\quad+\gamma^2 L\left(2-\frac{k}{d}\right)\left[\frac{\eta^2\tau}{p}\sum_{j=1}^{p}\sum_{c=0}^{\tau-1}\left\|\mathbf{g}_j^{(\ell,r)}\right\|^2_2+\frac{\eta^2\tau}{p}\sigma^2\right]\nonumber\\
    &=-\frac{\tau\eta\gamma}{2}\left\|\nabla f({\boldsymbol{x}}^{(r)})\right\|_2^2+\frac{\gamma}{2}\eta\frac{1}{p}\sum_{j=1}^p\sum_{\ell=0}^{\tau-1}\left[-\left\|\mathbf{g}_j^{(\ell,r)}\right\|_2^2+L^2\eta^2\tau^2\left\|{\mathbf{g}}_j^{(\ell,r)}\right\|_2^2\right]+\frac{\gamma\eta^3L^2\tau^2}{2}\sigma^2\nonumber\\
    &\quad+ \frac{\gamma}{2mL}\left\|\nabla f({\boldsymbol{x}}^{(r)})\right\|^2_2+\frac{mL\gamma}{2}\left(1-\frac{k}{d}\right)\frac{\tau\eta^2}{p}\sum_{j=1}^{p}\sum_{\ell=0}^{\tau-1}\left\|\mathbf{g}_j^{(\ell,r)}\right\|^2_2\nonumber\\
    &\quad+\gamma^2 L\left(2-\frac{k}{d}\right)\frac{\eta^2\tau}{p}\sum_{j=1}^{p}\sum_{\ell=0}^{\tau-1}\left\|\mathbf{g}_j^{(\ell,r)}\right\|^2_2+\gamma^2 L\left(2-\frac{k}{d}\right)\frac{\eta^2\tau}{p}\sigma^2\nonumber\\
    &=-\left(\frac{\tau\eta\gamma}{2}-\frac{\gamma}{2mL}\right)\left\|\nabla f({\boldsymbol{x}}^{(r)})\right\|_2^2\nonumber\\
    &\quad-\left(\frac{\eta\gamma}{2}-\frac{\eta\gamma}{2}\left(L^2\eta^2\tau^2\right)-\frac{mL\eta\gamma}{2}\left(1-\frac{k}{d}\right)\tau\eta-\gamma^2 L\eta^2\tau\left(2-\frac{k}{d}\right)\right)\frac{1}{p}\sum_{j=1}^{p}\sum_{\ell=0}^{\tau-1}\left\|\mathbf{g}_j^{(\ell,r)}\right\|^2_2\nonumber\\
    &\quad+\frac{\gamma\eta^3L^2\tau^2}{2}\sigma^2+\gamma^2 L\left(2-\frac{k}{d}\right)\frac{\eta^2\tau}{p}\sigma^2\nonumber\\
    &\stackrel{(a)}{\leq}-\left(\frac{\tau\eta\gamma}{2}-\frac{\gamma}{2mL}\right)\left\|\nabla f({\boldsymbol{x}}^{(r)})\right\|_2^2+\frac{\gamma\eta^3L^2\tau^2}{2}\sigma^2+\tau\eta^2\gamma^2 L\left(2-\frac{k}{d}\right)\frac{\sigma^2}{p}\label{eq:ncvx-mid-step}
\end{align}
where (a) follows from the learning rate choices of 
\begin{align}
    \frac{\eta\gamma}{2}-\frac{\eta\gamma}{2}\left(L^2\eta^2\tau^2\right)-\frac{mL\eta\gamma}{2}\left(1-\frac{k}{d}\right)\tau\eta-\gamma^2 L\eta^2\tau\left(2-\frac{k}{d}\right)\geq 0
\end{align}
which can be simplified further as follows:
\begin{align}
    1-L^2\eta^2\tau^2-mL\tau\eta\left(1-\frac{k}{d}\right)-2\gamma L\eta\tau\left(2-\frac{k}{d}\right)\geq 0
\end{align}
Then using Eq.~(\ref{eq:ncvx-mid-step}) we obtain:
\begin{align}
  \frac{\tau\gamma}{2} \left({\eta}-\frac{1}{\tau mL}\right)\left\|\nabla f({\boldsymbol{x}}^{(r)})\right\|_2^2\leq \mathbb{E}\left[\mathbb{E}_\mathbf{S}\Big[f({\boldsymbol{x}}^{(r+1)})-f({\boldsymbol{x}}^{(r)})\Big]\right]+\tau\eta^2\gamma^2 L\left(2-\frac{k}{d}\right)\frac{\sigma^2}{p}+\frac{\gamma\eta^3L^2\tau^2}{2}\sigma^2
\end{align}
which leads to the following bound:
\begin{align}
     \left\|\nabla f({\boldsymbol{x}}^{(r)})\right\|_2^2\leq \frac{2 \mathbb{E}\left[\mathbb{E}_\mathbf{S}\Big[f({\boldsymbol{x}}^{(r+1)})-f({\boldsymbol{x}}^{(r)})\Big]\right]}{\tau \gamma \left({\eta}-\frac{1}{\tau mL}\right)}+\frac{2\eta^2\gamma L\left(2-\frac{k}{d}\right)\frac{\sigma^2}{p}}{ \left({\eta}-\frac{1}{\tau mL}\right)}+\frac{\eta^3L^2\tau}{\left({\eta}-\frac{1}{\tau mL}\right)}\sigma^2 
\end{align}
Now averaging over $r$ communication rounds we achieve:
\begin{align}
    \frac{1}{R}\sum_{r=0}^{R-1}\left\|\nabla f({\boldsymbol{x}}^{(r)})\right\|_2^2\leq \frac{2 \mathbb{E}\left[\mathbb{E}_\mathbf{S}\Big[f({\boldsymbol{x}}^{(0)})-f({\boldsymbol{x}}^{(*)})\Big]\right]}{R\tau \gamma \left({\eta}-\frac{1}{\tau mL}\right)}+\frac{2\eta^2\gamma L\left(2-\frac{k}{d}\right)\frac{\sigma^2}{p}}{ \left({\eta}-\frac{1}{\tau mL}\right)}+\frac{\eta^3L^2\tau}{\left({\eta}-\frac{1}{\tau mL}\right)}\sigma^2 
\end{align}
We note that for this case we have the following conditions over learning rate:
\begin{align}
    L^2\eta^2\tau^2+mL\tau\eta\left(1-\frac{k}{d}\right)+2\gamma L\eta\tau\left(2-\frac{k}{d}\right)\leq 1,\:\eta> \frac{1}{mL\tau},
\end{align}

\subsection{Proof of Theorem~\ref{thm:pl-iid}}
From Eq.~(\ref{eq:ncvx-mid-step}) under condition with:
\begin{align}
       L^2\eta^2\tau^2+mL\tau\eta\left(1-\frac{k}{d}\right)+2\gamma L\eta\tau\left(2-\frac{k}{d}\right)\leq 1, \label{eq:step_size_cnd_mmr}
\end{align}
we obtain:
\begin{align}
         \mathbb{E}\left[f({\boldsymbol{w}}^{(r+1)})-f({\boldsymbol{w}}^{(r)})\right]&\leq -\left(\frac{\tau\eta\gamma}{2}-\frac{\gamma}{2mL}\right)\left\|\nabla f({\boldsymbol{x}}^{(r)})\right\|_2^2+\frac{\gamma\eta^3L^2\tau^2}{2}\sigma^2+\tau\eta^2\gamma^2 L\left(2-\frac{k}{d}\right)\frac{\sigma^2}{p}\nonumber\\
         &\stackrel{(PL)}{\leq} -\left({\tau\mu\eta\gamma}-\frac{\mu\gamma}{mL}\right)\left[f({\boldsymbol{w}}^{(r)})-f({\boldsymbol{w}}^{(*)})\right]+\frac{\gamma\eta^3L^2\tau^2}{2}\sigma^2+\tau\eta^2\gamma^2 L\left(2-\frac{k}{d}\right)\frac{\sigma^2}{p} 
\end{align}
which leads to the following bound:
\begin{align}
            \mathbb{E}\Big[f({\boldsymbol{w}}^{(r+1)})-f({\boldsymbol{w}}^{(*)})\Big]&\leq \left(1-\eta\mu\gamma{\tau}+\frac{\mu\gamma}{mL}\right) \Big[f({\boldsymbol{w}}^{(r)})-f({\boldsymbol{w}}^{(*)})\Big]+\frac{\gamma\eta^3L^2\tau^2}{2}\sigma^2+\tau\eta^2\gamma^2 L\left(2-\frac{k}{d}\right)\frac{\sigma^2}{p} 
\end{align}
which leads to the following bound by setting $\Delta\triangleq1-\eta\mu\gamma{\tau}+\frac{\mu\gamma}{mL}=1-\mu\gamma\tau\left(\eta-\frac{1}{mL\tau}\right)$:
\begin{align}
            \mathbb{E}\Big[f({\boldsymbol{w}}^{(R)})-f({\boldsymbol{w}}^{(*)})\Big]&\leq \Delta^R \Big[f({\boldsymbol{w}}^{(0)})-f({\boldsymbol{w}}^{(*)})\Big]+\frac{1-\Delta^R}{1-\Delta}\left(\frac{\gamma\eta^3L^2\tau^2}{2}\sigma^2+\tau\eta^2\gamma^2 L\left(2-\frac{k}{d}\right)\frac{\sigma^2}{p} \right)\nonumber\\
            &\leq \Delta^R \Big[f({\boldsymbol{w}}^{(0)})-f({\boldsymbol{w}}^{(*)})\Big]+\frac{1}{1-\Delta}\left(\frac{\gamma\eta^3L^2\tau^2}{2}\sigma^2+\tau\eta^2\gamma^2 L\left(2-\frac{k}{d}\right)\frac{\sigma^2}{p} \right)\nonumber\\
            &={\left(1-\mu\gamma\tau\left(\eta-\frac{1}{mL\tau}\right)\right)}^R \Big[f({\boldsymbol{w}}^{(0)})-f({\boldsymbol{w}}^{(*)})\Big]+\frac{\left(\frac{\gamma\eta^3L^2\tau^2}{2}\sigma^2+\tau\eta^2\gamma^2 L\left(2-\frac{k}{d}\right)\frac{\sigma^2}{p} \right)}{\mu\gamma\tau\left(\eta-\frac{1}{m L\tau}\right)}\nonumber\\
            &\leq \exp{-\left(\mu\gamma\tau\left(\eta-\frac{1}{m L\tau}\right)R\right)}\Big[f({\boldsymbol{w}}^{(0)})-f({\boldsymbol{w}}^{(*)})\Big]+\frac{\left(\frac{\gamma\eta^3L^2\tau}{2}\sigma^2+\eta^2\gamma^2 L\left(2-\frac{k}{d}\right)\frac{\sigma^2}{p} \right)}{\mu\gamma\left(\eta-\frac{1}{mL\tau}\right)}
\end{align}
Then for the choice of $\eta=\frac{n}{mL\tau}$, for $m>n>1$, we obtain:


\begin{align}
                \mathbb{E}\Big[f({\boldsymbol{w}}^{(R)})-f({\boldsymbol{w}}^{(*)})\Big]&\leq \exp{-\left(\frac{\gamma\left(n-1\right) R}{m\kappa}\right) }\Big[f({\boldsymbol{w}}^{(0)})-f({\boldsymbol{w}}^{(*)})\Big]+\frac{\left(\frac{\gamma n^3L^2\tau}{2m^3L^3\tau^3}\sigma^2+\frac{n^2}{m^2L^2\tau^2}\gamma^2 L\left(2-\frac{k}{d}\right)\frac{\sigma^2}{p} \right)}{\mu\gamma\left(\frac{n-1}{mL\tau}\right)}\nonumber\\
                &=\exp{-\left(\frac{\gamma\left(n-1\right) R}{m\kappa}\right) }\Big[f({\boldsymbol{w}}^{(0)})-f({\boldsymbol{w}}^{(*)})\Big]+\frac{\left(\frac{ n^3}{2m^2}+\frac{n^2}{m}\gamma L\left(2-\frac{k}{d}\right)\frac{1}{p} \right)}{\mu\tau\left(n-1\right)}\sigma^2
\end{align}

We note that regarding condition in Eq.~(\ref{eq:step_size_cnd_mmr}), if we let $\eta=\frac{n}{m L\tau}$ for $m>n>1$, we need to satisfy the following condition:
\begin{align}
    \frac{n^2}{m^2}+n\left(1-\frac{k}{d}\right)+\frac{2n\gamma\left(1-\frac{k}{d}\right)}{m}\leq 1
\end{align}
Now if you let $\gamma=\frac{m}{2}$, we need to impose the following condition over $k$ and $d$ as follows:
\begin{align}
    n\left(1-\frac{k}{d}\right)\leq \frac{1}{3}\implies d\left(1-\frac{1}{3n}\right)\leq k\leq d
\end{align}
\todo{Will fix these later!}





\end{document}
