\documentclass{article}

\usepackage[margin=1in]{geometry}
\usepackage[colorlinks,linkcolor=blue,filecolor=blue,citecolor=magenta,urlcolor=blue]{hyperref}
\usepackage{bm,amsmath,amsthm,amssymb,multicol,algorithmic,algorithm,enumitem,graphicx,subfigure}
\usepackage{xargs}
\usepackage{stmaryrd}


\def\M{\mathcal{M}}
\def\A{\mathcal{A}}
\def\Z{\mathcal{Z}}
\def\S{\mathcal{S}}
\def\D{\mathcal{D}}
\def\R{\mathcal{R}}
\def\P{\mathcal{P}}
\def\K{\mathcal{K}}
\def\E{\mathbb{E}}
\def\F{\mathfrak{F}}
\def\l{\boldsymbol{\ell}}

\newtheorem{Fact}{Fact}
\newtheorem{Lemma}{Lemma}
\newtheorem{Prop}{Proposition}
\newtheorem{Theorem}{Theorem} 
\newtheorem{Def}{Definition}
\newtheorem{Corollary}{Corollary}
\newtheorem{Conjecture}{Conjecture}
\newtheorem{Property}{Property}
\newtheorem{Observation}{Observation}
\newtheorem{Exa}{Example}
\newtheorem{assumption}{H\!\!}
\newtheorem{Remark}{Remark}
\newtheorem*{Lemma*}{Lemma}
\newtheorem*{Theorem*}{Theorem}
\newtheorem*{Corollary*}{Corollary}
 
\newcommand{\eqsp}{\;}
\newcommand{\beq}{\begin{equation}}
\newcommand{\eeq}{\end{equation}}
\newcommand{\eqdef}{\mathrel{\mathop:}=}
\def\EE{\mathbb{E}}
\newcommand{\norm}[1]{\left\Vert #1 \right\Vert}
\newcommand{\pscal}[2]{\left\langle#1\,|\,#2 \right\rangle}
\def\major{\mathsf{M}}
\def\rset{\ensuremath{\mathbb{R}}}


\begin{document}



\title{AniLA: Anisotropic Langevin Dynamics for training Energy-Based Models}

 \author{\textbf{Belhal Karimi, Jianwen Xie, Ping Li} \\\\
 Cognitive Computing Lab\\
 Baidu Research\\
   10900 NE 8th St. Bellevue, WA 98004, USA
 }

\date{}
\maketitle

\begin{abstract}
\end{abstract}

\section{Introduction}

\section{MCMC based EBM}

\paragraph{Energy Based Models}

\paragraph{MCMC procedures}

\paragraph{Focus on Langevin Diffusion}

\section{ANILA sampler based EBM}

\subsection{Curvature informed MCMC}

We introduce a new sampler based on the Langevin updates presented above.

\begin{algorithm}[H]
\caption{\textsc{AniLA EBM}} \label{alg:ldams}
\begin{algorithmic}[1]
%\small
\STATE \textbf{Input}: parameter s
\FOR{$t=1$ to $T$}
\STATE Set $\theta_{r,i}^{0} = \bar{\theta}_{r-1}$
\ENDFOR
\STATE ds
\end{algorithmic}
\end{algorithm}



\subsection{Geometric ergodicity of ANILA sampler}

\section{Numerical Experiments}


\section{Conclusion}

\newpage

\bibliographystyle{abbrvnat}
\bibliography{ref}



%-----------------------------------------------------------------------------
%\vspace{0.4cm}

\end{document} 