\documentclass{article}

\usepackage[margin=1in]{geometry}
\usepackage[colorlinks,linkcolor=blue,filecolor=blue,citecolor=magenta,urlcolor=blue]{hyperref}
\usepackage{bm,amsmath,amsthm,amssymb,multicol,algorithmic,algorithm,enumitem,graphicx,subfigure}
\usepackage{xargs}
\usepackage{stmaryrd}


\def\M{\mathcal{M}}
\def\A{\mathcal{A}}
\def\Z{\mathcal{Z}}
\def\S{\mathcal{S}}
\def\D{\mathcal{D}}
\def\R{\mathcal{R}}
\def\P{\mathcal{P}}
\def\K{\mathcal{K}}
\def\E{\mathbb{E}}
\def\F{\mathfrak{F}}
\def\l{\boldsymbol{\ell}}

\newtheorem{Fact}{Fact}
\newtheorem{Lemma}{Lemma}
\newtheorem{Prop}{Proposition}
\newtheorem{Theorem}{Theorem} 
\newtheorem{Def}{Definition}
\newtheorem{Corollary}{Corollary}
\newtheorem{Conjecture}{Conjecture}
\newtheorem{Property}{Property}
\newtheorem{Observation}{Observation}
\newtheorem{Exa}{Example}
\newtheorem{assumption}{H\!\!}
\newtheorem{Remark}{Remark}
\newtheorem*{Lemma*}{Lemma}
\newtheorem*{Theorem*}{Theorem}
\newtheorem*{Corollary*}{Corollary}
 
\newcommand{\eqsp}{\;}
\newcommand{\beq}{\begin{equation}}
\newcommand{\eeq}{\end{equation}}
\newcommand{\eqdef}{\mathrel{\mathop:}=}
\def\EE{\mathbb{E}}
\newcommand{\norm}[1]{\left\Vert #1 \right\Vert}
\newcommand{\pscal}[2]{\left\langle#1\,|\,#2 \right\rangle}
\def\major{\mathsf{M}}
\def\rset{\ensuremath{\mathbb{R}}}


\begin{document}



\title{AniLA: Anisotropic Langevin Dynamics for training Energy-Based Models}

 \author{\textbf{Belhal Karimi, Jianwen Xie, Ping Li} \\\\
 Cognitive Computing Lab\\
 Baidu Research\\
   10900 NE 8th St. Bellevue, WA 98004, USA
 }

\date{}
\maketitle

\begin{abstract}
\end{abstract}

\section{Introduction}

Given a stream of input noted $x$, the energy-based model (EBM) is a Gibbs distribution defined as:
\beq\label{eq:ebm}
p_{\theta}(x) = \frac{1}{Z(\theta)} \mathrm{exp}(f_{\theta}(x))
\eeq


\section{MCMC based EBM}

\paragraph{Energy Based Models: }
Energy based models are a class of generative models that leverages the power of Gibbs potential and high dimensional sampling techniques to produce high quality synthetic image samples.

\paragraph{MCMC procedures: }

\paragraph{Focus on Langevin Diffusion: }

\section{ANILA sampler based EBM}

\subsection{Curvature informed MCMC}

We introduce a new sampler based on the Langevin updates presented above.

\begin{algorithm}[H]
\caption{\textsc{AniLA for Energy-Based Model}} \label{alg:ldams}
\begin{algorithmic}[1]
%\small
\STATE \textbf{Input}: Total number of iterations $T$, number of MCMC transitions $K$ and of samples $M$ learning rate $\eta$, initial values $\theta_0$, $\{ z_{0}^m \}_{m=1}^M$ and $n$ observations $\{ x_{i} \}_{i=1}^n$.
\FOR{$t=1$ to $T$}
\STATE Compute the anisotropic stepsize as follows: \label{line:step}
\beq
\stepsize_t = \frac{b}{\max(b, | \nabla f_{\theta_t}(x) |}
\eeq
\STATE Draw $m$ samples $\{ z_{t}^m \}_{m=1}^M$ from the objective potential \eqref{eq:ebm} via Langevin diffusion:\label{line:langevin}
\beq
z_{t}^m = z_{t}^m + \stepsize_t/2  \nabla f_{\theta_t}(x) + \sqrt{\stepsize} \mathsf{B}_t
\eeq
where $\mathsf{B}_t$ is the brownian motion, drawn from a Normal distribution.
\STATE Samples $m$ positive observations $\{ x_{i} \}_{i=1}^m$ from the empirical data distribution
\STATE Compute the gradient of the empirical log-EBM \eqref{eq:ebm} as follows:
\beq
\nabla \sum_{i=1}^m \log p_{\theta_t}(x_i) = \mathbb{E}_{p_{\text {data }}}\left[\nabla_{\theta} f_{\theta_t}(x)\right]-\mathbb{E}_{p_{\theta}}\left[\nabla_{\theta_t} f_{\theta}(z_t^m)\right] \approx \frac{1}{m} \sum_{i=1}^{m} \nabla_{\theta} f_{\theta_t}\left(x_{i}\right)-\frac{1}{m} \sum_{i=1}^{m} \nabla_{\theta} f_{\theta_t}\left(z_t^m\right)
\eeq
\STATE Update the vector of global parameters of the EBM:
\beq
\theta_{t+1} = \theta_{t+1} + \eta \nabla \sum_{i=1}^m \log p_{\theta_t}(x_i)
\eeq
\ENDFOR
\STATE \textbf{Output:} Generated samples $\{ z_{T}^m \}_{m=1}^M$
\end{algorithmic}
\end{algorithm}



\subsection{Geometric ergodicity of ANILA sampler}
We will present in this subsection, a convergence result for the Markov Chain constructed using Line~\ref{line:step}-\ref{line:langevin}. 


\section{Numerical Experiments}


\section{Conclusion}

\newpage

\bibliographystyle{abbrvnat}
\bibliography{ref}



%-----------------------------------------------------------------------------
%\vspace{0.4cm}

\end{document} 