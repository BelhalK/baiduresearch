\documentclass{article}

\usepackage[margin=1in]{geometry}
\usepackage[colorlinks,linkcolor=blue,filecolor=blue,citecolor=magenta,urlcolor=blue]{hyperref}
\usepackage{bm,amsmath,amsthm,amssymb,multicol,algorithmic,algorithm,enumitem,graphicx,subfigure}
\usepackage{xargs}
\usepackage{stmaryrd}
\usepackage{natbib}
\usepackage{listings}
\usepackage{xcolor}
\usepackage{booktabs} % for professional tables
\definecolor{codegreen}{rgb}{0,0.6,0}
\definecolor{codegray}{rgb}{0.5,0.5,0.5}
\definecolor{codepurple}{rgb}{0.58,0,0.82}
\definecolor{backcolour}{rgb}{0.95,0.95,0.92}

\newtheorem{definition}{Definition}
\newcommand{\algo}{\textsc{eff-EBM}}
\lstdefinestyle{mystyle}{
    backgroundcolor=\color{backcolour},   
    commentstyle=\color{codegreen},
    keywordstyle=\color{magenta},
    numberstyle=\tiny\color{codegray},
    stringstyle=\color{codepurple},
    basicstyle=\ttfamily\footnotesize,
    breakatwhitespace=false,         
    breaklines=true,                 
    captionpos=b,                    
    keepspaces=true,                 
    numbers=left,                    
    numbersep=5pt,                  
    showspaces=false,                
    showstringspaces=false,
    showtabs=false,                  
    tabsize=2
}


\def\M{\mathcal{M}}
\def\A{\mathcal{A}}
\def\Z{\mathcal{Z}}
\def\S{\mathcal{S}}
\def\D{\mathcal{D}}
\def\R{\mathcal{R}}
\def\P{\mathcal{P}}
\def\K{\mathcal{K}}
\def\E{\mathbb{E}}
\def\F{\mathfrak{F}}
\def\l{\boldsymbol{\ell}}

\newtheorem{Fact}{Fact}
\newtheorem{Lemma}{Lemma}
\newtheorem{Prop}{Proposition}
\newtheorem{Theorem}{Theorem} 
\newtheorem{Def}{Definition}
\newtheorem{Corollary}{Corollary}
\newtheorem{Conjecture}{Conjecture}
\newtheorem{Property}{Property}
\newtheorem{Observation}{Observation}
\newtheorem{Exa}{Example}
\newtheorem{assumption}{H\!\!}
\newtheorem{Remark}{Remark}
\newtheorem*{Lemma*}{Lemma}
\newtheorem*{Theorem*}{Theorem}
\newtheorem*{Corollary*}{Corollary}
 
\newcommand{\eqsp}{\;}
\newcommand{\beq}{\begin{equation}}
\newcommand{\eeq}{\end{equation}}
\newcommand{\eqdef}{\mathrel{\mathop:}=}
\def\EE{\mathbb{E}}
\newcommand{\norm}[1]{\left\Vert #1 \right\Vert}
\newcommand{\pscal}[2]{\left\langle#1\,|\,#2 \right\rangle}
\def\major{\mathsf{M}}
\def\rset{\ensuremath{\mathbb{R}}}





\begin{document}



\title{Weekly Report KARIMI-2021-12-10}


\date{}
\maketitle




My work this week has mainly been towards
\begin{itemize}
\item IJCAI22 Papers
\begin{enumerate}
	\item STANLEY experiments: Improve baseline
	\item Continue on the Dist-EBM paper
\end{enumerate}
\end{itemize}

\section{STANLEY experiments: Improve baseline}

I have been focussing on the STANLEY paper again this week.
While I have greatly improved the experiments section with CIFAR, Flowers, Gaussian and CelebA datasets for the AAAI deadline, it did not pay off.
Hence, I am working towards improving the celebA inpatinting experiments.
This consists in 

\begin{itemize}
\item Improving the baseline, Vanilla Langevin in the image completion task, i.e. making sure that the main baseline is producing the best quality of missing images. Right now, it appears to be slightly blurry. Main focus is to try as much hyperparameters as I can. No convincing results for now as they are as good as I already have.
\item Adding more baselines as far as visual checks. While I do have GD and HMC baselines for the FID curves, I have not been able to produce convincing images for the visual checks. 
In general FID checks should prevail over the visual ones as they are quantified at each iteration (or thousands) but having more images to show seems to be what reviewers want.
\end{itemize}



\section{Continue on the Dist-EBM paper}
This week, and after all the deadlines and rebuttals of last week, I also took over the Compressed and Distributed EBM project.
Did not do much progress on this so far (except having written the main algorithm which in my opinion is correct).

Next steps to do:
\begin{itemize}
\item Implement the compression in the EBM (last step of the algorithm, i.e. the Gradient Step only.
\item Derive some convergence guarantees: this will be important as adding parallelization is commonly done by practicioners but never understood, let alone adding a compression step.
\end{itemize}


\bibliographystyle{plain}
\bibliography{ref}


\end{document} 