\documentclass{article}

\usepackage[margin=1in]{geometry}
\usepackage[colorlinks,linkcolor=blue,filecolor=blue,citecolor=magenta,urlcolor=blue]{hyperref}
\usepackage{bm,amsmath,amsthm,amssymb,multicol,algorithmic,algorithm,enumitem,graphicx,subfigure}
\usepackage{xargs}
\usepackage{stmaryrd}
\usepackage{natbib}

\def\M{\mathcal{M}}
\def\A{\mathcal{A}}
\def\Z{\mathcal{Z}}
\def\S{\mathcal{S}}
\def\D{\mathcal{D}}
\def\R{\mathcal{R}}
\def\P{\mathcal{P}}
\def\K{\mathcal{K}}
\def\E{\mathbb{E}}
\def\F{\mathfrak{F}}
\def\l{\boldsymbol{\ell}}

\newtheorem{Fact}{Fact}
\newtheorem{Lemma}{Lemma}
\newtheorem{Prop}{Proposition}
\newtheorem{Theorem}{Theorem} 
\newtheorem{Def}{Definition}
\newtheorem{Corollary}{Corollary}
\newtheorem{Conjecture}{Conjecture}
\newtheorem{Property}{Property}
\newtheorem{Observation}{Observation}
\newtheorem{Exa}{Example}
\newtheorem{assumption}{H\!\!}
\newtheorem{Remark}{Remark}
\newtheorem*{Lemma*}{Lemma}
\newtheorem*{Theorem*}{Theorem}
\newtheorem*{Corollary*}{Corollary}
 
\newcommand{\eqsp}{\;}
\newcommand{\beq}{\begin{equation}}
\newcommand{\eeq}{\end{equation}}
\newcommand{\eqdef}{\mathrel{\mathop:}=}
\def\EE{\mathbb{E}}
\newcommand{\norm}[1]{\left\Vert #1 \right\Vert}
\newcommand{\pscal}[2]{\left\langle#1\,|\,#2 \right\rangle}
\def\major{\mathsf{M}}
\def\rset{\ensuremath{\mathbb{R}}}
\newcommand{\inter}{\llbracket n \rrbracket}
\newcommand{\interl}{\llbracket L \rrbracket}




\begin{document}



\title{Weekly Report KARIMI 2021-07-16}


\date{}
\maketitle

This week, I have mainly focussed my work towards developing a Federated EM algorithm.
Two settings are possible: 
\begin{itemize}
\item The expectations are tractable and we want to scale to large datasets with a random data index sampling while being distributed and private (this would be a sEM method, where sEM stands for Stochastic EM).
\item The expectations are not tractable and thus we would use the SAEM under the FL settings (this is the setting of the my talk from last week).
\end{itemize}

\section{SAEM for Federated Learning}

For computational purposes and privacy enhanced matter, I have chosen to study and develop the second algorithms that I proposed in my last week's report.
In that algorithm, one does not compute a periodic averaging of the local models (this would requires performing as many M-steps as there are workers).
Rather, workers compute local statistics and send them to the central server for a periodic averaging of those vectors and the latter computes one M-step to update the global model.

\begin{algorithm}[H]
\caption{FL-SAEM with statistics averaging} \label{alg:flsaem2}
\begin{algorithmic}[1]
%\small
\STATE \textbf{Input}: \textcolor{red}{TO COMPLETE}
\STATE Init: $\theta_{0} \in \Theta \subseteq \mathbb R^d $, as the global model and $\bar{\theta}_0 =  \frac{1}{n} \sum_{i=1}^n \theta_0$.
\FOR{$r=1$ to $R$}
\FOR{parallel for device $i \in D^{r}$}
\STATE Set $\hat{\theta}^{(0,r)}_i = \hat{\theta}^{(r)}$.
\STATE Draw M samples $z_{i,m}^{(r)}$ under model $\hat{\theta}^{(r)}_i$ \label{line:sampling}
\STATE Compute the surrogate sufficient statistics $\tilde{S}_{i}^{(r+1)}$ \label{line:compute}
\STATE Devices send local statistics $\tilde{S}_{i}^{(k+1)}$ to server.
\ENDFOR
\STATE Server computes \textbf{global model using the aggregated statistics}:
$$
\hat{\theta}^{(r+1)} = \overline{\theta}( \tilde{S}^{(r+1)}) 
$$
where $\tilde{S}^{(r+1)} = (\tilde{S}_i^{(r+1)}, i \in D_r)$  and send global model back to the devices. 
\ENDFOR
\end{algorithmic}
\end{algorithm}


\subsection{Challenges with Algorithm~\ref{alg:flsaem2}}
While Algorithm~\ref{alg:flsaem2} is a distributed variant of the SAEM, it is neither (a) \emph private nor (b) \emph communication-efficient.
Indeed, we remark that broadcasting the vector of statistics are a potential breach to the data observations as their expression is related $y$ and the latent data $z$. With a simple knowledge of the model used, the data could be retrieved if one extracts those statistics.
Also regarding (b), the broadcast of $n$ vector of statistics $S(y_i,z_i)$ can be cumbersome when the size of the latent space and the parameter space of the model are huge.

I am incorporating respective solutions to those problems below.

\subsection{Algorithmic solutions}

\textbf{Line~\ref{line:sampling} -- Quantization:} 
The first step is to quantize the gradient in the Stochastic Langevin Dynamics step used in our sampling scheme Line~\ref{line:sampling} of Algorithm~\ref{alg:flsaem2}.
Inspired by \citep{alistarh2017qsgd}, we use an extension of the QSGD algorithm for our latent samples.
Define the quantization operator as follows:

\beq\label{eq:operator}
\mathsf{C}_{j}^{(\ell)}\left(g, \xi_{j}\right)=\|v\| \cdot \textrm{sign}\left(g_{j}\right) \cdot\left(\left\lfloor \ell \left|g_{j}\right| /\|v\|\right\rfloor+\mathbf{1}\left\{\xi_{j} \leq \ell \left|g_{j}\right| /\|v\|-\left\lfloor \ell \left|g_{j}\right| /\|v\|\right\rfloor\right\}\right) /\ell
\eeq
where $\ell$ is the level of quantization and $j \in [d]$ denotes the dimension of the gradient.

Hence, for the sampling step, Line~\ref{line:sampling}, we use the modified SGLD below, to be compliant with the privacy of our method.
\begin{algorithm}[H]
\caption{Langevin Dynamics with Quantization for worker $i$} \label{alg:quant}
\begin{algorithmic}[1]
%\small
\STATE \textbf{Input}: Current local model $\hat{\theta}^{(r)}_i$ for worker $i \in \inter$.

\STATE Draw $M$ samples $\{ z_{i}^{(r,m} \}_{m=1}^M$ from the posterior distirbution $p(z_i| y_i; \hat{\theta}^{(k)}_i)$ via Langevin diffusion with a quantized gradient:\label{line:langevin}
\FOR{$k=1$ to $K$}
\STATE Compute the quantized gradient of $\nabla \log p(z_i| y_i; \hat{\theta}^{(k)}_i)$:
\beq\label{eq:grad}
g_i{(k,m)} = \mathsf{C}_{j}^{(\ell)}\left(\nabla_j f_{\theta_t}(z_i^{(k-1,m)}), \xi^{(k)}_{j}\right)
\eeq
where $\xi^{(k)}_{j}$ is a realization of a uniform random variable.
\STATE Construct the Markov Chain:
\beq\label{eq:lang}
z_i^{(k,m)} = z_i^{(k-1,m)} + \frac{\gamma_k}{2}  g_i{(k,m)} + \sqrt{\gamma_k}  \mathsf{B}_k \eqsp,
\eeq
where $\mathsf{B}_t$ denotes the Brownian motion.
\ENDFOR
\STATE Assign $\{ z_{i}^{(r,m} \}_{m=1}^M \leftarrow \{ z_i^{(K,m)} \}_{m=1}^M$.
\STATE \textbf{Output:} latent data $z_{i,m}^{(k)}$ under model $\hat{\theta}^{(t,k)}_i$ 
\end{algorithmic}
\end{algorithm}



\noindent \textbf{Line~\ref{line:compute} -- Compression MCMC output:}
We use the notorious \textbf{Top-$k$} operator that we define as $\mathcal C(x)_i=x_i$, if $i\in \mathcal S$; $\mathcal C(x)_i=0$ otherwise and where $\mathcal S$ is the set of size $k<p$.
Recall that after Line~\ref{line:sampling} we compute the local statistics $\tilde{S}_{i}^{(k+1)}$ using the output latent variables from Algorithm~\ref{alg:quant}.
We now use those statistics and compress them using Algorithm~\ref{alg:spars} as follows:

\begin{algorithm}[H]
\caption{Sparsified Statistics with \textbf{Top-$k$}} \label{alg:spars}
\begin{algorithmic}[1]
%\small
\STATE \textbf{Input}: Current local statistics $\tilde{S}_{i}^{(k+1)}$ for worker $i \in \inter$. Sparsification level $k$.
\STATE Apply \textbf{Top-$k$}:
\beq\label{eq:topkstats}
\ddot{S}_{i}^{(k+1)} = \mathcal C \left( \tilde{S}_{i}^{(k+1)}\right)
\eeq
\STATE \textbf{Output:} Compressed local statistics for worker $i$ denoted $\ddot{S}_{i}^{(k+1)}$.
\end{algorithmic}
\end{algorithm}


%
%Final method:
%
%\begin{algorithm}[H]
%\caption{FL-SAEM with Periodic Compressed and Quantized Statistics Averaging} \label{alg:flsaem}
%\begin{algorithmic}[1]
%%\small
%\STATE \textbf{Input}: .
%\STATE Init: $\theta_{0} \in \Theta \subseteq \mathbb R^d $, as the global model and $\bar{\theta}_0 =  \frac{1}{n} \sum_{i=1}^n \theta_0$.
%\FOR{$r=1$ to $R$}
%\FOR{parallel for device $i \in D^{r}$}
%\STATE Set $\hat{\theta}^{(0,k)}_i = \hat{\theta}^{(k)}$.
%\FOR{$t=1$ to $T$}
%\STATE Draw M samples $z_{i,m}^{(k)}$ under model $\hat{\theta}^{(t,k)}_i$
%\STATE Compute the surrogate sufficient statistics $\tilde{S}_{i}^{(t,k+1)}$
%\ENDFOR
%\STATE Devices send local statistics $\tilde{S}_{i}^{(t,k+1)}$ to server.
%\ENDFOR
%\STATE Server computes \textbf{global model using the aggregated statistics}:
%$$
%\hat{\theta}^{(k+1)} = \overline{\theta}( \tilde{S}^{(t,k+1)}) 
%$$
%where $\tilde{S}^{(t,k+1)} = (\tilde{S}_i^{(t,k+1)}, i \in D_r)$  and send global model back to the devices. 
%\ENDFOR
%\end{algorithmic}
%\end{algorithm}
\bibliographystyle{plain}
\bibliography{ref}

\end{document} 