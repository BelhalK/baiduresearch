\documentclass[11pt]{article}
%\usepackage{fullpage,graphicx,algorithm,algorithmic,bm,amsmath,amsthm,amssymb,color,hyperref,cite,natbib}

% if you need to pass options to natbib, use, e.g.:
%\PassOptionsToPackage{numbers}{natbib}
\usepackage{natbib,fullpage}
\usepackage{bm,amsmath,amsthm,amssymb,multicol,algorithmic,algorithm,enumitem}
\usepackage{wrapfig,lipsum}
\usepackage[textwidth=1cm,textsize=footnotesize]{todonotes}

% ready for submission
\usepackage{neurips_2020}

\usepackage[colorlinks=true,
linkcolor=red,
urlcolor=blue,
citecolor=blue]{hyperref}
\usepackage{hyperref}
\usepackage{cleveref}

\setlength{\parskip}{.2cm}

\newtheorem{Fact}{Fact}
\newtheorem{Lemma}{Lemma}
\newtheorem{Prop}{Proposition}
\newtheorem{Theorem}{Theorem}
\newtheorem{Def}{Definition}
\newtheorem{Corollary}{Corollary}
\newtheorem{Conjecture}{Conjecture}
\newtheorem{Property}{Property}
\newtheorem{Observation}{Observation}
%\theorembodyfont{\rmfamily}
\newtheorem{Exa}{Example}
\newtheorem{assumption}{H\!\!}
\newtheorem{assumptionA}{S\!\!}
\newtheorem{assumptionL}{L\!\!}
\newtheorem{Remark}{Remark}
\newtheorem*{Lemma*}{Lemma}
\newtheorem*{Theorem*}{Theorem}
 \makeatletter
\renewenvironment{proof}[1][\proofname]{%
   \par\pushQED{\qed}\normalfont%
   \topsep6\p@\@plus6\p@\relax
   \trivlist\item[\hskip\labelsep\bfseries#1]%
   \ignorespaces
}{%
   \popQED\endtrivlist\@endpefalse
}
\makeatother

%%%%%%%%%%% Stuffs for Tikz %%%%%%%%%%%%%%%%%%
\usepackage{pgfplots}
\usepackage{xargs}
\usepackage{stmaryrd}
\usetikzlibrary{arrows,shapes,calc,tikzmark,backgrounds,matrix,decorations.markings}
\usepgfplotslibrary{fillbetween}

\pgfplotsset{compat=1.3}

\usepackage{relsize}
\tikzset{fontscale/.style = {font=\relsize{#1}}
    }

\definecolor{lavander}{cmyk}{0,0.48,0,0}
\definecolor{violet}{cmyk}{0.79,0.88,0,0}
\definecolor{burntorange}{cmyk}{0,0.52,1,0}

\def\lav{lavander!90}
\def\oran{orange!30}

\definecolor{asuorange}{rgb}{1,0.699,0.0625}
\definecolor{asured}{rgb}{0.598,0,0.199}
\definecolor{asuborder}{rgb}{0.953,0.484,0}
\definecolor{asugrey}{rgb}{0.309,0.332,0.340}
\definecolor{asublue}{rgb}{0,0.555,0.836}
\definecolor{asugold}{rgb}{1,0.777,0.008}

%%%%%%%%%%%%%%%%%%%%%%%%%%%%%%%%%%%%%


\usepackage{shortcuts_OPT}

%\renewcommand{\textwidth}{5.5in}

% Here's the definition of Sb, stolen from amstex
    \makeatletter
    \def\multilimits@{\bgroup
  \Let@
  \restore@math@cr
  \default@tag
 \baselineskip\fontdimen10 \scriptfont\tw@
 \advance\baselineskip\fontdimen12 \scriptfont\tw@
 \lineskip\thr@@\fontdimen8 \scriptfont\thr@@
 \lineskiplimit\lineskip
 \vbox\bgroup\ialign\bgroup\hfil$\m@th\scriptstyle{##}$\hfil\crcr}
    \def\Sb{_\multilimits@}
    \def\endSb{\crcr\egroup\egroup\egroup}
\makeatother

\newtheoremstyle{k}         %name
    {\baselineskip}{2\topsep}      %space above and below
    {\rm}                   %Body font
    {0pt}{\bfseries}  %Heading indent and font
    {}                      %after heading
    { }                      %head after space
    {\thmname{#1}\thmnumber{#2}.}

\theoremstyle{k}
\newtheorem{q}{Q}
\parindent=0pt



\makeatletter
\DeclareRobustCommand*\cal{\@fontswitch\relax\mathcal}
\makeatother

\begin{document}
\title{Optimistic Acceleration of AMSGrad for Nonconvex Optimization.}
%\author{}
\date{\today}

\maketitle

%\begin{abstract}
%\end{abstract}

\section{Nonconvex Analysis}
We tackle the following classical optimization problem:
\beq\label{eq:minproblem}
\min \limits_{w \in \Theta} f(w) \eqdef \EE[ f(w, \xi)]
\eeq
where $\xi$ is some random noise and only noisy versions of the objective function are accessible in this work.
The objective function $f(w)$ is (potentially) nonconvex and has Lipschitz gradients.

\textbf{Optimistic Algorithm}
We present here the algorithm studied in this paper to tackle problem \eqref{eq:minproblem}.
Set the terminating iteration number, $K \in \{0,\dots,K_{\sf max}-1\}$, as a discrete r.v.~with:
\beq \label{eq:random}
   P( K = \ell ) = \frac{ \eta_{\ell} }{\sum_{j=0}^{K_{\sf max}-1} \eta_j}.
\eeq
where $K_{\sf max}$ $\leftarrow$ is the maximum number of iteration.
The random termination number \eqref{eq:random} is inspired by \citep{ghadimi2013stochastic} which enables one to show non-asymptotic convergence to stationary point for non-convex optimization. 
Consider constants $(\beta_1, \beta_2) \in [0,1]$, a sequence of decreasing stepsizes $\{\eta_k\}_{k>0}$, Algorithm~\ref{alg:optamsgrad} introduces the new optimistic AMSGrad method.

\begin{algorithm}[H]
\caption{OPTIMISTIC-AMSGRAD}\label{alg:optamsgrad}
  \begin{algorithmic}[1]
  \STATE \textbf{Input:} Parameters $\beta_{1}, \beta_{2}, \epsilon, \eta_{k}$
  \STATE \textbf{Init.:} $w_{1}=w_{-1 / 2} \in \mathcal{K} \subseteq \mathbb{R}^{d} \text { and } v_{0}=\epsilon \mathbf{1} \in \mathbb{R}^{d}$
  \FOR {$k=1,2,\dots, K$}
  \STATE Get mini-batch stochastic gradient $g_{k}$ at $w_{k}$
   \STATE $\theta_{k}=\beta_{1} \theta_{k-1}+\left(1-\beta_{1}\right) g_{k}$
   \STATE $v_{k}=\beta_{2} v_{k-1}+\left(1-\beta_{2}\right) g_{k}^{2}$
   \STATE $ \hat{v}_{k}=\max \left(\hat{v}_{k-1}, v_{k}\right)$
   \STATE $ w_{k+\frac{1}{2}}=\Pi_{\mathcal{K}}\left[w_k-\eta_{k} \frac{\theta_{k}}{\sqrt{\hat{v}_{k}}}\right]$
   \STATE $ w_{k+1}=\Pi_{\mathcal{K}}\left[w_{k+\frac{1}{2}}-\eta_{k} \frac{h_{k+1}}{\sqrt{v}_{k}}\right]$
   \STATE $ \quad \text{where} \quad h_{k+1}:=\beta_{1} \theta_{k-1} + (1-\beta_{1}) m_{k+1}$
      \STATE $ \quad\quad \text{and} \quad m_{k+1}$ is a guess of $g_{k+1}$
\ENDFOR
\STATE \textbf{Return}: $w_{K+1}$.
  \end{algorithmic}
\end{algorithm}\vspace{.1cm}
The final update at iteration $k$ can be summarized as:
\beq\label{eq:finalupdate}
w_{k+1}=w_{k}-\eta_{k} \frac{\theta_{k}}{\sqrt{\hat{v}_{k}}}-\eta_{k} \frac{h_{k+1}}{\sqrt{v}_{k}}
\eeq



We make the following assumptions:
\begin{assumption}\label{ass:nonconv}
The loss function $f(w)$ is nonconvex \wrt the parameter $w$.
\end{assumption}

\begin{assumption}\label{ass:boundedparam}
For any $k >0$, the estimated weight $w_k$ stays within a $\ell_{\infty}-$ball. There exists a constant $W >0$ such that:
\beq
\norm{w_k} \leq W \quad \textrm{almost surely}
\eeq
\end{assumption}

\begin{assumption}\label{ass:smooth}
The function $f(w)$ is $L$-smooth (has $L$-Lipschitz gradients) w.r.t. the parameter w.
There exist some constant $L > 0$ such that for $(w, \vartheta) \in \Theta^2$:
\beq
f(w) - f(\vartheta) - \nabla f(\vartheta)^\top(w - \vartheta) \leq \frac{L}{2} \norm{w - \vartheta}^2\eqsp.
\eeq
\end{assumption}
We assume that the optimistic guess $m_k$ at iteration $k$ and the true gradient $g_k$ are correlated:
\begin{assumption}\label{ass:guessbound}
There exists a constant $a \in \rset $ such that for any $k >0$:
$$
 \pscal{m_k}{ g_k}  \leq a \|g_k\|^2
$$
\end{assumption}
Classically in nonconvex optimization, see \citep{ghadimi2013stochastic}, we make an assumption on the magnitude of the gradient:
\begin{assumption}\label{ass:bounded}
There exists a constant $\major >0$ such that 
$$
\norm{\nabla f(w, \xi)} < \major \quad \textrm{for any $w$ and $\xi$}
$$
\end{assumption}

We begin with some auxiliary Lemmas important for the analysis. 
The first one ensures bounded norms of various quantities of interests (resulting from the classical stochastic gradient boundedness assumption):
\begin{Lemma}\label{lem:bound}
Assume assumption H~\ref{ass:bounded}, then the quantities defined in Algorithm~\ref{alg:optamsgrad} satisfy for any $w \in \Theta$ and $k>0$:
$$ \|\nabla f(w_k)\| < \major ,~~~\|\theta_k \| < \major ,~~~\|\hat{v}_k\| < \major^2 \eqsp.$$
\end{Lemma}
See Proof in Appendix~\ref{app:lembound}.

Then, following \citep{yan2018unified} and their study of the SGD with Momentum (not AMSGrad but simple momentum) we denote for any $k >0$:
\beq\label{eq:deftilde}
\overline{w}_k = w_k + \frac{\beta_1}{1 - \beta_1} (w_k - w_{k-1}) = \frac{1}{1 - \beta_1} w_k -  \frac{\beta_1}{1 - \beta_1} w_{k-1} \eqsp,
\eeq
and derive an important Lemma:
\begin{Lemma}\label{lem:momentum}
Assume a strictly positive and non increasing sequence of stepsizes $\{\eta_k \}_{k>0}$, $\beta_ \in [0,1]$, then the following holds:
\beq
\overline{w}_{k+1} - \overline{w}_k \leq \frac{\beta_1}{1 - \beta_1} \tilde{\theta}_{k-1} \left[ \eta_{k-1} \hat{v}_{k-1}^{-1/2} - \eta_{k} \hat{v}_{k}^{-1/2}\right] - \eta_{k} \hat{v}_{k}^{-1/2} \tilde{g}_k \eqsp,
\eeq
where $\tilde{\theta}_k = \theta_k + \beta_1 \theta_{k-1}$ and $\tilde{g}_k = g_k - \beta_1 m_k + \beta_1 g_{k-1} + m_{k+1} $.
\end{Lemma}
See Proof in Appendix~\ref{app:lemmomentum}

\begin{Lemma}\label{lem:squarev}
Assume H~\ref{ass:bounded}, a strictly positive and a sequence of constant stepsizes $\{\eta_k \}_{k>0}$, $\beta_ \in [0,1]$, then the following holds:
\beq
\sum_{k=1}^{K_{\sf max}} \eta_{k}^{2} \EE \left[\left\|\hat{v}_{k}^{-1/2} \theta_{k}\right\|_{2}^{2}\right] \leq  \frac{\eta^{2} d K_{\sf max} (1- \beta_1)}{(1 - \beta_2)(1-\gamma)} 
\eeq
\end{Lemma}
See Proof in Appendix~\ref{app:lemsquarev}.

We now formulate the main result of our paper giving a finite-time upper bound of the quantity $\EE\left[\|\nabla f(w_K)\|^2\right]$ where K is a random termination number distributed according to \ref{eq:random}, see \citep{ghadimi2013stochastic}.

\begin{Theorem}\label{thm:boundopt}
Assume H~\ref{ass:smooth}-H~\ref{ass:bounded}, $(\beta_1, \beta_2) \in [0,1]$ and a sequence of decreasing stepsizes $\{\eta_k\}_{k>0}$, then the following result holds:
\beq
\begin{split}
\EE\left[\|\nabla f(w_K)\|^2\right] \leq \tilde{C}_1 \sqrt{\frac{d}{K_{\sf max}}} + \tilde{C}_2 \frac{1}{K_{\sf max}}
\end{split}
\eeq
where $K$ is a random termination number distributed according \eqref{eq:random} and the constants are defined as follows:
\beq
\begin{split}
&\tilde{C}_1 = C_1 +  \frac{\major}{(1 - a\beta_1) + (\beta_1 + a)} \left[ \frac{a(1 - \beta_1)^2}{1-\beta_2} + 2L \frac{1}{1-\beta_2}  \right]\\
&C_1 = \frac{\major}{(1 - a\beta_1) + (\beta_1 + a)}  \Delta f + \frac{ 4L \left(\frac{\beta_1}{1 - \beta_1}\right)^2 \major}{(1 - a\beta_1) + (\beta_1 + a)} \frac{(1 + \beta_1^2) (1- \beta_1)}{(1 - \beta_2)(1-\gamma)}\\
&\tilde{C}_2 = \frac{\major}{(1 - \beta_1) \left((1 - a\beta_1) + (\beta_1 + a)\right)} \tilde{\major}^2   \EE\left[ \norm{\hat{v}_{0}^{-1/2}}    \right]
\end{split}
\eeq
\end{Theorem}
See proof in Appendix~\ref{app:thmboundopt}.

We remark that the bound for our OPT-AMSGrad method matched the complexity bound of $\mathcal{O}\left( \sqrt{\frac{d}{K_{\sf max}}} + \frac{1}{K_{\sf max}} \right)$ of \citep{ghadimi2013stochastic} for SGD and \citep{zhou2018convergence} for AMSGrad method.

\section{Checking H~\ref{ass:boundedparam} for a Deep Neural Network}
We show in this section that the weights satisfy assumption H~\ref{ass:boundedparam} and stay in a bounded set when the model we are fitting, using our method, is a fully connected feed forward neural network. 
The activation function for this section will be sigmoid function and we add a $\ell_2$ regularization. 

For the sake of notation, we assume $\beta_1 = 0$.
We consider a fully connected feed forward neural network with $L$ layers modeled by the function $\textsf{MLN}(w, \xi): \rset^l \to \rset$:
\beq\label{eq:dnnmodel}
\textsf{MLN}( w, \xi) = \sigma\left(w^{(L)} \sigma\left(w^{(L-1)} \ldots \sigma\left(w^{(1)} \xi \right)\right)\right)
\eeq
where $w = [w^{(1)}, w^{(2)}, \cdots , w^{(L)}]$ is the vector of parameters, $\xi \in \rset^l$ is the input data and $\sigma$ is the sigmoid activation function. We assume a $l$ dimension input data and a scalar output for simplicity.
The stochastic objective function \eqref{eq:minproblem} reads:
\beq\label{eq:lossmln}
f(w, \xi) = \mathcal{L}(\textsf{MLN}( w, \xi), y) +\frac{\lambda}{2}\norm{w}^2
\eeq
where $\mathcal{L}(\cdot, y)$ is the loss function (can be Huber loss or cross entropy), $y$ are the true labels and $\lambda >0$ is the regularization parameter.
For any layer index $\ell \in [1, L]$ we denote the output of layer $\ell$ by $h^{(\ell)}(w,\xi)$:
$$
h^{(\ell)}(w,\xi) = \sigma\left(w^{(\ell)} \sigma\left(w^{(\ell-1)} \ldots \sigma\left(w^{(1)} \xi \right)\right)\right)
$$

The following Lemma verifies that assumption H~\ref{ass:boundedparam} is satisfied with a fully connected feed forward neural network \eqref{eq:dnnmodel}:
\begin{Lemma}\label{lem:dnnh2} 
Given the multilayer model \eqref{eq:dnnmodel}, assume the boundedness of the input data and of the loss function, \ie for any $\xi \in \rset^l$ and $y \in \rset$ there is a constant $T >0$ such that:
\beq\label{eq:mildassumptions}
\norm{\xi} \leq 1 \quad \textrm{a.s.} \quad \textrm{and} |\mathcal{L}'(\cdot, y)| \leq T
\eeq
where $\mathcal{L}'(\cdot, y)$ denotes its derivative \wrt the parameter. Then for each layer $\ell \in [1,L]$, there exist a constant $A_{(\ell)}$ such that:
$$
\norm{w^{(\ell)}} \leq A_{(\ell)}
$$
\end{Lemma}
See Proof in Appendix~\ref{app:lemdnnh2}


\clearpage

\section{Convex Analysis:}

\begin{Theorem} \label{thm:convexmain}
Let $\beta_{1}=0$. Suppose the learner incurs a sequence of convex loss functions $\{ \ell_{t}(\cdot) \}$.
\textsc{Optimistic-AMSGrad} (Algorithm~\ref{alg:optamsgrad}) 
has regret 
\begin{equation} \label{bound:optada}
\begin{aligned}
\textstyle
 \text{Regret}_T &  \leq  \frac{1}{\eta_{\min}} D_{\infty}^2 \sum_{i=1}^d \hat{v}_{T}^{1/2}[i] + \frac{ B_{\psi_1}(w^*, w_{1/2})}{\eta_1} + \sum_{t=1}^T\frac{\eta_t}{2} \| g_t - m_t  \|_{\psi_{t-1}^*}^2 \eqsp,
\end{aligned}
\end{equation}
where $g_{t}:= \nabla \ell_{t}(w_t)$ and $\eta_{{\min}} := \min_{{t}} \eta_{t}$.
The result holds for any benchmark $w^{*} \in \Theta$ and any step size sequence $\{ \eta_t \}$.
\end{Theorem}


\begin{proof}
By regret decomposition, we have that
\begin{equation} \label{nn1}
\begin{aligned}
\textstyle
 \text{\it Regret}_T  & := \sum_{t=1}^T \ell_t(w_t) - \min_{w \in \Theta} \sum_{t=1}^T \ell_t(w)  \textstyle  \\
 & \leq \sum_{t=1}^T  \langle w_t - w^*, \nabla \ell_t(w_t) \rangle
\\ & \textstyle = \sum_{t=1}^T \langle  w_t - w_{t+\frac{1}{2}} , g_t - m_t \rangle + \langle w_t - w_{t+\frac{1}{2}}, m_t \rangle + \langle w_{t+\frac{1}{2}} - w^*, g_t  \rangle,
\end{aligned}
\end{equation}
where we denote $g_t:=\nabla \ell_t(w_t)$.


Recall the notation $\psi_t(x)$ and the Bregman divergence $B_{\psi_t}(u,v)$
we defined in the beginning of this section.
Now we are going to exploit a useful inequality (which appears in e.g.,~\cite{T08}); for any update of the form $\hat{w} = \arg\min_{w \in \Theta} \langle w, \theta \rangle + B_{\psi}(w, v)$, it holds that
\begin{equation} \label{ii}
\langle \hat{w} - u, \theta \rangle \leq B_{\psi}( u, v ) - B_{\psi}( u, \hat{w}) - B_{\psi}( \hat{w}, v),
\end{equation}
for any $u \in \Theta$.

For $\beta_1=0$, we can rewrite the update on line 8 of (Algorithm~\ref{alg:optamsgrad}) as
\begin{equation} \label{nc1}
\textstyle w_{t+\frac{1}{2}} = \arg\min_{w \in \Theta} \eta_t \langle w, g_t \rangle
+ B_{\psi_t}(w, w_{t} ),
\end{equation}
By using (\ref{ii}) for (\ref{nc1}) with $\hat{w} = w_{t+\frac{1}{2}}$ (the output of the minimization problem), $u = w^*$ and $v = w_t$, we have
\begin{equation} \label{nn2}
\begin{aligned}
 \langle w_{t+\frac{1}{2}} - w^*, & g_t \rangle
\leq \frac{1}{\eta_t}\big[ B_{\psi_t}( w^*, w_{t}) -B_{\psi_t}(w^*,  w_{t+\frac{1}{2}} ) - B_{\psi_t}(w_{t+\frac{1}{2}}, w_{t}) \big].
\end{aligned}
\end{equation}

We can also rewrite the update on line 9 of (Algorithm~\ref{alg:optamsgrad}) at time $t$ as
%\begin{equation} \label{nc2}
%\textstyle w_{t} = \arg\min_{w \in \Theta} \eta_{t} \langle w, m_{t} \rangle + B_{\psi_{t-1}}(w, w_{t-\frac{1}{2}} ).
%\end{equation}
\begin{equation} \label{nc2}
\textstyle w_{t+1} = \arg\min_{w \in \Theta} \eta_{t+1} \langle w, m_{t+1} \rangle + B_{\psi_t}(w, w_{t+\frac{1}{2}} ).
\end{equation}
and, by using (\ref{ii}) for (\ref{nc2}), with $\hat{w} = w_{t}$ (the output of the minimization problem), $u = w_{t + \frac{1}{2}}$ and $v = w_{t- \frac{1}{2}}$, we have
\begin{equation} \label{nn3}
\begin{aligned}
\langle w_t - w_{t+\frac{1}{2}}, & m_t  \rangle
\leq \frac{1}{\eta_t}\big[ B_{\psi_{t-1}}( w_{t+\frac{1}{2}}, w_{t-\frac{1}{2}}) - B_{\psi_{t-1}}(w_{t+\frac{1}{2}}, w_t ) - B_{\psi_{t-1}}(w_{t}, w_{t-\frac{1}{2}}) \big],
\end{aligned}
\end{equation}
By (\ref{nn1}), (\ref{nn2}), and (\ref{nn3}), we obtain
\begin{equation} \label{nnn}
\begin{aligned}
 \text{Regret}_T & \overset{(\ref{nn1})}{\leq} \sum_{t=1}^T \langle  w_t - w_{t+\frac{1}{2}} , g_t - m_t \rangle + \langle w_t - w_{t+\frac{1}{2}}, m_t \rangle + \langle w_{t \frac{1}{2}} - w^*, g_t  \rangle \\
& \overset{(\ref{nn2}), (\ref{nn3})}{\leq}  \sum_{t=1}^T \| w_t - w_{t+\frac{1}{2} } \|_{\psi_{t-1}} \| g_t - m_t  \|_{\psi_{t-1}^*}
+ \frac{1}{\eta_t} \big[ B_{\psi_{t-1}}( w_{t+\frac{1}{2}}, w_{t-\frac{1}{2}}) - B_{\psi_{t-1}}(w_{t+\frac{1}{2}}, w_t ) \\
& - B_{\psi_{t-1}}(w_{t}, w_{t-\frac{1}{2}}) + B_{\psi_t}( w^*, w_{t}) -B_{\psi_t}(w^*,  w_{t+\frac{1}{2}} ) - B_{\psi_t}(w_{t+\frac{1}{2}}, w_{t}) \big],
\end{aligned}
\end{equation}
which is further bounded by
\begin{equation} \label{nnnn}
\begin{aligned}
 \text{Regret}_T &  \overset{(a)}{\leq} \sum_{t=1}^T \Big\{ \frac{1}{2 \eta_t} \| w_t - w_{t+\frac{1}{2} } \|_{\psi_{t-1}}^2 + \frac{\eta_t}{2} \| g_t - m_t  \|_{\psi_{t-1}^*}^2 + \frac{1}{\eta_t} \big( B_{\psi_{t-1}}( w_{t+\frac{1}{2}}, w_{t-\frac{1}{2}}) - \frac{1}{2} \| w_{t+\frac{1}{2}} - w_t \|_{\psi_{t-1}}^2\\
& - B_{\psi_{t-1}}(w_{t}, w_{t-\frac{1}{2}}) +  B_{\psi_t}( w^*, w_{t}) - B_{\psi_t}(w^*,  w_{t+\frac{1}{2}} ) - B_{\psi_t}(w_{t+\frac{1}{2}}, w_{t})  \big) \Big\}\\
& \leq \sum_{t=1}^T \Big\{ \frac{\eta_t}{2} \| g_t - m_t  \|_{\psi_{t-1}^*} \\
& + dd
\Big\},
\end{aligned}
\end{equation}
where $(a)$ is because
$
\| w_t - w_{t+\frac{1}{2} } \|_{\psi_{t-1}} \| g_t - m_t  \|_{\psi_{t-1}^*} = \inf_{ \beta > 0 }   \frac{1}{2\beta}
\| w_t - w_{t+\frac{1}{2} } \|_{\psi_{t-1}}^2 +
 \frac{\beta}{2} \| g_t - m_t  \|_{\psi_{t-1}^*}^2$ by Young's inequality and the 1-strongly convex of $\psi_{t-1}(\cdot)$ with respect to $\| \cdot \|_{\psi_{t-1}}$.
%
%To proceed,
%notice that
%\begin{equation}  \label{nn4}
%\begin{aligned}
%\textstyle
% B_{\psi_{t+1}}( w^*, w_{t+\frac{1}{2}}) - B_{\psi_t}( w^*, w_{t+\frac{1}{2}}) &= \langle w^* - w_{t+\frac{1}{2}} , \text{diag}(\hat{v}_{t+1}^{1/2} -\hat{v}_t^{1/2} ) ( w^* - w_{t+\frac{1}{2}} ) \rangle\\
%  & \leq ( \max_i (w^*[i] - w_{t+\frac{1}{2}}[i] )^2  )\cdot ( \sum_{i=1}^d \hat{v}_{t+1}^{1/2}[i] -\hat{v}_t^{1/2}[i] )
%\end{aligned}
%\end{equation}
%and
%\begin{equation} \label{nn5}
%\begin{aligned}
%\textstyle
% B_{\psi_{t-1}}( w_{t+\frac{1}{2}}, w_{t-\frac{1}{2}}) - B_{\psi_t}(w_{t+\frac{1}{2}}, w_{t-\frac{1}{2}}) & = \langle w_{t+\frac{1}{2}} - w_{t-\frac{1}{2}} , \text{diag}(\hat{v}_{t-1}^{1/2} -\hat{v}_t^{1/2} ) ( w_{t+\frac{1}{2}} - w_{t-\frac{1}{2}} ) \rangle\\
%& \leq 0 \eqsp,
%\end{aligned}
%\end{equation}
%as the sequence $\{\hat{v}_t\}$ is non-decreasing.
%Therefore, by (\ref{nnnn}),(\ref{nn4}),(\ref{nn5}), we have
%\begin{equation}\notag
%\begin{aligned}
%\textstyle
% \text{\it Regret}_T \leq \frac{1}{\eta_{\min}} D_{\infty}^2 \sum_{i=1}^d \hat{v}_{T}^{1/2}[i] + \frac{ B_{\psi_1}(w^*, w_{1/2})}{\eta_1}
%+ \sum_{t=1}^T\frac{\eta_t}{2} \| g_t - m_t  \|_{\psi_{t-1}^*}^2 \eqsp.
%\end{aligned}
%\end{equation}
%This completes the proof.

\end{proof}



\newpage

\bibliographystyle{abbrvnat}
\bibliography{reference}
\newpage

\appendix
\section{Proofs of Auxiliary Lemmas}
\subsection{Proof of Lemma~\ref{lem:bound}}\label{app:lembound}
\begin{Lemma*}
Assume assumption H~\ref{ass:bounded}, then the quantities defined in Algorithm~\ref{alg:optamsgrad} satisfy for any $w \in \Theta$ and $k>0$:
$$ \|\nabla f(w_k)\| < \major ,~~~\|\theta_k \| < \major ,~~~\|\hat{v}_k\| < \major^2 \eqsp.$$
\end{Lemma*}
\begin{proof}
Assume assumption H~\ref{ass:bounded} we have:
$$
\norm{\nabla f(w)} = \norm{\EE[\nabla f(w, \xi)]} \leq \EE[\norm{\nabla f(w, \xi)}] \leq \major
$$
By induction reasoning, since $\norm{\theta_0} = 0 \leq \major$ and suppose that for $\norm{\theta_k}\leq \major$ then we have 
\beq
\begin{split}
\norm{\theta_{k+1}}  =\norm{\beta_{1} \theta_{k}+\left(1-\beta_{1}\right) g_{k+1}} \leq \beta_1 \norm{\theta_{k}} + (1 - \beta_1) \norm{g_{k+1}} \leq \major
\end{split}
\eeq
Using the same induction reasoning we prove that
\beq
\begin{split}
\norm{\hat{v}_{k+1}}  =\norm{\beta_{2} \hat{v}_{k}+\left(1-\beta_{2}\right) g_{k+1}^2} \leq \beta_2 \norm{\hat{v}_{k}} + (1 - \beta_1) \norm{g^2_{k+1}} \leq \major^2
\end{split}
\eeq
\end{proof}

\subsection{Proof of Lemma~\ref{lem:momentum} }\label{app:lemmomentum}
\begin{Lemma*}
Assume a strictly positive and non increasing sequence of stepsizes $\{\eta_k \}_{k>0}$, $\beta_ \in [0,1]$, then the following holds:
\beq
\overline{w}_{k+1} - \overline{w}_k \leq \frac{\beta_1}{1 - \beta_1} \tilde{\theta}_{k-1} \left[ \eta_{k-1} \hat{v}_{k-1}^{-1/2} - \eta_{k} \hat{v}_{k}^{-1/2}\right] - \eta_{k} \hat{v}_{k}^{-1/2} \tilde{g}_k \eqsp,
\eeq
where $\tilde{\theta}_k = \theta_k + \beta_1 \theta_{k-1}$ and $\tilde{g}_k = g_k - \beta_1 m_k + \beta_1 g_{k-1} + m_{k+1} $.
\end{Lemma*}
\begin{proof}
By definition \eqref{eq:deftilde} and using the Algorithm updates, we have:
\beq
\begin{split}
\overline{w}_{k+1} - \overline{w}_k  &= \frac{1}{1 - \beta_1} ( w_{k+1} - w_k)  -  \frac{\beta_1}{1 - \beta_1}( w_{k} - w_{k-1})\\
& = - \frac{1}{1 - \beta_1} \eta_{k} \hat{v}_{k}^{-1/2} (\theta_k + h_{k+1})  +  \frac{\beta_1}{1 - \beta_1}\eta_{k-1} \hat{v}_{k-1}^{-1/2} (\theta_{k-1} + h_{k})\\
& = - \frac{1}{1 - \beta_1}  \eta_{k} \hat{v}_{k}^{-1/2} (\theta_k + \beta_1 \theta_{k-1}) -\frac{1}{1 - \beta_1}  \eta_{k} \hat{v}_{k}^{-1/2} (1- \beta_1) m_{k+1}\\
& + \frac{\beta_1}{1 - \beta_1} \eta_{k-1} \hat{v}_{k-1}^{-1/2} (\theta_{k-1} + \beta_1 \theta_{k-2}) + \frac{\beta_1}{1 - \beta_1}  \eta_{k-1} \hat{v}_{k-1}^{-1/2} (1- \beta_1) m_{k}
\end{split}
\eeq
Denote $\tilde{\theta}_k = \theta_k + \beta_1 \theta_{k-1}$ and $\tilde{g}_k = g_k - \beta_1 m_k + \beta_1 g_{k-1} + m_{k+1} $.
Notice that $\tilde{\theta}_k = \beta_1 \tilde{\theta}_{k-1} + (1 - \beta_1) (g_k + \beta_1 g_{k-1})$.
\beq
\begin{split}
\overline{w}_{k+1} - \overline{w}_k \leq \frac{\beta_1}{1- \beta_1} \tilde{\theta}_{k-1} \left[ \eta_{k-1} \hat{v}_{k-1}^{-1/2} - \eta_{k} \hat{v}_{k}^{-1/2} \right] - \eta_k \hat{v}_k^{-1/2} \tilde{g}_k
\end{split}
\eeq
\end{proof}

\subsection{Proof of Lemma~\ref{lem:squarev} }\label{app:lemsquarev}
\begin{Lemma*}
Assume H~\ref{ass:bounded}, a strictly positive and a sequence of constant stepsizes $\{\eta_k \}_{k>0}$, $\beta_ \in [0,1]$, then the following holds:
\beq
\sum_{k=1}^{K_{\sf max}} \eta_{k}^{2} \EE \left[\left\|\hat{v}_{k}^{-1/2} \theta_{k}\right\|_{2}^{2}\right] \leq  \frac{\eta^{2} d K_{\sf max} (1- \beta_1)}{(1 - \beta_2)(1-\gamma)} 
\eeq
\end{Lemma*}
\begin{proof}
We denote by index $p \in [1,d]$ the dimension of each component of vectors of interest. 
Noting that for any $k >0$ and dimension $p$ we have $\hat{v}_{k,p} \geq v_{k,p}$, then:
\beq
\begin{split}
\eta_{k}^{2} \EE \left[\left\|\hat{v}_{k}^{-1/2} \theta_{k}\right\|_{2}^{2}\right] &=\eta_{k}^{2} \mathbb{E}\left[\sum_{p=1}^{d} \frac{\theta_{k, p}^{2}}{\hat{v}_{k, p}}\right]  \\
& \leq \eta_{k}^{2} \mathbb{E}\left[\sum_{i=1}^{d} \frac{\theta_{k, p}^{2}}{v_{k, p}}\right] \\
& \leq \eta_{k}^{2} \mathbb{E}\left[\sum_{i=1}^{d} \frac{( \sum_{t=1}^k (1 - \beta_1) \beta_1^{k-t} g_{t,p})^{2}}{ \sum_{t=1}^k (1 - \beta_2) \beta_2^{k-t} g^2_{t,p}}\right] 
\end{split}
\eeq
where the last inequality is due to initializations.
Denote $\gamma = \frac{\beta_1}{\beta_2}$.
Then,
\beq
\begin{split}
\eta_{k}^{2} \EE \left[\left\|\hat{v}_{k}^{-1/2} \theta_{k}\right\|_{2}^{2}\right] &\leq \frac{\eta_{k}^{2} (1- \beta_1)^2}{1 - \beta_2}  \mathbb{E}\left[\sum_{i=1}^{d} \frac{( \sum_{t=1}^k \beta_1^{k-t} g_{t,p})^{2}}{ \sum_{t=1}^k \beta_2^{k-t} g^2_{t,p}}\right] \\
& \overset{(a)}{\leq}\frac{\eta_{k}^{2} (1- \beta_1)}{1 - \beta_2}  \mathbb{E}\left[\sum_{i=1}^{d} \frac{ \sum_{t=1}^k \beta_1^{k-t} g_{t,p}^{2}}{ \sum_{t=1}^k \beta_2^{k-t} g^2_{t,p}}\right]\\
& \leq \frac{\eta_{k}^{2} (1- \beta_1)}{1 - \beta_2}  \mathbb{E}\left[\sum_{i=1}^{d}\sum_{t=1}^k \gamma^{k-t}\right]  = \frac{\eta_{k}^{2} d (1- \beta_1)}{1 - \beta_2}  \mathbb{E}\left[\sum_{t=1}^k  \gamma^{k-t}\right] 
\end{split}
\eeq
where $(a)$ is due to $ \sum_{t=1}^k \beta_1^{k-t} \leq \frac{1}{1 - \beta_1}$.
Summing from  $k =1$ to $k = K_{\sf max}$ on both sides yields:
\beq
\begin{split}
\sum_{k=1}^{K_{\sf max}} \eta_{k}^{2} \EE \left[\left\|\hat{v}_{k}^{-1/2} \theta_{k}\right\|_{2}^{2}\right] &\leq   \frac{\eta_{k}^{2} d (1- \beta_1)}{1 - \beta_2}  \mathbb{E}\left[ \sum_{k=1}^{K_{\sf max}} \sum_{t=1}^k  \gamma^{k-t}\right]\\
& \leq  \frac{\eta^{2} d K (1- \beta_1)}{1 - \beta_2}  \mathbb{E}\left[ \sum_{t=t}^k   \gamma^{k-t}\right]\\
& \leq  \frac{\eta^{2} d K (1- \beta_1)}{(1 - \beta_2)(1-\gamma)} 
\end{split}
\eeq
where the last inequality is due to $\sum_{t=1}^k   \gamma^{k-t} \leq \frac{1}{1 - \gamma}$ by definition of $\gamma$.
\end{proof}



\section{Proofs of Theorem~\ref{thm:boundopt}}\label{app:thmboundopt}
\begin{Theorem*}
Assume H~\ref{ass:smooth}-H~\ref{ass:bounded}, $(\beta_1, \beta_2) \in [0,1]$ and a sequence of decreasing stepsizes $\{\eta_k\}_{k>0}$, then the following result holds:
\beq
\begin{split}
\EE\left[\|\nabla f(w_K)\|^2\right] \leq \tilde{C}_1 \sqrt{\frac{d}{K_{\sf max}}} + \tilde{C}_2 \frac{1}{K_{\sf max}}
\end{split}
\eeq
where $K$ is a random termination number distributed according \eqref{eq:random} and the constants are defined as follows:
\beq
\begin{split}
&\tilde{C}_1 = C_1 +  \frac{\major}{(1 - a\beta_1) + (\beta_1 + a)} \left[ \frac{a(1 - \beta_1)^2}{1-\beta_2} + 2L \frac{1}{1-\beta_2}  \right]\\
&C_1 = \frac{\major}{(1 - a\beta_1) + (\beta_1 + a)}  \Delta f + \frac{ 4L \left(\frac{\beta_1}{1 - \beta_1}\right)^2 \major}{(1 - a\beta_1) + (\beta_1 + a)} \frac{(1 + \beta_1^2) (1- \beta_1)}{(1 - \beta_2)(1-\gamma)}\\
&\tilde{C}_2 = \frac{\major}{(1 - \beta_1) \left((1 - a\beta_1) + (\beta_1 + a)\right)} \tilde{\major}^2   \EE\left[ \norm{\hat{v}_{0}^{-1/2}}    \right]
\end{split}
\eeq
\end{Theorem*}

\begin{proof}
Using H~\ref{ass:smooth} and the iterate $\overline{w}_k$ we have:
\beq\label{eq:smoothness}
\begin{split}
f(\overline{w}_{k+1}) & \leq  f(\overline{w}_k) + \nabla f(\overline{w}_k)^\top (\overline{w}_{k+1} - \overline{w}_k) + \frac{L}{2} \norm{\overline{w}_{k+1} - \overline{w}_k}^2\\
& \leq f(\overline{w}_k) + \underbrace{ \nabla f(w_k)^\top (\overline{w}_{k+1} - \overline{w}_k)}_{A} + \underbrace{  \left( \nabla f(\overline{w}_k) -  \nabla f(w_k)\right)^\top (\overline{w}_{k+1} - \overline{w}_k)}_{B} + \frac{L}{2} \norm{\overline{w}_{k+1} - \overline{w}_k}
\end{split}
\eeq

\textbf{Term A}.
Using Lemma~\ref{lem:momentum}, we have that:
\beq
\begin{split}
\nabla f(w_k)^\top (\overline{w}_{k+1} - \overline{w}_k) & \leq \nabla f(w_k)^\top \left[\frac{\beta_1}{1 - \beta_1} \tilde{\theta}_{k-1} \left[ \eta_{k-1} \hat{v}_{k-1}^{-1/2} - \eta_{k} \hat{v}_{k}^{-1/2}\right] - \eta_{k} \hat{v}_{k}^{-1/2} \tilde{g}_k \right]\\
& \leq  \frac{\beta_1}{1 - \beta_1}  \norm{ \nabla f(w_k)} \norm{\eta_{k-1} \hat{v}_{k-1}^{-1/2} - \eta_{k} \hat{v}_{k}^{-1/2} } \norm{\tilde{\theta}_{k-1}} - \nabla f(w_k)^\top\eta_{k} \hat{v}_{k}^{-1/2} \tilde{g}_k 
\end{split}
\eeq
where the inequality is due to trivial inequality for positive diagonal matrix.
Using Lemma~\ref{lem:bound} and assumption H\ref{ass:guessbound} we obtain:
\beq\label{eq:termA1}
\begin{split}
\nabla f(w_k)^\top (\overline{w}_{k+1} - \overline{w}_k)  \leq  \frac{\beta_1 (1+\beta_1)}{1 - \beta_1} \major^2 \left[ \norm{\eta_{k-1} \hat{v}_{k-1}^{-1/2}} - \norm{\eta_{k} \hat{v}_{k}^{-1/2} }\right] - \nabla f(w_k)^\top\eta_{k} \hat{v}_{k}^{-1/2} \tilde{g}_k 
\end{split}
\eeq
where we have used the fact that $\eta_{k} \hat{v}_{k}^{-1/2} $ is a diagonal matrix such that $\eta_{k-1} \hat{v}_{k-1}^{-1/2} \succcurlyeq \eta_{k} \hat{v}_{k}^{-1/2}\succcurlyeq 0$ (decreasing stepsize and $\max$ operator).
Also note that:
\beq\label{eq:termA2}
\begin{split}
 - \nabla f(w_k)^\top\eta_{k} \hat{v}_{k}^{-1/2} \tilde{g}_k  &=  - \nabla f(w_k)^\top\eta_{k-1} \hat{v}_{k-1}^{-1/2} \bar{g}_k   -  \nabla f(w_k)^\top\left[ \eta_{k} \hat{v}_{k}^{-1/2} -\eta_{k} \hat{v}_{k}^{-1/2} \right] \bar{g}_k  \\ 
&   - \nabla f(w_k)^\top\eta_{k-1} \hat{v}_{k-1}^{-1/2} (\beta_1 g_{k-1} + m_{k+1})\\
 & \leq  - \nabla f(w_k)^\top\eta_{k-1} \hat{v}_{k-1}^{-1/2} \bar{g}_k +(1-a\beta_1)\major^2    \left[ \norm{\eta_{k-1} \hat{v}_{k-1}^{-1/2}} - \norm{\eta_{k} \hat{v}_{k}^{-1/2} }\right] \\
 &  - \nabla f(w_k)^\top\eta_{k} \hat{v}_{k}^{-1/2} (\beta_1 g_{k-1} + m_{k+1})
\end{split}
\eeq
using Lemma~\ref{lem:bound} on $\norm{g_k}$ and where that $\tilde{g}_k = \bar{g}_k  + \beta_1 g_{k-1} + m_{k+1} = g_k - \beta_1 m_k + \beta_1 g_{k-1} + m_{k+1} $.
Plugging \eqref{eq:termA2} into \eqref{eq:termA1} yields:
\beq\label{eq:termA}
\begin{split}
&\nabla f(w_k)^\top (\overline{w}_{k+1} - \overline{w}_k)\\
&  \leq   - \nabla f(w_k)^\top\eta_{k-1} \hat{v}_{k-1}^{-1/2} \bar{g}_k + \frac{1}{1 - \beta_1} (a\beta_1^2 -2 a \beta_1 + \beta 1)\major^2 \left[ \norm{\eta_{k-1} \hat{v}_{k-1}^{-1/2}} - \norm{\eta_{k} \hat{v}_{k}^{-1/2} }\right] \\
&  - \nabla f(w_k)^\top\eta_{k} \hat{v}_{k}^{-1/2} (\beta_1 g_{k-1} + m_{k+1})
\end{split}
\eeq

\textbf{Term B}.
By Cauchy-Schwarz (CS) inequality we have:
\beq\label{eq:termB1}
 \left( \nabla f(\overline{w}_k) -  \nabla f(w_k)\right)^\top (\overline{w}_{k+1} - \overline{w}_k) \leq  \norm{ \nabla f(\overline{w}_k) -  \nabla f(w_k)}  \norm{\overline{w}_{k+1} - \overline{w}_k}
 \eeq
 Using smoothness assumption H~\ref{ass:smooth}:
\beq\label{eq:termB2}
 \begin{split}
  \norm{ \nabla f(\overline{w}_k) -  \nabla f(w_k)} & \leq L \norm{ \overline{w}_k - w_k}\\
  & \leq L \frac{\beta_1}{1 - \beta_1} \norm{w_k - w_{k-1}}
 \end{split}
 \eeq
By Lemma~\ref{lem:momentum} we also have:
 \beq
 \begin{split}
\overline{w}_{k+1} - \overline{w}_k & = \frac{\beta_1}{1 - \beta_1} \tilde{\theta}_{k-1} \left[ \eta_{k-1} \hat{v}_{k-1}^{-1/2} - \eta_{k} \hat{v}_{k}^{-1/2}\right] - \eta_{k} \hat{v}_{k}^{-1/2} \tilde{g}_k \\
& = \frac{\beta_1}{1 - \beta_1} \tilde{\theta}_{k-1}\eta_{k-1} \hat{v}_{k-1}^{-1/2} \left[ I - (\eta_{k} \hat{v}_{k}^{-1/2}) (\eta_{k-1} \hat{v}_{k-1}^{-1/2})^{-1} \right] - \eta_{k} \hat{v}_{k}^{-1/2} \tilde{g}_k \\
& = \frac{\beta_1}{1 - \beta_1} \left[ I - (\eta_{k} \hat{v}_{k}^{-1/2}) (\eta_{k-1} \hat{v}_{k-1}^{-1/2})^{-1} \right] (w_{k-1} - w_k) - \eta_{k} \hat{v}_{k}^{-1/2} \tilde{g}_k
 \end{split}
 \eeq
 where the last equality is due to $ \tilde{\theta}_{k-1}\eta_{k-1} \hat{v}_{k-1}^{-1/2} = w_{k-1} - w_k$ by construction of $\tilde{\theta}_k$.
 Taking the norms on both sides, observing $\norm{ I - (\eta_{k} \hat{v}_{k}^{-1/2}) (\eta_{k-1} \hat{v}_{k-1}^{-1/2})^{-1}} \leq 1$ due to the decreasing stepsize and the construction of $\hat{v}_k$ and using CS inequality yield:
\beq\label{eq:termB3}
 \begin{split}
\norm{\overline{w}_{k+1} - \overline{w}_k} & \leq \frac{\beta_1}{1 - \beta_1} \norm{w_{k-1} - w_k} + \norm{\eta_{k} \hat{v}_{k}^{-1/2} \tilde{g}_k}
 \end{split}
 \eeq 
 We recall Young's inequality with a constant $\delta \in (0,1)$ as follows:
$$
\pscal{X}{Y} \leq \frac{1}{\delta} \norm{X}^2 + \delta \norm{Y}^2
$$

 Plugging \eqref{eq:termB2} and \eqref{eq:termB3} into \eqref{eq:termB1} returns:
 \beq
 \begin{split}
 \left( \nabla f(\overline{w}_k) -  \nabla f(w_k)\right)^\top (\overline{w}_{k+1} - \overline{w}_k) \leq & L \frac{\beta_1}{1 - \beta_1} \norm{\eta_{k} \hat{v}_{k}^{-1/2} \tilde{g}_k}  \norm{w_k - w_{k-1}}\\
 & +  L\left(\frac{\beta_1}{1 - \beta_1} \right)^2 \norm{w_{k-1} - w_k}^2
  \end{split}
 \eeq
 
Applying Young's inequality with $\delta \to \frac{\beta_1}{1 - \beta_1}$ on the product $ \norm{\eta_{k} \hat{v}_{k}^{-1/2} \tilde{g}_k}  \norm{w_k - w_{k-1}}$ yields:
 \beq\label{eq:termB}
 \left( \nabla f(\overline{w}_k) -  \nabla f(w_k)\right)^\top (\overline{w}_{k+1} - \overline{w}_k) \leq  L \norm{\eta_{k} \hat{v}_{k}^{-1/2} \tilde{g}_k}^2 +  2L\left(\frac{\beta_1}{1 - \beta_1} \right)^2 \norm{w_{k-1} - w_k}^2
 \eeq
 
 The last term $ \frac{L}{2} \norm{\overline{w}_{k+1} - \overline{w}_k}$ can be upper bounded using \eqref{eq:termB3}:
\beq\label{eq:term3} 
\begin{split}
 \frac{L}{2} \norm{\overline{w}_{k+1} - \overline{w}_k}^2 & \leq  \frac{L}{2} \left[ \frac{\beta_1}{1 - \beta_1} \norm{w_{k-1} - w_k} + \norm{\eta_{k} \hat{v}_{k}^{-1/2} \tilde{g}_k}\right]\\
 &  \leq L \norm{\eta_{k} \hat{v}_{k}^{-1/2} \tilde{g}_k}^2 + 2L  \left(\frac{\beta_1}{1 - \beta_1}\right)^2 \norm{w_{k-1} - w_k}^2 
\end{split}
\eeq


Plugging \eqref{eq:termA}, \eqref{eq:termB} and \eqref{eq:term3} into \eqref{eq:smoothness} and taking the expectations on both sides give:
\beq
\begin{split}
& \EE\left[f(\overline{w}_{k+1})  +   \frac{1}{1 - \beta_1}\tilde{\major}^2  \norm{\eta_{k} \hat{v}_{k}^{-1/2} }  - \left( f(\overline{w}_{k}) + \frac{1}{1 - \beta_1}\tilde{\major}^2 \norm{\eta_{k-1} \hat{v}_{k-1}^{-1/2}} \right)        \right] \\
& \leq \EE \left[ - \nabla f(w_k)^\top\eta_{k-1} \hat{v}_{k-1}^{-1/2} \bar{g}_k  - \nabla f(w_k)^\top\eta_{k} \hat{v}_{k}^{-1/2} ( \beta_1 g_{k-1} +m_{k+1})   \right]\\
& + \EE \left[ 2L \norm{\eta_{k} \hat{v}_{k}^{-1/2} \tilde{g}_k}^2 + 4L  \left(\frac{\beta_1}{1 - \beta_1}\right)^2 \norm{w_{k-1} - w_k}^2  \right]
\end{split}
\eeq
where $ \tilde{\major}^2 = (a\beta_1^2 -2 a \beta_1 + \beta 1)\major^2$.
Note that the expectation of $ \tilde{g}_k $ conditioned on the filtration $\mathcal{F}_{k}$ reads as follows
\beq\label{eq:expectationtildegrad}
\begin{split}
\EE\left[    \nabla f(w_k)^\top \bar{g}_k  \right] & = \EE\left[  \nabla  f(w_k)^\top (g_k  - \beta_1 m_{k})  \right]\\
& = (1-a\beta_1) \| \nabla f(w_k) \|^2
\end{split}
\eeq
Summing from $k=1$ to $k=K$ leads to 
\beq\label{eq:bound1}
\begin{split}
& \frac{1}{\major} \sum_{k=1}^{K_{\sf max}} \left( (1 - a\beta_1)   \eta_{k-1} + (\beta_1 + a)   \eta_{k} \right) \norm{\nabla f(w_k)}^2 \leq\\
&  \EE\left[  f(\overline{w}_{1}) + \frac{1}{1 - \beta_1}\tilde{\major}^2 \norm{\eta_{0} \hat{v}_{0}^{-1/2}}    - \left(f(\overline{w}_{K_{\sf max}+1})  +   \frac{1}{1 - \beta_1}\tilde{\major}^2  \norm{\eta_{K_{\sf max}} \hat{v}_{K_{\sf max}}^{-1/2} } \right)      \right]\\
& +2L  \sum_{k=1}^{K_{\sf max}}  \EE \left[  \norm{\eta_{k} \hat{v}_{k}^{-1/2} \tilde{g}_k}^2 \right] + 4L \left(\frac{\beta_1}{1 - \beta_1}\right)^2 \sum_{k=1}^{K_{\sf max}}  \EE \left[  \norm{w_{k-1} - w_k}^2  \right]\\
& \leq  \EE\left[  \Delta f  + \frac{1}{1 - \beta_1}\tilde{\major}^2 \norm{\eta_{0} \hat{v}_{0}^{-1/2}}    \right] +2L  \sum_{k=1}^{K_{\sf max}}  \EE \left[  \norm{\eta_{k} \hat{v}_{k}^{-1/2} \tilde{g}_k}^2 \right] + 4L \left(\frac{\beta_1}{1 - \beta_1}\right)^2 \sum_{k=1}^{K_{\sf max}}  \EE \left[  \norm{w_{k-1} - w_k}^2  \right]\\
\end{split}
\eeq
where $ \Delta f = f(\overline{w}_{1}) - f(\overline{w}_{K_{\sf max}+1})$.
We note that by definition of $\hat{v}_k$, and a constant learning rate $\eta_k$, we have
\beq
\begin{split}
\norm{w_{k-1} - w_k}^2 & =\norm{\eta_{k-1} \hat{v}_{k-1}^{-1/2} (\theta_{k-1} + h_{k})}^2 \\
& =\norm{\eta_{k-1} \hat{v}_{k-1}^{-1/2} (\theta_{k-1} + \beta_{1} \theta_{k-2} + (1-\beta_{1}) m_{k})}^2\\
& \leq \norm{\eta_{k-1} \hat{v}_{k-1}^{-1/2}\theta_{k-1}}^2 + \norm{\eta_{k-2} \hat{v}_{k-2}^{-1/2} \beta_{1} \theta_{k-2}}^2 + (1-\beta_{1})^2 \norm{\eta_{k-1} \hat{v}_{k-1}^{-1/2}m_{k}}^2
\end{split}
\eeq
Using Lemma~\ref{lem:squarev} we have
\beq
\begin{split}
& \sum_{k=1}^{K_{\sf max}} \EE \left[  \norm{w_{k-1} - w_k}^2  \right]\\ 
& \leq (1 + \beta_1^2) \frac{\eta^{2} d K_{\sf max} (1- \beta_1)}{(1 - \beta_2)(1-\gamma)} + (1 - \beta_1)^2 \sum_{k=1}^{K_{\sf max}} \EE \left[ \norm{\eta_{k-1} \hat{v}_{k-1}^{-1/2}m_{k}}^2\right]
\end{split}
\eeq
And thus, setting the learning rate to a constant value $\eta$ and injecting in \eqref{eq:bound1} yields:
\beq
\begin{split}
&\EE\left[\|\nabla f(w_K)\|^2\right] = \frac{ 1 }{\sum_{j=1}^{K_{\sf max}} \eta_j}  \sum_{k=1}^{K_{\sf max}} \eta_{k} \norm{\nabla f(w_k)}^2 \\
& \leq \frac{\major}{(1 - a\beta_1) + (\beta_1 + a)}  \frac{ 1 }{\sum_{j=1}^{K_{\sf max}} \eta_j}   \EE\left[  \Delta f  + \frac{1}{1 - \beta_1}\tilde{\major}^2 \norm{\eta_{0} \hat{v}_{0}^{-1/2}}    \right]\\
& +  \frac{ 4L \left(\frac{\beta_1}{1 - \beta_1}\right)^2 \major}{(1 - a\beta_1) + (\beta_1 + a)}  \frac{ 1 }{\sum_{j=1}^{K_{\sf max}} \eta_j}  (1 + \beta_1^2) \frac{\eta^{2} d K_{\sf max} (1- \beta_1)}{(1 - \beta_2)(1-\gamma)}\\
& + \frac{\major}{(1 - a\beta_1) + (\beta_1 + a)}  \frac{ 1 }{\sum_{j=1}^{K_{\sf max}} \eta_j} (1 - \beta_1)^2 \sum_{k=1}^{K_{\sf max}} \EE \left[ \norm{\eta_{k-1} \hat{v}_{k-1}^{-1/2}m_{k}}^2\right]\\
& +  \frac{2L\major}{(1 - a\beta_1) + (\beta_1 + a)}  \frac{ 1 }{\sum_{j=1}^{K_{\sf max}} \eta_j}   \sum_{k=1}^{K_{\sf max}}  \EE \left[  \norm{\eta_{k} \hat{v}_{k}^{-1/2} \tilde{g}_k}^2 \right] 
\end{split}
\eeq
where $K$ is a random termination number distributed according \eqref{eq:random}.
Setting the stepsize to $\eta = \frac{1}{\sqrt{d K_{\sf max}}}$ yields :
\beq
\begin{split}
&\EE\left[\|\nabla f(w_K)\|^2\right]\\
& \leq C_1 \sqrt{\frac{d}{K_{\sf max}}} + C_2 \frac{1}{K_{\sf max}}\\
& + D_1 \frac{\eta}{K_{\sf max}} \sum_{k=1}^{K_{\sf max}} \EE \left[ \norm{ \hat{v}_{k-1}^{-1/2}m_{k}}^2\right] + D_2 \frac{\eta}{K_{\sf max}} \sum_{k=1}^{K_{\sf max}} \EE \left[ \norm{ \hat{v}_{k-1}^{-1/2} \tilde{g}_{k}}^2\right] 
\end{split}
\eeq
where
\beq
\begin{split}
&C_1 = \frac{\major}{(1 - a\beta_1) + (\beta_1 + a)}  \Delta f + \frac{ 4L \left(\frac{\beta_1}{1 - \beta_1}\right)^2 \major}{(1 - a\beta_1) + (\beta_1 + a)} \frac{(1 + \beta_1^2) (1- \beta_1)}{(1 - \beta_2)(1-\gamma)} \\
&C_2 =\frac{\major}{(1 - \beta_1) \left((1 - a\beta_1) + (\beta_1 + a)\right)} \tilde{\major}^2   \EE\left[ \norm{\hat{v}_{0}^{-1/2}}    \right]
\end{split}
\eeq

\textbf{Simple case as in \citep{zhou2018convergence}:} if $\beta_1 = 0$ then $ \tilde{g}_{k} = g_k + m_{k+1}$ and $g_k = \theta_k$. Also using Lemma~\ref{lem:squarev} we have that:
\beq
\sum_{k=1}^{K_{\sf max}} \eta_{k}^{2} \EE \left[\left\|\hat{v}_{k}^{-1/2} g_{k}\right\|_{2}^{2}\right] \leq  \frac{\eta^{2} d K_{\sf max}}{(1 - \beta_2)} 
\eeq
which leads to the final bound:
\beq
\begin{split}
&\EE\left[\|\nabla f(w_K)\|^2\right]\\
& \leq \tilde{C}_1 \sqrt{\frac{d}{K_{\sf max}}} + \tilde{C}_2 \frac{1}{K_{\sf max}}
\end{split}
\eeq
where
\beq
\begin{split}
&\tilde{C}_1 = C_1 +  \frac{\major}{(1 - a\beta_1) + (\beta_1 + a)} \left[ \frac{a(1 - \beta_1)^2}{1-\beta_2} + 2L \frac{1}{1-\beta_2}  \right]\\
&\tilde{C}_2 = C_2 =\frac{\major}{(1 - \beta_1) \left((1 - a\beta_1) + (\beta_1 + a)\right)} \tilde{\major}^2   \EE\left[ \norm{\hat{v}_{0}^{-1/2}}    \right]
\end{split}
\eeq
\end{proof}

\section{Proof of Lemma~\ref{lem:dnnh2} (Boundedness of the iterates)}\label{app:lemdnnh2}
\begin{Lemma*}
Given the multilayer model \eqref{eq:dnnmodel}, assume the boundedness of the input data and of the loss function, \ie for any $\xi \in \rset^l$ and $y \in \rset$ there is a constant $T >0$ such that:
\beq\label{eq:mildassumptions}
\norm{\xi} \leq 1 \quad \textrm{a.s.} \quad \textrm{and} |\mathcal{L}'(\cdot, y)| \leq T
\eeq
where $\mathcal{L}'(\cdot, y)$ denotes its derivative \wrt the parameter. Then for each layer $\ell \in [1,L]$, there exist a constant $A_{(\ell)}$ such that:
$$
\norm{w^{(\ell)}} \leq A_{(\ell)}
$$
\end{Lemma*}
\begin{proof}
Recall that for any layer index $\ell \in [1, L]$ we denote the output of layer $\ell$ by $h^{(\ell)}(w,\xi)$:
$$
h^{(\ell)}(w,\xi) = \sigma\left(w^{(\ell)} \sigma\left(w^{(\ell-1)} \ldots \sigma\left(w^{(1)} \xi \right)\right)\right)
$$
Given the sigmoid assumption we have $\norm{h^{(\ell)}(w,\xi)} \leq 1$ for any $\ell \in [1,L]$ and any $(w, \xi) \in \rset^d \times \rset^l$.
Observe that at the last layer $L$:
\beq\label{eq:boundderivativeloss}
\begin{split}
\norm{\nabla_{w^{(L)}}  \mathcal{L}(\textsf{MLN}( w, \xi), y)} & =  \norm{\mathcal{L}'(\textsf{MLN}( w, \xi), y)\nabla_{w^{(L)}}\textsf{MLN}( w, \xi)}\\
&  = \norm{\mathcal{L}'(\textsf{MLN}( w, \xi), y)\sigma'(w^{(L)} h^{(L-1)}(w,\xi))h^{(L-1)}(w,\xi)}\\
& \leq \frac{T}{4}
\end{split}
\eeq
where the last equality is due to mild assumptions \eqref{eq:mildassumptions} and to the fact that the norm of the derivative of the sigmoid function is upperbounded by $1/4$.

From Algorithm~\ref{alg:optamsgrad}, with $\beta_1 = 0$ we have for iteration index $k >0$:
\beq
\begin{split}
\norm{w_k - w_{k-1}} & = \norm{-\eta_k \hat{v}_k^{-1/2} (\theta_k + h_{k+1})}\\
&  = \norm{\eta_k \hat{v}_k^{-1/2} (g_k + m_{k+1})}\\
& \leq \hat{\eta} \norm{\hat{v}_k^{-1/2} g_k} + \hat{\eta} a \norm{\hat{v}_k^{-1/2} g_{k+1}}
\end{split}
\eeq
where $\hat{\eta} = \max \limits_{k >0} \eta_k$.
For any dimension $p \in [1,d]$, using assumption H~\ref{ass:guessbound}, we note that 
$$\sqrt{\hat{v}_{k,p}} \geq \sqrt{1-\beta_2} g_{k,p} \quad \textrm{and} \quad m_{k+1} \leq  a \norm{g_{k+1}}$$ .
Thus:
\beq
\begin{split}
\norm{w_k - w_{k-1}} & \leq \hat{\eta} \left( \norm{\hat{v}_k^{-1/2} g_k} +  a \norm{\hat{v}_k^{-1/2} g_{k+1}} \right) \\
& \leq \hat{\eta} \frac{a +1}{\sqrt{1 - \beta_2}} 
\end{split}
\eeq
In short there exist a constant $B$ such that $\norm{w_k - w_{k-1}} \leq B$.

\textbf{Proof by induction:} As in \citep{defossez2020convergence}, we will prove the containment of the weights by induction.
Suppose an iteration index $K$ and a coordinate $i$ of the last layer $L$ such that $w^{(L)}_{K, i} \geq \frac{T}{4\lambda} + B$.
Using \eqref{eq:boundderivativeloss}, we have
$$
\nabla_i f(w^{(L)}_K\geq - \frac{T}{4} + \lambda \frac{T}{\lambda4} \geq 0
$$
where $f(\cdot)$ is defined by \eqref{eq:lossmln} and is the loss of our MLN.
This last equation yields $\theta^{(L)}_{K,i} \geq 0$ (given the algorithm and $\beta_1 = 0$) and using the fact that $\norm{w_k - w_{k-1}} \leq B$ we have
\beq\label{eq:decrease}
0 \leq w^{(L)}_{K-1,i} - B \leq w^{(L)}_{K,i} \leq w^{(L)}_{K-1,i}
\eeq
which means that $| w^{(L)}_{K,i}| \leq w^{(L)}_{K-1,i}$.
So if the first assumption of that induction reasoning holds, \ie $w^{(L)}_{K-1, i} \geq \frac{T}{4\lambda} + B$, then the next iterates $w^{(L)}_{K, i}$ decreases, see \eqref{eq:decrease} and go below $\frac{T}{4\lambda} + B$. This yields that for any iteration index $k >0$ we have 
$$
w^{(L)}_{K, i} \leq \frac{T}{4\lambda} + 2B
$$
since $B$ is the biggest jump an iterate can do since $\norm{w_k - w_{k-1}} \leq B$.
Likewise we can end up showing that 
$$
|w^{(L)}_{K, i}| \leq \frac{T}{4\lambda} + 2B
$$
meaning that the weights of the last layer at any iteration is bounded in some matrix norm.

Now that we have shown this boundedness property for the last layer $L$, we will do the same for the previous layers and conclude the verification of assumption H~\ref{ass:boundedparam} by induction.

For any layer $\ell \in [1, L-1]$, we have:
\beq\label{eq:gradientatell}
\nabla_{w^{(\ell)}}  \mathcal{L}(\textsf{MLN}( w, \xi), y)  =  \mathcal{L}'(\textsf{MLN}( w, \xi), y) \left(\prod_{j=1}^{\ell+1} \sigma'\left(w^{(j)} h^{(j-1)}(w,\xi) \right) \right) h^{(\ell-1)}(w,\xi) 
\eeq
This last quantity is bounded as long as we can prove that for any layer $\ell$ the weights $w^{(\ell)}$ are bounded in some matrix norm as $\norm{w^{(\ell)}}_{F} \leq F_\ell$ with the Frobenius norm.
Suppose we have shown $\norm{w^{(r)}}_{F} \leq F_r$ for any layer $r > \ell$. 
Then having this gradient \eqref{eq:gradientatell} bounded we can use the same lines of proof for the last layer $L$ and show that the norm of the weights at the selected layer $\ell$ satisfy
$$
\norm{w^{(\ell)}} \leq \frac{T \prod_{k > \ell} F_k}{4^{L-\ell+1}} + 2B
$$
Showing that the weights of the previous layers $\ell \in [1, L-1]$ as well as for the last layer $L$ of our fully connected feed forward neural network are bounded at each iteration, leads by induction, to the boundedness (at each iteration) assumption we want to check.
\end{proof}

%-----------------------------------------------------------------------------
%\vspace{0.4cm}

\end{document} 