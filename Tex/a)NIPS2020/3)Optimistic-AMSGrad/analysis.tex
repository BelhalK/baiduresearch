\documentclass[11pt]{article}
%\usepackage{fullpage,graphicx,algorithm,algorithmic,bm,amsmath,amsthm,amssymb,color,hyperref,cite,natbib}

% if you need to pass options to natbib, use, e.g.:
%\PassOptionsToPackage{numbers}{natbib}
\usepackage{natbib,fullpage}
\usepackage{bm,amsmath,amsthm,amssymb,multicol,algorithmic,algorithm,enumitem}
\usepackage{wrapfig,lipsum}
\usepackage[textwidth=1cm,textsize=footnotesize]{todonotes}

% ready for submission
\usepackage{neurips_2020}

\usepackage[colorlinks=true,
linkcolor=red,
urlcolor=blue,
citecolor=blue]{hyperref}
\usepackage{hyperref}
\usepackage{cleveref}

\setlength{\parskip}{.2cm}

\newtheorem{Fact}{Fact}
\newtheorem{Lemma}{Lemma}
\newtheorem{Prop}{Proposition}
\newtheorem{Theorem}{Theorem}
\newtheorem{Def}{Definition}
\newtheorem{Corollary}{Corollary}
\newtheorem{Conjecture}{Conjecture}
\newtheorem{Property}{Property}
\newtheorem{Observation}{Observation}
%\theorembodyfont{\rmfamily}
\newtheorem{Exa}{Example}
\newtheorem{assumption}{H\!\!}
\newtheorem{assumptionA}{S\!\!}
\newtheorem{assumptionL}{L\!\!}
\newtheorem{Remark}{Remark}
\newtheorem*{Lemma*}{Lemma}
\newtheorem*{Theorem*}{Theorem}
 \makeatletter
\renewenvironment{proof}[1][\proofname]{%
   \par\pushQED{\qed}\normalfont%
   \topsep6\p@\@plus6\p@\relax
   \trivlist\item[\hskip\labelsep\bfseries#1]%
   \ignorespaces
}{%
   \popQED\endtrivlist\@endpefalse
}
\makeatother

%%%%%%%%%%% Stuffs for Tikz %%%%%%%%%%%%%%%%%%
\usepackage{pgfplots}
\usepackage{xargs}
\usepackage{stmaryrd}
\usetikzlibrary{arrows,shapes,calc,tikzmark,backgrounds,matrix,decorations.markings}
\usepgfplotslibrary{fillbetween}

\pgfplotsset{compat=1.3}

\usepackage{relsize}
\tikzset{fontscale/.style = {font=\relsize{#1}}
    }

\definecolor{lavander}{cmyk}{0,0.48,0,0}
\definecolor{violet}{cmyk}{0.79,0.88,0,0}
\definecolor{burntorange}{cmyk}{0,0.52,1,0}

\def\lav{lavander!90}
\def\oran{orange!30}

\definecolor{asuorange}{rgb}{1,0.699,0.0625}
\definecolor{asured}{rgb}{0.598,0,0.199}
\definecolor{asuborder}{rgb}{0.953,0.484,0}
\definecolor{asugrey}{rgb}{0.309,0.332,0.340}
\definecolor{asublue}{rgb}{0,0.555,0.836}
\definecolor{asugold}{rgb}{1,0.777,0.008}

%%%%%%%%%%%%%%%%%%%%%%%%%%%%%%%%%%%%%


\usepackage{shortcuts_OPT}

%\renewcommand{\textwidth}{5.5in}

% Here's the definition of Sb, stolen from amstex
    \makeatletter
    \def\multilimits@{\bgroup
  \Let@
  \restore@math@cr
  \default@tag
 \baselineskip\fontdimen10 \scriptfont\tw@
 \advance\baselineskip\fontdimen12 \scriptfont\tw@
 \lineskip\thr@@\fontdimen8 \scriptfont\thr@@
 \lineskiplimit\lineskip
 \vbox\bgroup\ialign\bgroup\hfil$\m@th\scriptstyle{##}$\hfil\crcr}
    \def\Sb{_\multilimits@}
    \def\endSb{\crcr\egroup\egroup\egroup}
\makeatother

\newtheoremstyle{k}         %name
    {\baselineskip}{2\topsep}      %space above and below
    {\rm}                   %Body font
    {0pt}{\bfseries}  %Heading indent and font
    {}                      %after heading
    { }                      %head after space
    {\thmname{#1}\thmnumber{#2}.}

\theoremstyle{k}
\newtheorem{q}{Q}
\parindent=0pt

%\newcommand{\eric}[1]{\todo[color=red!20]{{\bf EM:} #1}}
%\newcommand{\erici}[1]{\todo[color=red!20,inline]{{\bf EM:} #1}}
%\newcommand{\belhal}[1]{\todo[color=green!20]{{\bf BK:} #1}}
%\newcommand{\belhali}[1]{\todo[color=green!20,inline]{{\bf BK:} #1}}
%\newcommand{\toco}[1]{\todo[color=yellow!20]{{\bf To:} #1}}



\makeatletter
\DeclareRobustCommand*\cal{\@fontswitch\relax\mathcal}
\makeatother

\begin{document}
\title{Optimistic Acceleration of AMSGrad for Nonconvex Optimization.}
%\author{}
\date{\today}

\maketitle

%\begin{abstract}
%\end{abstract}

\section{Nonconvex Analysis}
We tackle the following classical optimization problem:
\beq\label{eq:minproblem}
\min \limits_{w \in \Theta} f(w) \eqdef \EE[ f(w, \xi)]
\eeq
where $\xi$ is some random noise and only noisy versions of the objective function are accessible in this work.
The objective function $f(w)$ is (potentially) nonconvex and has Lipschitz gradients.

\textbf{Optimistic Algorithm}
We present here the algorithm studied in this paper to tackle problem \eqref{eq:minproblem}.
Set the terminating iteration number, $K \in \{0,\dots,K_{\sf max}-1\}$, as a discrete r.v.~with:
\beq \label{eq:random}
   P( K = k ) = \frac{ \eta_{k} }{\sum_{f=0}^{K_{\sf max}-1} \eta_f}.
\eeq
where $K_{\sf max}$ $\leftarrow$ is the maximum number of iteration.
The random termination number \eqref{eq:random} is inspired by \citep{ghadimi2013stochastic} which enables one to show non-asymptotic convergence to stationary point for non-convex optimization. 
Consider constants $(\beta_1, \beta_2) \in [0,1]$, a sequence of decreasing stepsizes $\{\eta_k\}_{k>0}$, Algorithm~\ref{alg:opt-ams} introduces the new optimistic AMSGrad method.

\begin{algorithm}[H]
\caption{OPTIMISTIC-AMSGRAD}\label{alg:opt-ams}
  \begin{algorithmic}[1]
  \STATE \textbf{Input:} Parameters $\beta_{1}, \beta_{2}, \epsilon, \eta_{k}$
  \STATE \textbf{Init.:} $w_{1}=w_{-1 / 2} \in \mathcal{K} \subseteq \mathbb{R}^{d} \text { and } v_{0}=\epsilon \mathbf{1} \in \mathbb{R}^{d}$
  \FOR {$k=0,1,2,\dots, K$}
  \STATE Get mini-batch stochastic gradient $g_{k}$ at $w_{k}$
   \STATE $\theta_{k}=\beta_{1} \theta_{k-1}+\left(1-\beta_{1}\right) g_{k}$
   \STATE $v_{k}=\beta_{2} v_{k-1}+\left(1-\beta_{2}\right) g_{k}^{2}$
   \STATE $ \hat{v}_{k}=\max \left(\hat{v}_{k-1}, v_{k}\right)$
   \STATE $ w_{k+\frac{1}{2}}=\Pi_{\mathcal{K}}\left[w_k-\eta_{k} \frac{\theta_{k}}{\sqrt{\hat{v}_{k}}}\right]$
   \STATE $ w_{k+1}=\Pi_{\mathcal{K}}\left[w_{k+\frac{1}{2}}-\eta_{k} \frac{h_{k+1}}{\sqrt{v}_{k}}\right]$
   \STATE $ \quad \text{where} \quad h_{k+1}:=\beta_{1} \theta_{k-1} + (1-\beta_{1}) m_{k+1}$
      \STATE $ \quad\quad \text{and} \quad m_{k+1}$ is a guess of $g_{k+1}$
\ENDFOR
\STATE \textbf{Return}: $w_{K+1}$.
  \end{algorithmic}
\end{algorithm}\vspace{.1cm}
The final update at iteration $k$ can be summarized as:
\beq\label{eq:finalupdate}
w_{k+1}=w_{k}-\eta_{k} \frac{\theta_{k}}{\sqrt{\hat{v}_{k}}}-\eta_{k} \frac{h_{k+1}}{\sqrt{v}_{k}}
\eeq



We make the following assumptions:
\begin{assumption}\label{ass:nonconv}
The loss function $f(w)$ is nonconvex \wrt the parameter $w$.
\end{assumption}
\begin{assumption}\label{ass:smooth}
The function $f(w)$ is $L$-smooth w.r.t. the parameter w.
There exist some constant $L > 0$ such that for $(w, \vartheta) \in \Theta^2$:
\beq
f(w) - f(\vartheta) - \nabla f(\vartheta)^\top(w - \vartheta) \leq \frac{L}{2} \norm{w - \vartheta}^2\eqsp.
\eeq
\end{assumption}
\begin{assumption}\label{ass:guessbound}
There exists a constant $a >0$ such that for any $k >0$:
$$
\norm{m_{k+1}} \leq a \norm{g_{k+1}}
$$
\end{assumption}
Classically (see \citep{ghadimi2013stochastic}) in nonconvex optimization, we make an assumption on the magnitude of the gradient:
\begin{assumption}\label{ass:bounded}
There exists a constant $\major >0$ such that 
$$
\|\nabla f(w, \xi)\| < \major \quad \textrm{for any $w$ and $\xi$}
$$
\end{assumption}

We begin with some auxiliary Lemmas important for the analysis. 
The first one ensures bounded norms of various quantities of interests (boiling down from the classical stochastic gradient boundedness assumption):
\begin{Lemma}\label{lem:bound}
Assume assumption H~\ref{ass:bounded}, then the quantities defined in Algorithm~\ref{alg:opt-ams} satisfy for any $w \in \Theta$ and $k>0$:
$$ \|\nabla f(w)\| < \major ,~~~\|\theta_k \| < \major^2 ,~~~\|\hat{v}_k\| < \major \eqsp.$$
\end{Lemma}
Then, following \citep{yan2018unified} and their study of the SGD with Momentum (not AMSGrad but simple momentum) we denote for any $k >0$:
\beq\label{eq:deftilde}
\overline{w}_k = w_k + \frac{\beta_1}{1 - \beta_1} (w_k - w_{k-1}) = \frac{1}{1 - \beta_1} w_k -  \frac{\beta_1}{1 - \beta_1} w_{k-1} \eqsp,
\eeq
and derive an important Lemma:
\begin{Lemma}\label{lem:momentum}
Assume a strictly positive and non increasing sequence of stepsizes $\{\eta_k \}_{k>0}$, $\beta_ \in [0,1]$, then the following holds:
\beq
\overline{w}_{k+1} - \overline{w}_k = \frac{\beta_1}{1 - \beta_1} \tilde{\theta}_{k-1} \left[ \eta_{k-1} v_{k-1}^{-1/2} - \eta_{k} v_{k}^{-1/2}\right] - \eta_{k} v_{k}^{-1/2} \tilde{g}_k \eqsp,
\eeq
where $\tilde{\theta}_{k} = \theta_k + \beta_1 \theta_{k-1} + (1 - \beta_1) m_{k+1} $ and $\tilde{g}_k = g_k - \beta_1 g_{k-1}$ 
\end{Lemma}
\begin{proof}
By definition \eqref{eq:deftilde} and using the Algorithm updates, we have:
\beq
\begin{split}
\overline{w}_{k+1} - \overline{w}_k  &= \frac{1}{1 - \beta_1} ( w_{k+1} - w_k)  -  \frac{\beta_1}{1 - \beta_1}( w_{k} - w_{k-1})\\
& = - \frac{1}{1 - \beta_1} \eta_{k} v_{k}^{-1/2} (\theta_k + h_{k+1})  +  \frac{\beta_1}{1 - \beta_1}\eta_{k-1} v_{k-1}^{-1/2} (\theta_{k-1} + h_{k})\\
& = - \frac{1}{1 - \beta_1}  \eta_{k} v_{k}^{-1/2} (\theta_k + \beta_1 \theta_{k-1}) -\frac{1}{1 - \beta_1}  \eta_{k} v_{k}^{-1/2} (1- \beta_1) m_{k+1}\\
& + \frac{\beta_1}{1 - \beta_1} \eta_{k-1} v_{k-1}^{-1/2} (\theta_{k-1} + \beta_1 \theta_{k-2}) + \frac{\beta_1}{1 - \beta_1}  \eta_{k-1} v_{k-1}^{-1/2} (1- \beta_1) m_{k}
\end{split}
\eeq
Denote $\tilde{\theta}_k = \theta_k + \beta_1 \theta_{k-1}$ we notice that $\tilde{\theta}_k = \beta_1 \tilde{\theta}_{k-1} + (1 - \beta_1) (g_k + \beta_1 g_{k-1})$.
\beq
\begin{split}
\overline{w}_{k+1} - \overline{w}_k  & \leq
\end{split}
\eeq
\end{proof}

We now formulate the main result of our paper giving an finite-time upper bound of the quantity $\EE\left[\|\nabla f(w_K)\|^2\right]$ where K is a random termination number distributed according to \ref{eq:random}, see \citep{ghadimi2013stochastic}.

\begin{Theorem}
Assume H~\ref{ass:smooth}-H~\ref{ass:bounded}, $(\beta_1, \beta_2) \in [0,1]$ and a sequence of decreasing stepsizes $\{\eta_k\}_{k>0}$, then the following result holds:
\beq
\EE\left[\|\nabla f(w_K)\|^2\right] \leq tocomplete
\eeq
\end{Theorem}
\begin{proof}
Using H~\ref{ass:smooth} and the iterate $\overline{w}_k$ we have:
\beq\label{eq:smoothness}
\begin{split}
f(\overline{w}_{k+1}) & \leq  f(\overline{w}_k) + \nabla f(\overline{w}_k)^\top (\overline{w}_{k+1} - \overline{w}_k) + \frac{L}{2} \norm{\overline{w}_{k+1} - \overline{w}_k}^2\\
& \leq f(\overline{w}_k) + \underbrace{ \nabla f(w_k)^\top (\overline{w}_{k+1} - \overline{w}_k)}_{A} + \underbrace{  \left( \nabla f(\overline{w}_k) -  \nabla f(w_k)\right)^\top (\overline{w}_{k+1} - \overline{w}_k)}_{B} + \frac{L}{2} \norm{\overline{w}_{k+1} - \overline{w}_k}
\end{split}
\eeq

\textbf{Term $A$}.
Using Lemma~\ref{lem:momentum}, we have that:
\beq
\begin{split}
\nabla f(w_k)^\top (\overline{w}_{k+1} - \overline{w}_k) & = \nabla f(w_k)^\top \left[\frac{\beta_1}{1 - \beta_1} \tilde{\theta}_{k-1} \left[ \eta_{k-1} v_{k-1}^{-1/2} - \eta_{k} v_{k}^{-1/2}\right] - \eta_{k} v_{k}^{-1/2} \tilde{g}_k \right]\\
& \leq  \frac{\beta_1}{1 - \beta_1}  \norm{ \nabla f(w_k)} \norm{\eta_{k-1} v_{k-1}^{-1/2} - \eta_{k} v_{k}^{-1/2} } \norm{\tilde{\theta}_{k-1}} - \nabla f(w_k)^\top\eta_{k} v_{k}^{-1/2} \tilde{g}_k 
\end{split}
\eeq
where the inequality is due to trivial inequality for positive diagonal matrix.
Using Lemma~\ref{lem:bound} and assumption H\ref{ass:guessbound} we obtain:
\beq
\begin{split}
\nabla f(w_k)^\top (\overline{w}_{k+1} - \overline{w}_k)  \leq  \frac{\beta_1 (a+2)}{1 - \beta_1} \major^2 \left[ \norm{\eta_{k-1} v_{k-1}^{-1/2}} - \norm{\eta_{k} v_{k}^{-1/2} }\right] - \nabla f(w_k)^\top\eta_{k} v_{k}^{-1/2} \tilde{g}_k 
\end{split}
\eeq
where we have used the fact that $\eta_{k} v_{k}^{-1/2} $ is a diagonal matrix such that $\eta_{k-1} v_{k-1}^{-1/2} \succcurlyeq \eta_{k} v_{k}^{-1/2}\succcurlyeq 0$ (decreasing stepsize and $\max$ operator).
Also note that:
\beq
\begin{split}
 - \nabla f(w_k)^\top\eta_{k} v_{k}^{-1/2} \tilde{g}_k  &=  - \nabla f(w_k)^\top\eta_{k-1} v_{k-1}^{-1/2} \tilde{g}_k   -  - \nabla f(w_k)^\top\left[ \eta_{k} v_{k}^{-1/2} -\eta_{k-1} v_{k-1}^{-1/2} \right] \tilde{g}_k  \\ 
 & \leq  - \nabla f(w_k)^\top\eta_{k-1} v_{k-1}^{-1/2} \tilde{g}_k +(1-\beta_1)\major^2    \left[ \norm{\eta_{k-1} v_{k-1}^{-1/2}} - \norm{\eta_{k} v_{k}^{-1/2} }\right] 
\end{split}
\eeq
using Lemma~\ref{lem:bound} on $\norm{g_k}$ and recalling that $\tilde{g}_k = g_k -\beta_1 g_{k-1}$ 
\end{proof}


\section{Containment of the iterates for a DNN}


\newpage

\bibliographystyle{abbrvnat}
\bibliography{referencesB}



%-----------------------------------------------------------------------------
%\vspace{0.4cm}

\end{document} 