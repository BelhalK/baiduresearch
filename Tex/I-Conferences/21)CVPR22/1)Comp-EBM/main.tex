% CVPR 2022 Paper Template
% based on the CVPR template provided by Ming-Ming Cheng (https://github.com/MCG-NKU/CVPR_Template)
% modified and extended by Stefan Roth (stefan.roth@NOSPAMtu-darmstadt.de)
% modified and extended by Andrea Tagliasacchi (taiya.github.io)

\documentclass[10pt,twocolumn,letterpaper]{article}

%%%%%%%%% PAPER TYPE  - PLEASE UPDATE FOR FINAL VERSION
% \usepackage[review]{cvpr}      % To produce the REVIEW version
% \usepackage{cvpr}              % To produce the CAMERA-READY version
\usepackage[pagenumbers]{cvpr} % To force page numbers, e.g. for an arXiv version

% Include other packages here, before hyperref.
\usepackage{graphicx}
\usepackage{amsmath}
\usepackage{amssymb}
\usepackage{booktabs}

% It is strongly recommended to use hyperref, especially for the review version.
% hyperref with option pagebackref eases the reviewers' job.
% Please disable hyperref *only* if you encounter grave issues, e.g. with the
% file validation for the camera-ready version.
%
% If you comment hyperref and then uncomment it, you should delete
% ReviewTempalte.aux before re-running LaTeX.
% (Or just hit 'q' on the first LaTeX run, let it finish, and you
%  should be clear).
\usepackage[pagebackref,breaklinks,colorlinks]{hyperref}


% Support for easy cross-referencing
\usepackage[capitalize]{cleveref}
\crefname{section}{Sec.}{Secs.}
\Crefname{section}{Section}{Sections}
\Crefname{table}{Table}{Tables}
\crefname{table}{Tab.}{Tabs.}


%%%%%%%%% PAPER ID  - PLEASE UPDATE
\def\cvprPaperID{*****} % *** Enter the CVPR Paper ID here
\def\confName{CVPR}
\def\confYear{2022}

\input{preamble}
\begin{document}
\input{sec/0_metadata}
\maketitle
\input{sec/0_abstract}
\section{Introduction}
\label{sec:intro}
% 
\lorem{1}
\begin{figure}[t]
\begin{center}
% \begin{overpic} 
% [width=\linewidth]
% {example-image-a}
% \end{overpic}
\includegraphics[width=\linewidth]{example-image-golden}
\end{center}
\caption{
% 
\textbf{Teaser -- }
Some nice summarizing image.
}
\label{fig:teaser}
\end{figure}
\lorem{2}

\todo{
\paragraph{Contributions}
\begin{itemize}[leftmargin=*]
\setlength\itemsep{-.3em}
\item Neque porro quisquam est qui dolorem ipsum quia dolor sit amet, consectetur, adipisci velit...
\item Neque porro quisquam est qui dolorem ipsum quia dolor sit amet, consectetur, adipisci velit...
\item Neque porro quisquam est qui dolorem ipsum quia dolor sit amet, consectetur, adipisci velit...
\item Neque porro quisquam est qui dolorem ipsum quia dolor sit amet, consectetur, adipisci velit...
\end{itemize}
} % \todo
\clearpage
\begin{figure*}
\begin{center}
\includegraphics[width=.99\columnwidth]{example-image-golden}
\hfill
\includegraphics[width=.99\columnwidth]{example-image-golden}
\end{center}
\caption{
% 
\textbf{Outline -- }
A figure summarizing the entire algorithm
}
\label{fig:overview}
\end{figure*} %< bloody latex and its heuristics for figure placement

\section{Related works}
\label{sec:related}

\AT{introduce subtopics, mention surveys here}
\lorem{1}

\paragraph{Topic}
\cite{acne}
\lorem{2}

\paragraph{Topic}
\cite{nasa}
\lorem{2}

\paragraph{Topic}
\cite{bspnet}
\lorem{2}


\section{Method}
\lorem{1}


\input{tab/sota}
\input{tab/ablations}

\section{Results}
See \Table{sota} and \Table{ablations}.
\lorem{5}

\section{Conclusions}
\lorem{3}


%%%%%%%%% REFERENCES
{
    % \clearpage
    \small
    \bibliographystyle{ieee_fullname}
    \bibliography{macros,main}
}

% --- supplementary material
\appendix

% --- PDF will be split by an editor (e.g. macOS preview), so need to restart from page 1
\setcounter{page}{1}

% --- repeat the title (AT: haven't found a more elegant way to do this...)
\twocolumn[
\centering
\Large
\textbf{[CVPR2022] Official LaTeX Template} \\
\vspace{0.5em}Supplementary Material \\
\vspace{1.0em}
] %< twocolumn
\appendix

\section{Extra experiments}
\lorem{2}

\section{Dataset description}
\lorem{2}

\section{Qualitative results}
\lorem{2}

% --- uncomment this to read the instructions
% \input{sec/X_instructions}

\end{document}
