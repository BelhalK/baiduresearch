\documentclass{article}
\usepackage{xcolor,bm}

\usepackage[utf8]{inputenc} % allow utf-8 input
\usepackage[T1]{fontenc}    % use 8-bit T1 fonts
\usepackage{amssymb}
\usepackage{mathtools}
\usepackage{amsthm}
\usepackage{hyperref}       % hyperlinks
\usepackage{url}            % simple URL typesetting
\usepackage{booktabs}       % professional-quality tables
\usepackage{amsfonts}       % blackboard math symbols
\usepackage{nicefrac}       % compact symbols for 1/2, etc.
\usepackage{microtype}      % microtypography
\usepackage{xcolor}
\usepackage{lipsum}


\begin{document}






\section{Reviewer J9Ez (7;4):}

Weaknesses:

The practical performance of the method is reasonably strong, but I find the experiment on CIFAR-10 a bit puzzling; for the proposed MISSO algorithm, the ELBO starts increasing, while MC-Adam seems to converge stably. Is this due to a perhaps suboptimal choice of parameters in the MISSO algorithm?

The method requires one to store the previous surrogate models, which could be problematic in large-scale applications. I am surprised it was feasible to apply the algorithm to a ResNet-18 on CIFAR-10. Or am I missing something here?

Note that there are stronger baselines than "Bayes by Backprop" for Bayesian deep learning. Since the main contribution of this work is the theoretical analysis, I don't expect a comparison to those (but it still would be nice). But Perhaps they could be cited in the paper:

Fast and Scalable Bayesian Deep Learning by Weight-Perturbation in Adam, Khan et al., ICML 2018
Practical Deep Learning with Bayesian Principles, Osawa et al., NeurIPS 2020


It is unclear to what extend the theoretical analysis can be applied to understand the considered applications, e.g. I am not sure whether the Bayesian deep learning setting fulfills assumption H1-H4.

Note that when using a location-scale parametrization for the variational Gaussian distribution, it is known that the expectation in the cost function is L-smooth and convex for convex and L-smooth integrands, see "Provable Smoothness Guarantees for Black-Box Variational Inference", Domke, ICML 2020. Perhaps this fact could be used to rigorously verify the assumptions for the considered applications?

In general, my suggestion is to address in the experimental section to what extend the theory applies to the considered experiments.





\textcolor{purple}{Our reply:}
Thank you for your detailed review and suggestions.

In order to avoid the confusion about the convergence of MISSO in the Resnet-18 example, we will extend the number of training epochs to 200.
200 epochs will stabilize the ELBO for MISSO, reaching similar values as MC-Adam.
While both methods reach the same local minima, we acknowledge that in this particular example, MC-Adam does reach an $\epsilon-$stationary points in fewer iterations than MISSO, probably suffering from a high variance, hence the twists and turns in the objective loss function.
At this stage, we are not sure if it is possible to generalize a class of problems where baselines can beat MISSO in terms of convergence speed.

We thank you for your references and will make sure to include more baselines relevant to training BNNs.

Lastly, we stress  that while the MISSO scheme does not beat  the SOTA (such as MC-ADAM) on every example, this paper proposes a simple yet \emph{general} incremental optimization framework which encompasses several existing algorithms for large-scale data. 
We have tackled the challenging analysis for an algorithm with double stochasticity (index and latent variable sampling), which is not a minor contribution in our opinion.


\textbf{Verification of Assumptions:} 
Assumption H2 in particular can be easily checked considering that the surrogate function can include the nonsmooth component of the objective function and hence the gap, noted $e$, verifies a smoothness condition (which trivially implies H2).


Our analysis also requires the parameter to be in a compact set.
For the two estimation problems considered, in practice this can be enforced by restricting the parameters in a ball. 
In our simulation, we did not implement the algorithms that stick closely to the compactness requirement for illustrative purposes. However, we observe empirically that the parameters are always bounded.
The update rules can be easily modified to respect the requirement, e.g., for logistic regression, we can use $\|\beta\| \leq R_1, \Omega \succeq \delta {\rm I}, {\rm Tr}(\Omega) \leq R_2$ and adding log-barrier functions as regularizer; for VI, we recall the surrogate functions are quadratic (Eq.(11)) and indeed a simple projection step suffices to ensure boundedness of the iterates.


\section{Reviewer N978 (5;2):}

My rating is based on several concerns listed below.

A subproblem needs to be solved accurately at each iteration, which has a high complexity itself and should be taken account into the complexity analysis.

I am not clear about the technical difficulty to incorporate Monte Carlo and how it helps the theory.

What would go wrong if we abstractly merge the n distributions into one distribution, and draw a minibatch each iteration to make updates? I am not sure how this would affect the theory but it may not be too bad in practice.

\textcolor{purple}{Our reply:}

Thank you for your review.
We would like to clarify that our method applies when those surrogate functions are not directly computable but rather require an approximation (as explained example 1 and 2).
The subproblem that you rightly mentioned is always easily solvable at each MISSO iteration specifically because the surrogate functions are user-designed and hence are convex and tractable. 
Just like we explain in example 2 (VI), the surrogate functions are quadratic and the subproblem you are mentioning is solved using a single gradient step. 
As for general Majorization-Minimization methods, the surrogate minimization does not add substantial cost to the overall, and complex, optimization problem.





\section{Reviewer 2PVe (4;4):}

Cons and my concerns:

The author(s) stated that �MISSO is a general framework for nonconvex and nonsmooth finite sum problems�. I believe that mention nonsmoothness here is misleading. It is hard to imagine a nonsmooth function (such as the absolute value function at the origin) that could satisfy H2 in the paper. Also, the original MISO paper did not say it works for nonsmooth problems. I think it is necessary to remove emphasis on nonsmoothness in this paper.
In line 87, the author(s) stated that H2 can be implied by assuming 
 to be -smooth from [1, Proposition 1]. I checked [1, Proposition 1] and did find much connection to H2. Can you double-check if this is the right reference? I think [2, Theorem 2.1.5, eq (2.1.7)] is more related, but even though I how to prove H2 by assuming smoothness. Can you provide more detail on how to obtain H2? I am currently a little bit skeptical about your statement.
Using Monte-Carlo sampling seems to be a straightforward approach when the surrogate function is an expectation. The algorithm itself cannot claim much novelty, I think the major contribution of this paper is the motivating example and the convergence rates.
The notation is heavy, make the theory a little bit hard to follow.
I did not find the definition of , I guess it is the number of samples at the k-th iteration? Can the author(s) point me to the definition of ?
Eq (15), the definition of 
 is wrong.
Eq (16), $\bar{M}{(k)}\bar{M}{K_{max}}$?
In line 181, why does the term $M_k$
 disappear?
Better to have a precise definition of -stationary point, so readers can better understand line 180-185.
Line 183, $g$ is a scalar, why use $|g|$?
Line 187, set 
$M_k = k^2/n^2$, then $\sum M_k^{-1/2}$ goes to infinity?
Overall, the presentation of this paper should be improved.

[1] A unified convergence analysis of block successive minimization methods for nonsmooth optimization. Razaviyayn et al. SIOPT 2013. [2] Introductory Lectures on Convex optimization: A basic course. Yurii Nesterov. 2014.



\textcolor{purple}{Our reply:}

We thank you for your detailed analysis of our paper.

The smoothness of the gap function noted $e$ (or assumption H2) is easily checked when considering that the nonsmooth component of the objective function (1) can be included in the surrogate function. That way, both nonsmooth parts cancel out when computing the function $e$.

For clarity, the general objective function we are considering can be written as $\cal L(\theta) = f(\theta) + r(\theta)$ where $f$ is possibly nonconvex and $r$ is possibly nonsmooth (acting most of the time as a regularizer). In that case, the surrogate function $\hat L(\theta, \vartheta)$ can include $r(\vartheta)$ so that the gap function noted $\widehat{e}(\theta ; \{ \theta_i \}_{i=1}^n ) = \frac{1}{n} \sum_{i=1}^n \hat L_i(\theta;\theta_i) - {\cal L}( \theta)$ cancels out this undesirable quantity.
Assumption H2 is hence derived from the smoothness of the gap between two smooth functions as follows:
we know that under assumption H1, the gap is always positive or null, i.e. $0 \leq \hat{e}( \vartheta ; \{ \theta_i \}_{i=1}^n)$ for any $\vartheta \in \Theta$. In particular for $\vartheta = \theta - (1/L)*\nabla \hat{e}(\theta; \{ \theta_i \}_{i=1}^n)$ we have:
$$
0 \leq \hat{e}\big( \theta - (1/L)*\nabla \hat{e}(\theta; \{ \theta_i \}_{i=1}^n) ; \{ \theta_i \}_{i=1}^n\big)  \, .
$$

Hence, due to the smoothness of the function $\hat{e}$, we have
\begin{equation}
\begin{split}
    0 \leq \hat{e}( \theta - (1/L)*\nabla \hat{e}(\theta) ; \{ \theta_i \}_{i=1}^n) \leq \hat{e} ( \theta ; \{ \theta_i \}_{i=1}^n)) - (1/2L) || \nabla \hat{e}( \theta ; \{ \theta_i \}_{i=1}^n) ||^2
\end{split}
\end{equation}
Finally rearranging the terms yields:

$$
|| \nabla \hat{e}( \theta; \{ \theta_i \}_{i=1}^n)  ||^2 \leq 2L *  \hat{e} ( \theta; \{ \theta_i \}_{i=1}^n) 
$$
proving assumption H2 for all $\theta \in \Theta$.





The quantity $M_{(k)}$ is defined in Algorithm 2 Line 1 and Line 6 and corresponds to the number of Monte Carlo samples used in the approximation of the surrogate function.
Thank you for the typo Eq.(15) and (16): $(k)$ index should be $(K_{\sf max})$.

\section{Reviewer 1meD (6;4):}

Clarity: The paper is well-motivated and clearly written. For example, I can follow the intro very easily and understand the main idea of the paper and the distinction from prior works. The main theorems and experiments are also clear to me.

Quality: The theoretical development is solid and interesting. For example, Theorem 1 is derived for MISSO under stochastic setting but it can be specialized for MISO. From the experiments, it is not obvious that MISSO improves prior works for logistic regression with missing values or fitting Bayesian networks (as the authors themselves confessed).

Originality: The originality is moderate. Examples 1 and 2 are well-known. The MISSO algorithm is a minor modification of MISO that incoporates Monte Carlo sampling, the later technique being already well-known.

Significance: Researchers working on stochastic optimization and (non-)convex analysis might be interested in the paper. The paper might not attract practitioners in machine learning applications mentioned in the paper.

Other Comments: Does it make sense to specialize Theorems 1-2 to Examples 1-2, or to logistic regression or fitting Bayesian networks? Please make a clarification here. It is of interest to see how large L is in those applications, as Theorem 1 requires $(nL^3)/\epsilon$ samples.

\textcolor{purple}{Our reply:}
We would like to thank the reviewer for the suggestions.
We would like to recall the nature of our contribution to address the concerns on the originality and relevance of our work raised by the reviewer.

We want to stress on the generality of our incremental optimization framework, which tackles a \emph{constrained}, \emph{non-convex} and \emph{non-smooth} optimization problem. 
The main contribution of this paper is to propose and analyze a \textbf{unifying framework} for a large class of optimization algorithms which includes many well-known but not so well-studied algorithms.
The major idea here is to relax the class of surrogate functions used in MISO [Mairal, 2015] and to allow for intractable surrogate that can only be evaluated by Monte-Carlo approximations.
We provide a general algorithm and global convergence rate analysis under mild assumptions on the model and show that two examples, MLE for latent data models and Variational Inference, are its special instances.
Working at the crossroads of \emph{Optimization} and \emph{Sampling} constitutes what we believe to be the novelty and the technicality of our theoretical results, and is not a straightforward work.

As example 1 and 2 are indeed popular applications and algorithms, we actually show how our unifying framework can be applied. Making it a great tool and scheme to study popular and widely used methods. We also stress on the high applicability and usage of those models by ML practicioners.

% Regarding the quantification of the smoothness constant


\end{document}

