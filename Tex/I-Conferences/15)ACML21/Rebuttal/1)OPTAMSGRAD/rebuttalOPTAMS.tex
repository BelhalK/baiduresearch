%\documentclass[wcp,gray]{jmlr} % test grayscale version
\documentclass[wcp]{jmlr}

% The following packages will be automatically loaded:
% amsmath, amssymb, natbib, graphicx, url, algorithm2e

%\usepackage{rotating}% for sideways figures and tables
\usepackage{longtable}% for long tables

% The booktabs package is used by this sample document
% (it provides \toprule, \midrule and \bottomrule).
% Remove the next line if you don't require it.
\usepackage{booktabs}
% The siunitx package is used by this sample document
% to align numbers in a column by their decimal point.
% Remove the next line if you don't require it.
%\usepackage[load-configurations=version-1]{siunitx} % newer version
%\usepackage{siunitx}
%\usepackage{natbib}

% The following command is just for this sample document:
\newcommand{\cs}[1]{\texttt{\char`\\#1}}
\makeatletter
\let\Ginclude@graphics\@org@Ginclude@graphics 
\makeatother

\jmlrvolume{}
\jmlryear{2021}
\jmlrworkshop{ACML 2021}

\title[]{Rebuttal for 'An Optimistic Acceleration of AMSGrad for Nonconvex Optimization'}

 % Use \Name{Author Name} to specify the name.
 % If the surname contains spaces, enclose the surname
 % in braces, e.g. \Name{John {Smith Jones}} similarly
 % if the name has a "von" part, e.g \Name{Jane {de Winter}}.
 % If the first letter in the forenames is a diacritic
 % enclose the diacritic in braces, e.g. \Name{{\'E}louise Smith}

 % Two authors with the same address
 % \author{\Name{Author Name1} \Email{abc@sample.com}\and
 %  \Name{Author Name2} \Email{xyz@sample.com}\\
 %  \addr Address}

 % Three or more authors with the same address:
 % \author{\Name{Author Name1} \Email{an1@sample.com}\\
 %  \Name{Author Name2} \Email{an2@sample.com}\\
 %  \Name{Author Name3} \Email{an3@sample.com}\\
 %  \Name{Author Name4} \Email{an4@sample.com}\\
 %  \Name{Author Name5} \Email{an5@sample.com}\\
 %  \Name{Author Name6} \Email{an6@sample.com}\\
 %  \Name{Author Name7} \Email{an7@sample.com}\\
 %  \Name{Author Name8} \Email{an8@sample.com}\\
 %  \Name{Author Name9} \Email{an9@sample.com}\\
 %  \Name{Author Name10} \Email{an10@sample.com}\\
 %  \Name{Author Name11} \Email{an11@sample.com}\\
 %  \Name{Author Name12} \Email{an12@sample.com}\\
 %  \Name{Author Name13} \Email{an13@sample.com}\\
 %  \Name{Author Name14} \Email{an14@sample.com}\\
 %  \addr Address}


 % Authors with different addresses:
%  \author{\Name{Author Name1} \Email{abc@sample.com}\\
%  \addr Address 1
%  \AND
%  \Name{Author Name2} \Email{xyz@sample.com}\\
%  \addr Address 2
% }

%\editors{Vineeth N Balasubramanian and Ivor Tsang}

\begin{document}

\maketitle

%\begin{abstract}
%This is the abstract for this article.
%\end{abstract}
%\begin{keywords}
%List of keywords
%\end{keywords}

We sincerely thank the three reviewers for their valuable feedback. 
Upon acceptance, we will include in the final version \emph{{\sf (a)} improved notations} and \emph{{\sf (b)} an improved presentation of our bounds and proofs}. 


\vspace{0.05in}

\textbf{\textcolor{blue}{Reviewer 1:}} We thank the reviewer for the valuable comments.\vspace{-5pt}

\medskip

\textbf{Notations:} 
We will include the suggested notation for $T$.
We will also define the dual norm in the Notations paragraph in the revision of our paper.

\vspace{0.05in}

\textbf{\textcolor{brown}{Reviewer 2:}} We thank the reviewer for the valuable feedback on our contribution.

\vspace{0.05in}

\textbf{\textcolor{green!50!black}{Reviewer 3:}} We thank the reviewer for the thorough analysis. Our remarks are listed below:\vspace{-5pt}

\medskip
\textbf{-- Assumption H3:}
Thanks for your constructive comments. It is clear that in convex case a better prediction reduces the bound. In the non-convex case it holds as well, with some careful analysis. For \textbf{H3}, if we alternatively consider $0<m_t^T g_t=a\Vert g_t\Vert^2$ and $\Vert m_t\Vert\leq \Vert g_t\Vert$ (i.e. $m_t$ lies in the hemisphere with $g_t$ as its midline), we can show that $\tilde C_2$ reaches minimum when $a=1$ (i.e. $m_t=g_t$). Also, $\tilde C_1$ is minimized at $a=1$ under some conditions on the parameters ($\beta_1,\beta_2$ etc.). \textbf{That means the bound for non-convex case is tighter when $m_t$ predicts $g_t$ well, similar to the convex analysis.} We will adjust our discussion and presentation in the paper to address this point. \vspace{-5pt}

\medskip
\textbf{-- Proof Theorem 1:} As rightly mentioned by the reviewer, we use Eq (18) the inequality $\|w_t-\tilde{w}_{t+1}\|\|g_t-\tilde{m_t}\|\leq (1/2\eta)\|w_t-\tilde{w}_{t+1}\|^2+ (\eta/2)\|g_t-\tilde{m_t}\|^2$ which stems from an application of Young's inequality as explained page 18 under Eq (18). $m_t$ should read $\tilde{m}_t$, this is a typo since we can notice in the final bound that only $\tilde{m}_t$ appears. This typo is fixed and does not change the final bound.

When $\beta_1 = 0$, $h_t = \tilde{m}_t$ indeed. We will include that.

Equation (20) is a one line calculation. It corresponds to a weighted sum of squares bounded by the largest term times the sum of weights.
We will add an intermediate line in the revision.

\medskip
\textbf{-- Lemma 3:}
Note that Eq (36) uses the intermediate equality on the quantity $\overline{w}_{t+1} - \overline{w}_t$ (and not the upperbound) that is used in the proof of Lemma 3. We will clarify this point in our proof.
Hence, as we are using an equality, the problem you raised is no longer one.


We thank the reviewer for the typo (bad placement of subscript) in eq (6) that will be fixed in our revision.

\end{document}
