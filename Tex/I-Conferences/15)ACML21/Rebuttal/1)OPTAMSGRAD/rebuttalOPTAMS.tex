\documentclass{article}
\usepackage{neurips_2020_author_response,xcolor,bm}

\usepackage[utf8]{inputenc} % allow utf-8 input
\usepackage[T1]{fontenc}    % use 8-bit T1 fonts
\usepackage{hyperref}       % hyperlinks
\usepackage{url}            % simple URL typesetting
\usepackage{booktabs}       % professional-quality tables
\usepackage{amsfonts}       % blackboard math symbols
\usepackage{nicefrac}       % compact symbols for 1/2, etc.
\usepackage{microtype}      % microtypography
\usepackage{xcolor}
\usepackage{lipsum}


\begin{document}

We sincerely thank the four reviewers for their valuable feedback. 
% Upon acceptance, we will include in the final version \emph{{\sf (a)} improved notations} and \emph{{\sf (b)} an improved presentation of related work}. 
We first discuss a few common concerns shared by \textbf{\color{blue}reviewer 1}, \textbf{\color{red} reviewer 2}, \textbf{\color{green!50!black}reviewer 3} and \textbf{\color{purple}reviewer 4}. \vspace{-5pt}

${\color{blue}\bullet}\!~{\color{red}\bullet}$ \textbf{Non-convex bound:} 
 empirical edge of the optimistic update but also a theoretical. 
Thanks for your constructive comments. It is clear that in convex case a better prediction reduces the bound. In the non-convex case it holds as well, with some careful analysis. For \textbf{H3}, if we alternatively consider $0<m_t^T g_t=a\Vert g_t\Vert^2$ and $\Vert m_t\Vert\leq \Vert g_t\Vert$ (i.e. $m_t$ lies in the hemisphere with $g_t$ as its midline), we can show that $\tilde C_2$ reaches minimum when $a=1$ (i.e. $m_t=g_t$). Also, $\tilde C_1$ is minimized at $a=1$ under some conditions on the parameters ($\beta_1,\beta_2$ etc.). \textbf{That means the bound for non-convex case is tighter when $m_t$ predicts $g_t$ well, similar to the convex analysis.} We will adjust our discussion and presentation in the paper to address this point. \vspace{-5pt}
% Thanks again for the insightful suggestion.


\vspace{0.05in}

\textbf{\textcolor{blue}{Reviewer 1:}} We thank the reviewer for valuable comments. Our point-to-point response is as follows:\vspace{-5pt}


\textbf{Convex regret bound:} 


\vspace{0.05in}


\textbf{\textcolor{red}{Reviewer 2:}} We thank the reviewer for valuable comments. A proofreading is being done we clarify that:\vspace{-5pt}

\textbf{Novelty of the contribution:} 


\vspace{0.05in}
\textbf{\textcolor{green!50!black}{Reviewer 3:}} We thank the reviewer for the thorough analysis. Our remarks are listed below:\vspace{-5pt}

\textbf{Gradient prediction algorithm:}


\end{document}

