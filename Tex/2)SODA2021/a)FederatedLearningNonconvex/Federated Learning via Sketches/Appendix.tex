\section{Appendix}
\section{Proof of main Theorems}
The proof of Theorem~\ref{thm:homog_case} follows directly from the results in~\cite{haddadpour2020federated}. For the sake of the completeness we review an assumptions from this reference for the quantiziation with their notation.

\begin{assumption}[\cite{haddadpour2020federated}]\label{Assu:quant}
The output of the compression operator $Q(\mathbf{x})$ is an unbiased estimator of its input $\mathbf{x}$, and its variance grows with the squared of the squared of $\ell_2$-norm of its argument, i.e., $\mathbb{E}\left[Q(\mathbf{x})\right]=\mathbf{x}$ and $\mathbb{E}\left[\left\|Q(\mathbf{x})-\mathbf{x}\right\|^2\right]\leq q\left\|\mathbf{x}\right\|^2$ .
\end{assumption}


\subsection{Proof of Theorem~\ref{thm:homog_case}}
Based on Assumption~\ref{Assu:quant} we have:
\begin{theorem}[\cite{haddadpour2020federated}]\label{thm:fromhaddad}
 Consider \texttt{FedCOM} in \cite{haddadpour2020federated}. Suppose that the conditions in Assumptions~\ref{Assu:1}, \ref{Assu:1.5} and \ref{Assu:quant} hold. If the local data distributions of all users are identical (homogeneous setting), then we have  
 \begin{itemize}
     \item \textbf{Nonconvex:}  By choosing stepsizes as $\eta=\frac{1}{L\gamma}\sqrt{\frac{p}{R\tau\left(\frac{q}{p}+1\right)}}$ and $\gamma\geq p$, the sequence of iterates satisfies  $\frac{1}{R}\sum_{r=0}^{R-1}\left\|\nabla f({\boldsymbol{w}}^{(r)})\right\|_2^2\leq {\epsilon}$ if we set
     $R=O\left(\frac{1}{\epsilon}\right)$ and $ \tau=O\left(\frac{\frac{q}{p}+1}{{p}\epsilon}\right)$.
     \item \textbf{Strongly convex or PL:}
      By choosing stepsizes as $\eta=\frac{1}{2L\left(\frac{q}{p}+1\right)\tau\gamma}$ and $\gamma\geq m$, we obtain that the iterates satisfy $\mathbb{E}\Big[f({\boldsymbol{w}}^{(R)})-f({\boldsymbol{w}}^{(*)})\Big]\leq \epsilon$ if  we set
     $R=O\left(\left(\frac{q}{p}+1\right)\kappa\log\left(\frac{1}{\epsilon}\right)\right)$ and $ \tau=O\left(\frac{1}{p\epsilon}\right)$.
     \item \textbf{Convex:} By choosing stepsizes as $\eta=\frac{1}{2L\left(\frac{q}{p}+1\right)\tau\gamma}$ and $\gamma\geq p$, we obtain that the iterates satisfy $ \mathbb{E}\Big[f({\boldsymbol{w}}^{(R)})-f({\boldsymbol{w}}^{(*)})\Big]\leq \epsilon$ if we set
     $R=O\left(\frac{L\left(1+\frac{q}{p}\right)}{\epsilon}\log\left(\frac{1}{\epsilon}\right)\right)$ and $ \tau=O\left(\frac{1}{p\epsilon^2}\right)$.
 \end{itemize}
\end{theorem}

\begin{proof}
Since the sketching \texttt{PRIVIX} and \texttt{HEAPRIX}, satisfy the Assumption~\ref{Assu:quant} with $q=\mu^2d$ and $q=\mu^2d-1$ respectively with probablity $1-\delta$.  Therefore, all the results in Theorem~\ref{thm:homog_case}, conclude from Theorem~\ref{thm:fromhaddad} with probability $1-\delta$ and plugging $q=\mu^2d$ and $q=\mu^2d-1$ respectively into the corresponding convergence bounds.
\end{proof}


\subsection{Proof of Theorem~\ref{thm:hetreg_case}}
For the heterogeneous setting, the results in~\cite{haddadpour2020federated} requires the following extra assumption that naturally holds for the sketching: 

\begin{assumption}[\cite{haddadpour2020federated}]\label{assum:009}
The compression scheme $Q$ for the heterogeneous data distribution setting satisfies the following condition $
    \mathbb{E}_Q[\|\frac{1}{m}\sum_{j=1}^m Q(\boldsymbol{x}_j)\|^2-\|Q(\frac{1}{m}\sum_{j=1}^m \boldsymbol{x}_j)\|^2]\leq G_q$.
\end{assumption}
We note that since sketching is a linear compressor, in the case of our algorithms for heterogeneous setting we have $G_q=0$. 

Next, we restate the Theorem in \cite{haddadpour2020federated} here as follows:

\begin{theorem}\label{thm:fromhaddad-het}
 Consider \texttt{FedCOMGATE} in \cite{haddadpour2020federated}. If Assumptions~\ref{Assu:1}, \ref{Assu:2}, \ref{Assu:quant}  and \ref{assum:009} hold, then even for the case the local data distribution of users are different  (heterogeneous setting) we have
 \begin{itemize}
     \item \textbf{Non-convex:} By choosing stepsizes as $\eta=\frac{1}{L\gamma}\sqrt{\frac{p}{R\tau\left(q+1\right)}}$ and $\gamma\geq p$, we obtain that the iterates satsify  $\frac{1}{R}\sum_{r=0}^{R-1}\left\|\nabla f({\boldsymbol{w}}^{(r)})\right\|_2^2\leq \epsilon$ if we set
     $R=O\left(\frac{q+1}{\epsilon}\right)$ and $ \tau=O\left(\frac{1}{p\epsilon}\right)$.
     \item \textbf{Strongly convex or PL:}
      By choosing stepsizes as $\eta=\frac{1}{2L\left(\frac{q}{p}+1\right)\tau\gamma}$ and ${\gamma\geq \sqrt{p\tau}}$, we obtain that the iterates satisfy $\mathbb{E}\Big[f({\boldsymbol{w}}^{(R)})-f({\boldsymbol{w}}^{(*)})\Big]\leq \epsilon$ if we set
      $R=O\left(\left(q+1\right)\kappa\log\left(\frac{1}{\epsilon}\right)\right)$ and $ \tau=O\left(\frac{1}{p\epsilon}\right)$.
     \item \textbf{Convex:}  By choosing stepsizes as $\eta=\frac{1}{2L\left(q+1\right)\tau\gamma}$ and ${\gamma\geq \sqrt{p\tau}}$, we obtain that the iterates satisfy $\mathbb{E}\Big[f({\boldsymbol{w}}^{(R)})-f({\boldsymbol{w}}^{(*)})\Big]\leq \epsilon$ if we set
     $R=O\left(\frac{L\left(1+q\right)}{\epsilon}\log\left(\frac{1}{\epsilon}\right)\right)$ and $ \tau=O\left(\frac{1}{p\epsilon^2}\right)$.
 \end{itemize}
 
\end{theorem}
\begin{proof}
Since the sketching \texttt{PRIVIX} and \texttt{HEAPRIX}, satisfy the Assumption~\ref{Assu:quant} with $q=\mu^2d$ and $q=\mu^2d-1$ respectively with probablity $1-\delta$.  Therefore, all the results in Theorem~\ref{thm:hetreg_case}, conclude from Theorem~\ref{thm:fromhaddad-het} with probability $1-\delta$ and plugging $q=\mu^2d$ and $q=\mu^2d-1$ respectively into the convergence bounds.
\end{proof}
%%%%%%%%%%%%%%%%%%%%%%%%%%%%%%%%%%%%%%%%%%%%%%%%
%%%%%%%%%%%%%%%%%%%%%%%%%%%%%%%%%%%%%%%%%%%%%%%%
\section{Convergence result for \texttt{FEDSKETCH} without memory}
From the $L$-smoothness gradient assumption on global objective, by using  $\underline{\mathbf{S}}^{(r)}=\tilde{\mathbf{g}}^{(r)}$ in inequality (\ref{eq:decent-smoothe}) we have:
\begin{align}
    f({\boldsymbol{x}}^{(r+1)})-f({\boldsymbol{x}}^{(r)})\leq -\gamma \big\langle\nabla f({\boldsymbol{x}}^{(r)}),\tilde{\mathbf{g}}^{(r)}\big\rangle+\frac{\gamma^2 L}{2}\|\tilde{\mathbf{g}}^{(r)}\|^2\label{eq:Lipschitz-c1}
\end{align}
We define the following:
\begin{align}
    \tilde{\mathbf{g}}_{\mathbf{S}}^{(r)}=\frac{\eta}{p}\sum_{j=1}^{p}\mathbf{S}\left[\sum_{c=0}^{\tau-1}\tilde{\mathbf{g}}_j^{(c,r)}\right]
\end{align}
Additionally, we define an auxiliary variable as 
\begin{align}
    \tilde{\mathbf{g}}^{(r)}=\frac{\eta}{p}\sum_{j=1}^{p}\left[\sum_{c=0}^{\tau-1}\tilde{\mathbf{g}}_j^{(c,r)}\right]
\end{align}
%%%%%%%%%%%%%%%%%%%%%%%%%%%%%%%%%%%%%%%%%%%%
By taking expectation on both sides of above inequality over sampling, we get:
\begin{align}
    \mathbb{E}\left[\mathbb{E}_\mathbf{S}\Big[f({\boldsymbol{x}}^{(r+1)})-f({\boldsymbol{x}}^{(r)})\Big]\right]&\leq -\gamma\mathbb{E}\left[\mathbb{E}_\mathbf{S}\left[ \big\langle\nabla f({\boldsymbol{x}}^{(r)}),\tilde{\mathbf{g}}_\mathbf{S}^{(r)}\big\rangle\right]\right]+\frac{\gamma^2 L}{2}\mathbb{E}\left[\mathbb{E}_\mathbf{S}\|\tilde{\mathbf{g}}_\mathbf{S}^{(r)}\|^2\right]\nonumber\\
    &=-\gamma\mathbb{E}\left[\mathbb{E}_\mathbf{S}\left[ \big\langle\nabla f({\boldsymbol{x}}^{(r)}),\tilde{\mathbf{g}}^{(r)}\big\rangle\right]\right]+\gamma\mathbb{E}\left[\mathbb{E}_\mathbf{S}\left[ \big\langle\nabla f({\boldsymbol{x}}^{(r)}),\tilde{\mathbf{g}}^{(r)}-\tilde{\mathbf{g}}_{\mathbf{S}}^{(r)}\big\rangle\right]\right]\nonumber\\
    &\qquad+\frac{\gamma^2 L}{2}\mathbb{E}\left[\mathbb{E}_\mathbf{S}\|\tilde{\mathbf{g}}_\mathbf{S}^{(r)}-\tilde{\mathbf{g}}^{(r)}+\tilde{\mathbf{g}}^{(r)}\|^2\right] \nonumber\\
    &\stackrel{(a)}{=}-\gamma\mathbb{E}\left[\mathbb{E}_\mathbf{S}\left[ \big\langle\nabla f({\boldsymbol{x}}^{(r)}),\tilde{\mathbf{g}}^{(r)}\big\rangle\right]\right]+\gamma\left[\mathbb{E}_\mathbf{S}\left[ \big\langle\nabla f({\boldsymbol{x}}^{(r)}),{\mathbf{g}}^{(r)}-{\mathbf{g}}_{\mathbf{S}}^{(r)}\big\rangle\right]\right]\nonumber\\
    &\qquad+\frac{\gamma^2 L}{2}\mathbb{E}\left[\mathbb{E}_\mathbf{S}\|\tilde{\mathbf{g}}_\mathbf{S}^{(r)}-\tilde{\mathbf{g}}^{(r)}+\tilde{\mathbf{g}}^{(r)}\|^2\right]\nonumber\\
    &\stackrel{(b)}{\leq}-\gamma\mathbb{E}\left[\mathbb{E}_\mathbf{S}\left[ \big\langle\nabla f({\boldsymbol{x}}^{(r)}),\tilde{\mathbf{g}}^{(r)}\big\rangle\right]\right]+\frac{\gamma}{2}\left[ \frac{1}{mL}\left\|\nabla f({\boldsymbol{x}}^{(r)})\right\|^2_2+mL\mathbb{E}_\mathbf{S}\left[\left\|{\mathbf{g}}^{(r)}-{\mathbf{g}}_{\mathbf{S}}^{(r)}\right\|^2_2\right]\right]\nonumber\\
    &\qquad+{\gamma^2 L}\mathbb{E}\left[\mathbb{E}_\mathbf{S}\left\|\tilde{\mathbf{g}}_\mathbf{S}^{(r)}-\tilde{\mathbf{g}}^{(r)}\right\|+\left\|\tilde{\mathbf{g}}^{(r)}\right\|^2\right] \nonumber\\
    &\stackrel{(c)}{\leq}-\gamma\mathbb{E}\left[ \big\langle\nabla f({\boldsymbol{x}}^{(r)}),\tilde{\mathbf{g}}^{(r)}\big\rangle\right]+\frac{\gamma}{2}\left[ \frac{1}{mL}\left\|\nabla f({\boldsymbol{x}}^{(r)})\right\|^2_2+mL\left(1-\frac{k}{d}\right)\left\|{\mathbf{g}}^{(r)}\right\|^2_2\right]\nonumber\\
    &\qquad+{\gamma^2 L}\mathbb{E}\left[\left(1-\frac{k}{d}\right)\left\|\tilde{\mathbf{g}}^{(r)}\right\|_2^2+\left\|\tilde{\mathbf{g}}^{(r)}\right\|_2^2\right]\nonumber\\
    &\stackrel{(d)}{=}-\gamma\underbrace{\mathbb{E}\left[ \big\langle\nabla f({\boldsymbol{x}}^{(r)}),\tilde{\mathbf{g}}^{(r)}\big\rangle\right]}_{(\mathrm{I})}+ \frac{\gamma}{2mL}\left\|\nabla f({\boldsymbol{x}}^{(r)})\right\|^2_2+\frac{mL\gamma}{2}\left(1-\frac{k}{d}\right)\underbrace{\left\|{\mathbf{g}}^{(r)}\right\|^2_2}_{(\mathrm{II})}\nonumber\\
    &\qquad+{\gamma^2 L}\left(2-\frac{k}{d}\right)\underbrace{\mathbb{E}\left[\left\|\tilde{\mathbf{g}}^{(r)}\right\|_2^2\right]}_{(\mathrm{III})}\label{eq:Lipschitz-c-gd-alt}
\end{align}
To bound term ($\mathrm{I}$) in Eq.~(\ref{eq:Lipschitz-c-gd-alt}) we use the combination of Lemmas~\ref{} and \ref{} we obtain:
\begin{align}
    -\gamma\mathbb{E}\left[ \big\langle\nabla f({\boldsymbol{x}}^{(r)}),\tilde{\mathbf{g}}^{(r)}\big\rangle\right]\leq \frac{\gamma}{2}\eta\frac{1}{p}\sum_{j=1}^p\sum_{c=0}^{\tau-1}\left[-\left\|\nabla f({\boldsymbol{x}}^{(r)})\right\|_2^2-\left\|\mathbf{g}_j^{(\ell,r)}\right\|_2^2+L^2\eta^2\sum_{\ell=0}^{\tau-1}\left[\tau\left\|{\mathbf{g}}_j^{(\ell,r)}\right\|_2^2+\sigma^2\right]\right]
\end{align}
Term $(\mathrm{II})$ can be bounded simply as follows:
\begin{align}
    \left\|{\mathbf{g}}^{(r)}\right\|^2_2&=\left\|\frac{\eta}{p}\sum_{j=1}^{p}\left[\sum_{c=0}^{\tau-1}{\mathbf{g}}_j^{(c,r)}\right]\right\|^2_2\nonumber\\
    &\leq\frac{\tau\eta^2}{p}\sum_{j=1}^{p}\sum_{c=0}^{\tau-1}\left\|\mathbf{g}_j^{(c,r)}\right\|^2_2
\end{align}

Next we bound term $(\mathrm{III})$ using the following lemma:
\begin{lemma}
\begin{align}
    \mathbb{E}\left[\left\|\tilde{\mathbf{g}}^{(r)}\right\|_2^2\right]\leq \frac{\eta^2\tau}{p}\sum_{j=1}^{p}\sum_{c=0}^{\tau-1}\left\|\mathbf{g}_j^{(c,r)}\right\|^2_2+\frac{\eta^2\tau}{p}\sigma^2
\end{align}
\end{lemma}
\begin{proof}
\begin{align}
    \mathbb{E}\left[\left\|\tilde{\mathbf{g}}^{(r)}\right\|_2^2\right]&=\mathbb{E}\left[\left\|\tilde{\mathbf{g}}^{(r)}-\mathbb{E}\left[\tilde{\mathbf{g}}^{(r)}\right]\right\|_2^2\right]+\left\|\mathbb{E}\left[\tilde{\mathbf{g}}^{(r)}\right]\right\|^2_2\nonumber\\
    &= \mathbb{E}\left[\left\|\tilde{\mathbf{g}}^{(r)}-{\mathbf{g}}^{(r)}\right\|_2^2\right]+\left\|{\mathbf{g}}^{(r)}\right\|^2_2\nonumber\\
    &= \mathbb{E}\left[\left\|\frac{\eta}{p}\sum_{j=1}^{p}\left[\sum_{c=0}^{\tau-1}\tilde{\mathbf{g}}_j^{(c,r)}\right]-\frac{\eta}{p}\sum_{j=1}^{p}\left[\sum_{c=0}^{\tau-1}\mathbf{g}_j^{(c,r)}\right]\right\|_2^2\right]+\left\|\frac{\eta}{p}\sum_{j=1}^{p}\left[\sum_{c=0}^{\tau-1}\mathbf{g}_j^{(c,r)}\right]\right\|^2_2\nonumber\\
&=\frac{\eta^2}{p^2}\sum_{j=1}^{p}\sum_{c=0}^{\tau-1}\mathbb{E}\left[\left\|\tilde{\mathbf{g}}_j^{(c,r)}-\mathbf{g}_j^{(c,r)}\right\|_2^2\right]+\left\|\frac{\eta}{p}\sum_{j=1}^{p}\left[\sum_{c=0}^{\tau-1}\mathbf{g}_j^{(c,r)}\right]\right\|^2_2 \nonumber\\
&\leq \frac{\eta^2}{p^2}\sum_{j=1}^{p}\sum_{c=0}^{\tau-1}\mathbb{E}\left[\left\|\tilde{\mathbf{g}}_j^{(c,r)}-\mathbf{g}_j^{(c,r)}\right\|_2^2\right]+\frac{\eta^2\tau}{p}\sum_{j=1}^{p}\sum_{c=0}^{\tau-1}\left\|\mathbf{g}_j^{(c,r)}\right\|^2_2\nonumber\\
&\leq \frac{\eta^2}{p^2}\sum_{j=1}^{p}\sum_{c=0}^{\tau-1}\sigma^2+\frac{\eta^2\tau}{p}\sum_{j=1}^{p}\sum_{c=0}^{\tau-1}\left\|\mathbf{g}_j^{(c,r)}\right\|^2_2\nonumber\\
&=\frac{\eta^2\tau}{p}\sum_{j=1}^{p}\sum_{c=0}^{\tau-1}\left\|\mathbf{g}_j^{(c,r)}\right\|^2_2+\frac{\eta^2\tau}{p}\sigma^2
\end{align}
\end{proof}
Next, we put all the pieces together as follows:
\begin{align}
    \mathbb{E}\left[\mathbb{E}_\mathbf{S}\Big[f({\boldsymbol{x}}^{(r+1)})-f({\boldsymbol{x}}^{(r)})\Big]\right]&\leq \frac{\gamma}{2}\eta\frac{1}{p}\sum_{j=1}^p\sum_{\ell=0}^{\tau-1}\left[-\left\|\nabla f({\boldsymbol{x}}^{(r)})\right\|_2^2-\left\|\mathbf{g}_j^{(\ell,r)}\right\|_2^2+L^2\eta^2\sum_{\ell=0}^{\tau-1}\left[\tau\left\|{\mathbf{g}}_j^{(\ell,r)}\right\|_2^2+\sigma^2\right]\right]\nonumber\\
    &\quad+ \frac{\gamma}{2mL}\left\|\nabla f({\boldsymbol{x}}^{(r)})\right\|^2_2+\frac{mL\gamma}{2}\left(1-\frac{k}{d}\right)\frac{\tau\eta^2}{p}\sum_{j=1}^{p}\sum_{\ell=0}^{\tau-1}\left\|\mathbf{g}_j^{(\ell,r)}\right\|^2_2\nonumber\\
    &\quad+\gamma^2 L\left(2-\frac{k}{d}\right)\left[\frac{\eta^2\tau}{p}\sum_{j=1}^{p}\sum_{c=0}^{\tau-1}\left\|\mathbf{g}_j^{(\ell,r)}\right\|^2_2+\frac{\eta^2\tau}{p}\sigma^2\right]\nonumber\\
    &=-\frac{\tau\eta\gamma}{2}\left\|\nabla f({\boldsymbol{x}}^{(r)})\right\|_2^2+\frac{\gamma}{2}\eta\frac{1}{p}\sum_{j=1}^p\sum_{\ell=0}^{\tau-1}\left[-\left\|\mathbf{g}_j^{(\ell,r)}\right\|_2^2+L^2\eta^2\tau^2\left\|{\mathbf{g}}_j^{(\ell,r)}\right\|_2^2\right]+\frac{\gamma\eta^3L^2\tau^2}{2}\sigma^2\nonumber\\
    &\quad+ \frac{\gamma}{2mL}\left\|\nabla f({\boldsymbol{x}}^{(r)})\right\|^2_2+\frac{mL\gamma}{2}\left(1-\frac{k}{d}\right)\frac{\tau\eta^2}{p}\sum_{j=1}^{p}\sum_{\ell=0}^{\tau-1}\left\|\mathbf{g}_j^{(\ell,r)}\right\|^2_2\nonumber\\
    &\quad+\gamma^2 L\left(2-\frac{k}{d}\right)\frac{\eta^2\tau}{p}\sum_{j=1}^{p}\sum_{\ell=0}^{\tau-1}\left\|\mathbf{g}_j^{(\ell,r)}\right\|^2_2+\gamma^2 L\left(2-\frac{k}{d}\right)\frac{\eta^2\tau}{p}\sigma^2\nonumber\\
    &=-\left(\frac{\tau\eta\gamma}{2}-\frac{\gamma}{2mL}\right)\left\|\nabla f({\boldsymbol{x}}^{(r)})\right\|_2^2\nonumber\\
    &\quad-\left(\frac{\eta\gamma}{2}-\frac{\eta\gamma}{2}\left(L^2\eta^2\tau^2\right)-\frac{mL\eta\gamma}{2}\left(1-\frac{k}{d}\right)\tau\eta-\gamma^2 L\eta^2\tau\left(2-\frac{k}{d}\right)\right)\frac{1}{p}\sum_{j=1}^{p}\sum_{\ell=0}^{\tau-1}\left\|\mathbf{g}_j^{(\ell,r)}\right\|^2_2\nonumber\\
    &\quad+\frac{\gamma\eta^3L^2\tau^2}{2}\sigma^2+\gamma^2 L\left(2-\frac{k}{d}\right)\frac{\eta^2\tau}{p}\sigma^2\nonumber\\
    &\stackrel{(a)}{\leq}-\left(\frac{\tau\eta\gamma}{2}-\frac{\gamma}{2mL}\right)\left\|\nabla f({\boldsymbol{x}}^{(r)})\right\|_2^2+\frac{\gamma\eta^3L^2\tau^2}{2}\sigma^2+\tau\eta^2\gamma^2 L\left(2-\frac{k}{d}\right)\frac{\sigma^2}{p}\label{eq:ncvx-mid-step}
\end{align}
where (a) follows from the learning rate choices of 
\begin{align}
    \frac{\eta\gamma}{2}-\frac{\eta\gamma}{2}\left(L^2\eta^2\tau^2\right)-\frac{mL\eta\gamma}{2}\left(1-\frac{k}{d}\right)\tau\eta-\gamma^2 L\eta^2\tau\left(2-\frac{k}{d}\right)\geq 0
\end{align}
which can be simplified further as follows:
\begin{align}
    1-L^2\eta^2\tau^2-mL\tau\eta\left(1-\frac{k}{d}\right)-2\gamma L\eta\tau\left(2-\frac{k}{d}\right)\geq 0
\end{align}
Then using Eq.~(\ref{eq:ncvx-mid-step}) we obtain:
\begin{align}
  \frac{\tau\gamma}{2} \left({\eta}-\frac{1}{\tau mL}\right)\left\|\nabla f({\boldsymbol{x}}^{(r)})\right\|_2^2\leq \mathbb{E}\left[\mathbb{E}_\mathbf{S}\Big[f({\boldsymbol{x}}^{(r+1)})-f({\boldsymbol{x}}^{(r)})\Big]\right]+\tau\eta^2\gamma^2 L\left(2-\frac{k}{d}\right)\frac{\sigma^2}{p}+\frac{\gamma\eta^3L^2\tau^2}{2}\sigma^2
\end{align}
which leads to the following bound:
\begin{align}
     \left\|\nabla f({\boldsymbol{x}}^{(r)})\right\|_2^2\leq \frac{2 \mathbb{E}\left[\mathbb{E}_\mathbf{S}\Big[f({\boldsymbol{x}}^{(r+1)})-f({\boldsymbol{x}}^{(r)})\Big]\right]}{\tau \gamma \left({\eta}-\frac{1}{\tau mL}\right)}+\frac{2\eta^2\gamma L\left(2-\frac{k}{d}\right)\frac{\sigma^2}{p}}{ \left({\eta}-\frac{1}{\tau mL}\right)}+\frac{\eta^3L^2\tau}{\left({\eta}-\frac{1}{\tau mL}\right)}\sigma^2 
\end{align}
Now averaging over $r$ communication rounds we achieve:
\begin{align}
    \frac{1}{R}\sum_{r=0}^{R-1}\left\|\nabla f({\boldsymbol{x}}^{(r)})\right\|_2^2\leq \frac{2 \mathbb{E}\left[\mathbb{E}_\mathbf{S}\Big[f({\boldsymbol{x}}^{(0)})-f({\boldsymbol{x}}^{(*)})\Big]\right]}{R\tau \gamma \left({\eta}-\frac{1}{\tau mL}\right)}+\frac{2\eta^2\gamma L\left(2-\frac{k}{d}\right)\frac{\sigma^2}{p}}{ \left({\eta}-\frac{1}{\tau mL}\right)}+\frac{\eta^3L^2\tau}{\left({\eta}-\frac{1}{\tau mL}\right)}\sigma^2 
\end{align}
We note that for this case we have the following conditions over learning rate:
\begin{align}
    L^2\eta^2\tau^2+mL\tau\eta\left(1-\frac{k}{d}\right)+2\gamma L\eta\tau\left(2-\frac{k}{d}\right)\leq 1,\:\eta> \frac{1}{mL\tau},
\end{align}

\subsection{Proof of Theorem~\ref{thm:pl-iid}}
From Eq.~(\ref{eq:ncvx-mid-step}) under condition with:
\begin{align}
       L^2\eta^2\tau^2+mL\tau\eta\left(1-\frac{k}{d}\right)+2\gamma L\eta\tau\left(2-\frac{k}{d}\right)\leq 1, \label{eq:step_size_cnd_mmr}
\end{align}
we obtain:
\begin{align}
         \mathbb{E}\left[f({\boldsymbol{w}}^{(r+1)})-f({\boldsymbol{w}}^{(r)})\right]&\leq -\left(\frac{\tau\eta\gamma}{2}-\frac{\gamma}{2mL}\right)\left\|\nabla f({\boldsymbol{x}}^{(r)})\right\|_2^2+\frac{\gamma\eta^3L^2\tau^2}{2}\sigma^2+\tau\eta^2\gamma^2 L\left(2-\frac{k}{d}\right)\frac{\sigma^2}{p}\nonumber\\
         &\stackrel{(PL)}{\leq} -\left({\tau\mu\eta\gamma}-\frac{\mu\gamma}{mL}\right)\left[f({\boldsymbol{w}}^{(r)})-f({\boldsymbol{w}}^{(*)})\right]+\frac{\gamma\eta^3L^2\tau^2}{2}\sigma^2+\tau\eta^2\gamma^2 L\left(2-\frac{k}{d}\right)\frac{\sigma^2}{p} 
\end{align}
which leads to the following bound:
\begin{align}
            \mathbb{E}\Big[f({\boldsymbol{w}}^{(r+1)})-f({\boldsymbol{w}}^{(*)})\Big]&\leq \left(1-\eta\mu\gamma{\tau}+\frac{\mu\gamma}{mL}\right) \Big[f({\boldsymbol{w}}^{(r)})-f({\boldsymbol{w}}^{(*)})\Big]+\frac{\gamma\eta^3L^2\tau^2}{2}\sigma^2+\tau\eta^2\gamma^2 L\left(2-\frac{k}{d}\right)\frac{\sigma^2}{p} 
\end{align}
which leads to the following bound by setting $\Delta\triangleq1-\eta\mu\gamma{\tau}+\frac{\mu\gamma}{mL}=1-\mu\gamma\tau\left(\eta-\frac{1}{mL\tau}\right)$:
\begin{align}
            \mathbb{E}\Big[f({\boldsymbol{w}}^{(R)})-f({\boldsymbol{w}}^{(*)})\Big]&\leq \Delta^R \Big[f({\boldsymbol{w}}^{(0)})-f({\boldsymbol{w}}^{(*)})\Big]+\frac{1-\Delta^R}{1-\Delta}\left(\frac{\gamma\eta^3L^2\tau^2}{2}\sigma^2+\tau\eta^2\gamma^2 L\left(2-\frac{k}{d}\right)\frac{\sigma^2}{p} \right)\nonumber\\
            &\leq \Delta^R \Big[f({\boldsymbol{w}}^{(0)})-f({\boldsymbol{w}}^{(*)})\Big]+\frac{1}{1-\Delta}\left(\frac{\gamma\eta^3L^2\tau^2}{2}\sigma^2+\tau\eta^2\gamma^2 L\left(2-\frac{k}{d}\right)\frac{\sigma^2}{p} \right)\nonumber\\
            &={\left(1-\mu\gamma\tau\left(\eta-\frac{1}{mL\tau}\right)\right)}^R \Big[f({\boldsymbol{w}}^{(0)})-f({\boldsymbol{w}}^{(*)})\Big]+\frac{\left(\frac{\gamma\eta^3L^2\tau^2}{2}\sigma^2+\tau\eta^2\gamma^2 L\left(2-\frac{k}{d}\right)\frac{\sigma^2}{p} \right)}{\mu\gamma\tau\left(\eta-\frac{1}{m L\tau}\right)}\nonumber\\
            &\leq \exp{-\left(\mu\gamma\tau\left(\eta-\frac{1}{m L\tau}\right)R\right)}\Big[f({\boldsymbol{w}}^{(0)})-f({\boldsymbol{w}}^{(*)})\Big]+\frac{\left(\frac{\gamma\eta^3L^2\tau}{2}\sigma^2+\eta^2\gamma^2 L\left(2-\frac{k}{d}\right)\frac{\sigma^2}{p} \right)}{\mu\gamma\left(\eta-\frac{1}{mL\tau}\right)}
\end{align}
Then for the choice of $\eta=\frac{n}{mL\tau}$, for $m>n>1$, we obtain:


\begin{align}
                \mathbb{E}\Big[f({\boldsymbol{w}}^{(R)})-f({\boldsymbol{w}}^{(*)})\Big]&\leq \exp{-\left(\frac{\gamma\left(n-1\right) R}{m\kappa}\right) }\Big[f({\boldsymbol{w}}^{(0)})-f({\boldsymbol{w}}^{(*)})\Big]+\frac{\left(\frac{\gamma n^3L^2\tau}{2m^3L^3\tau^3}\sigma^2+\frac{n^2}{m^2L^2\tau^2}\gamma^2 L\left(2-\frac{k}{d}\right)\frac{\sigma^2}{p} \right)}{\mu\gamma\left(\frac{n-1}{mL\tau}\right)}\nonumber\\
                &=\exp{-\left(\frac{\gamma\left(n-1\right) R}{m\kappa}\right) }\Big[f({\boldsymbol{w}}^{(0)})-f({\boldsymbol{w}}^{(*)})\Big]+\frac{\left(\frac{ n^3}{2m^2}+\frac{n^2}{m}\gamma L\left(2-\frac{k}{d}\right)\frac{1}{p} \right)}{\mu\tau\left(n-1\right)}\sigma^2
\end{align}

We note that regarding condition in Eq.~(\ref{eq:step_size_cnd_mmr}), if we let $\eta=\frac{n}{m L\tau}$ for $m>n>1$, we need to satisfy the following condition:
\begin{align}
    \frac{n^2}{m^2}+n\left(1-\frac{k}{d}\right)+\frac{2n\gamma\left(1-\frac{k}{d}\right)}{m}\leq 1
\end{align}
Now if you let $\gamma=\frac{m}{2}$, we need to impose the following condition over $k$ and $d$ as follows:
\begin{align}
    n\left(1-\frac{k}{d}\right)\leq \frac{1}{3}\implies d\left(1-\frac{1}{3n}\right)\leq k\leq d
\end{align}
\todo{Will fix these later!}

