\documentclass{article}
\usepackage{nips_2018_author_response}

\usepackage[utf8]{inputenc} % allow utf-8 input
\usepackage[T1]{fontenc}    % use 8-bit T1 fonts
\usepackage{hyperref}       % hyperlinks
\usepackage{url}            % simple URL typesetting
\usepackage{booktabs}       % professional-quality tables
\usepackage{amsfonts}       % blackboard math symbols
\usepackage{nicefrac}       % compact symbols for 1/2, etc.
\usepackage{microtype}      % microtypography

\usepackage{lipsum}


\begin{document}
\textbf{Reviewer 1:}
\begin{itemize}
\item \textit{I want authors to clarify an advantage of MISSO over existing methods as commented above.}
\end{itemize}
\textbf{Reviewer 2:}
\begin{itemize}
\item \textit{However, the authors do not compare these convergence rates/iterations with previous stochastic MM methods to show if there are any improvements over previous methods theoretically.}

\item \textit{Will the assumption H1 be satisfied for all the times? It may be a very strong constraint for the r functions to be convex in $\theta$ or to be bounded.}

\item \textit{For example, why not compare with the ADAM algorithm for the first experiment? Also, are there other recent stochastic MM algorithms that can be compared with in the experiments?}

\item \textit{ Perhaps more details for the experiments will help.}
\end{itemize}

\textbf{Reviewer 3:}
\begin{itemize}
\item \textit{Author mention in line 43 that they will present a Monte-Carlo Expectation Maximization (EM) method,
but no such example can be found (at least under this term). They probably refer to Ex 1 as EM but
surprisingly with no mention of EM throughout the section.}

\item \textit{Assumption H2 has quantities that resemble empirical processes. These constants will have bad dimension
by the least and an investigation on how they behave is required to verify that bounds are nonvacous.}

\item \textit{The expectations in Theorems should be made explicit. In particular, the left hand side in equation 22
actually reduces to a average over iterates which can only imply that the best iterate gets a sublinear
convergence rate. Authors should clarify this in the main text.}
\end{itemize}


\end{document}

